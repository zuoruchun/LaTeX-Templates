
%%%%%%%%%%%%%%%%%%%% 引言 %%%%%%%%%%%%%%%%%%%%

\chapter{前言}\label{chap:Intro}

\section{研究背景}\label{sec:background}
近年来, 时间变换随机过程和时间变换随机微分方程(SDEs)作为描述次扩散过程的重要方法之一\cite{Umarov20181}, 受到了人们广泛的关注. Meerschaert和Scheffler在\cite{Meerschaert}中对时间变换的随机过程进行了详细的研究, 将具有无限均值等待时间的连续时间随机游走与时间变换的随机过程联系起来, 并提出了一个重要的极限定理. Deng和Schilling在\cite{Deng2017}中证明了时间变换的分数布朗运动的一些重要性质和基本不等式. Kobayashi在\cite{Kobayashi2011}中证明了时间变换随机微分方程的存在唯一性定理, 并提出了许多实用的分析方法. 在\cite{wu2016stability}中, Wu研究了时间变换布朗运动驱动的随机微分方程的稳定性. Nane和Li在\text{\cite{Nane20173085, Nane2018479}}中主要研究驱动噪声为时间变换的Lévy噪声的随机微分方程解的稳定性. Zhang和Yuan在\cite{Zhang2019689}中关注时间变换的随机泛函微分方程, 并探讨了此类方程的稳定性和收敛性问题. Yin等人在\cite{Yin20212338}中研究了一类具有脉冲效应的时间变换随机微分方程的稳定性. Shen等人在\cite{Shen2023}中讨论了由时间变换布朗运动驱动的分布依赖随机微分方程的存在唯一性和稳定性等性质. Li等人在\cite{Li2023}中研究了时间变换的McKean-Vlasov随机微分方程的一些理论结果, 该方程也是一种分布依赖的随机微分方程.

时间变换过程和时间变换随机微分方程被广泛应用于金融市场建模中. 在\cite{Magdziarz2009553}中Magdziarz通过使用经典的几何布朗运动和逆$\alpha$-稳定过程推导了 次扩散的Black-Scholes公式. 此外, Magdziarz等人在\cite{Magdziarz2011187}中提出了Bachelier模型的次扩散版本, 并深入研究了其在期权定价中的应用. Janczura等人在\cite{janczura2011subordinated}中研究了由$\alpha$-稳定过程驱动的时间变换的Ornstein-Uhlenbeck过程, 并探讨了该方法在金融数据建模和预测中的应用. 在\text{\cite{Chen2017168, Hahn2012262, Magdziarz20093238, Nane2016103}}中讨论了时间变换过程与各种确定性分数阶微分方程之间的联系.

时间变换随机微分方程的数值近似至关重要的原因主要有两点: 首先, 时间变换随机微分方程的解析解很少能显式地表达出来. 其次, 在实践中,时间变换随机微分方程模型的应用通常需要大量的样本路径进行统计学习, 如参数估计、检验和预测等. 在这种情况下, 即使某些类型的时间变换随机微分方程模型的解析解的显式表达式可用, 但在没有计算机模拟的帮助下进行这些计算也几乎是不可能的.

在随机过程中, 数值求解时间变换随机微分方程的一种思路是通过离散其对应的分数阶微分方程, 关于分数阶微分方程的数值方法已有众多的研究取得了显著的成果\text{\cite{Diethelm20023,Du2012667,Li20151,Wang20232125}}. 在本篇文章中, 我们将关注时间变换随机微分方程数值逼近的另一种方法, 即直接对时间变换随机微分方程进行离散. 在该方面, Kobayashi及其合作者在方程系数的空间变量上施加全局Lipschitz条件, 并研究了不同结构的时间变换随机微分方程的多种数值方法. Jum和Kobayashi在\cite{Jum2016201}中证明了一类时间变换随机微分方程的Euler–Maruyama(EM)方法的强收敛和弱收敛. 据我们所知, 这是关于时间变换随机微分方程解的样本路径模拟的首次研究. 近年, Jin和Kobayashi在\text{\cite{Jin2019619,Jin2021829}}中研究了更一般类型的时间变换随机微分方程的Euler型和Milstein型方法. 而\cite{Jum2016201}和\text{\cite{Jin2019619,Jin2021829}}在研究方法上的一个主要区别是\cite{Kobayashi2011}中建立的对偶原理在\cite{Jum2016201}中得到了应用, 而没有应用在\text{\cite{Jin2019619,Jin2021829}}中. 简而言之, 对偶原理揭示了经典随机微分方程与时间变换随机微分方程之间的关系, 提供了一种直接通过经典随机微分方程的数值方法来构造时间变换随机微分方程的数值方法的途径. 对于时间变换的McKean-Vlasov随机微分方程, Wen等人在\cite{Wen2023}中研究了相关的数值方法.

当随机微分方程中系数出现超线性项时, 经典的Euler型和Milstein型方法可能不会收敛\cite{Hutzenthaler20111563}, 而隐式方法和改进的显式方法通常是良好的选择. 当在时间变换随机微分方程漂移系数的空间变量上施加超线性增长条件时, Deng和Liu在\cite{Deng20201133}中研究了半隐式EM方法, Liu等人则在\cite{Liu202066}中借助对偶原理研究了截断EM方法. 另一方面, Li等人\cite{Li2023651}中在未采用对偶原理的前提下同样了研究了时间变换随机微分方程的截断EM方法. Liu等人在\cite{刘暐2020}中对随机方程的截断方法进行了综述和分析. Tang在\cite{汤婧雯2021}中对一类非自治随机微分方程的截断EM方法进行分析, 并通过对偶原理将截断EM方法的强收敛结论应用于时间变换随机微分方程的数值逼近研究中. Wu在\cite{吴硕2017}中对随机微分方程的截断θ方法进行了收敛性分析.

在本文中, 我们重点研究了具有超线性增长系数的时间变换随机微分方程的数值方法. 相较于过去的研究\text{\cite{Deng20201133, Liu202066,Li2023651}}, 我们提出了一种具有截断作用的Milstein型方法抑制超线性项. 该方法与Euler型方法相比具有更高的收敛阶, 更适用于在金融应用中流行的多层蒙特卡洛方法\text{\cite{giles2006,giles2018}}.

\section{结构安排}

本文的写作安排如下:

第一章主要介绍本文的研究背景和现状、研究目的以及文章的主要结构. 通过对时间变换随机微分方程的相关研究方法进行综述, 引出本文将要研究的一类特殊的高非线性非自治时间变换随机微分方程.

第二章首先给出进行本文研究所需的符号和假设条件, 接着在非自治系统中详细介绍时间变换随机微分方程的截断Milstein方法的构造过程, 然后列举出证明本文主要结论所需的重要引理, 对各个引理进行阐述说明并对其中缺少现有结论的引理进行详细证明.

第三章详细介绍本文的主要结论并对其进行证明. 通过使用前文所列的前提条件和假设, 本章证明了时间变换随机微分方程的截断Milstein方法的强收敛阶为
min$\{\gamma_f,\gamma_g,(1-2\varepsilon)\}$, 其中$\gamma_f$和$\gamma_g$分别是时间变换随机微分方程漂移项和扩散项中时间变量的H{\"o}lder连续指数, $\varepsilon$为任意小的数, 故得出此类时间变换随机微分方程的截断Milstein的收敛阶与时间变量的光滑性有关的结论.

第四章通过Matlab数值模拟对一维和二维时间变换随机微分方程算例的理论结果进行验证, 在两个时间变换随机微分方程中取不同的时间变量H{\"o}lder连续指数, 将数值模拟所得的误差阶和本文的理论结果进行对比分析.

第五章首先对本文的主要内容和结论进行总结, 然后在本文研究的一类时间变换随机微分方程重要性质的基础上, 进一步探讨未来的研究方向. 
