
%%%%%%%%%%% 中文摘要内容和关键字  %%%%%%%%%%%%%%

\begin{cnabstract}

时间变换随机微分方程(SDEs)作为描述次扩散现象的重要工具被广泛应用于金融市场建模中, 因而关于时间变换随机微分方程数值方法的研究备受人们关注. 传统的求解方法主要包括Euler-Maruyama方法、Milstein方法等经典的显式方法. 系数满足全局Lipschitz的时间变换随机微分方程通常采用简单易实现、计算效率高的经典的显式方法进行数值逼近. 然而对一类具有超线性增长项的时间变换随机微分方程, 经典的显式方法易发散. 

本文针对一类高非线性非自治时间变换的随机微分方程, 利用截断的思想抑制超线性增长项提出了截断Milstein方法, 该方法保留了显式方法的优势. 文章证明了有限时间区间的强收敛性, 并获得了收敛阶. 

本文研究的一类时间变换随机微分方程中, 漂移项和扩散项系数中的空间变量满足超线性增长条件, 时间变量满足H{\"o}lder连续条件. 此外, 这类时间变换随机微分方程不适用对偶原理, 从而无法建立经典随机微分方程与时间变换随机微分方程之间的联系, 故只能直接对时间变换随机微分方程构造数值方法. 文中通过构造截断Milstein方法, 并证明了该数值方法的收敛阶为${\min\{\gamma_f,\gamma_g,(1-2\varepsilon)\}}$,其中$\gamma_f$和$\gamma_g$为时间变量H{\"o}lder连续指数, $\varepsilon$为任意小的数. 根据该结论可知截断Milstein的收敛阶与时间变量的光滑性有关, 当光滑性较差时, 数值方法的收敛阶由时间变量H{\"o}lder连续指数决定; 当光滑性较好时,该方法的收敛阶接近于1阶.
文章最后通过Matlab对一维和二维时间变换随机微分方程数值算例的理论结果进行验证. 
\cnkeywords{时间变换随机微分方程; 高非线性; 非自治; 截断Milstein方法; 强收敛}

\end{cnabstract}

