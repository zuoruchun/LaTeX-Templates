
%%%%%%%%%%%%%%%%%%% 表格环境 %%%%%%%%%%%%%%%%%

\chapter{结论与展望}

 由于时间变换随机微分方程在众多领域具有广泛的运用, 本文针对一类高非线性非自治的时间变换随机微分方程进行研究. 此类时间变换随机微分方程中漂移项和扩散项系数中空间变量满足超线性增长, 且时间变量满足H{\"o}lder连续的条件, 由于方程结构的特殊性, 我们并不能简单通过对偶原理由经典随机微分方程的数值方法来构造时间变换随机微分方程的数值方法. 为处理好时间变换随机微分方程高非线性条件的同时提高数值方法的收敛阶, 本文提出了截断Milstein方法, 并研究了在有限时间区间的强收敛性, 得到了强收敛阶为${\min\{\gamma_f,\gamma_g,(1-2\varepsilon)\}}$.

总的来说, 本文与现有研究的主要区别在于提出的时间变换随机微分方程不适用对偶原理, 同时处理的是非自治和高非线性情况, 构造了具有较高收敛阶的数值方法处理方程. 文章最后列举的两个数值例子验证了理论结果. 
\par
此外, 在一类高非线性非自治时间变换随机微分方程的数值分析中, 数值方法的稳定性也是一个重要的研究方向. 因为随机微分方程的数值求解中存在着随机项和数值误差的影响, 将导致问题研究更加复杂以及数值解的不稳定性, 因而探讨数值方法的稳定性至关重要. 然而这方面的研究将涉及到更复杂的符号和假设分析. 考虑本文篇幅有限, 我们将在未来的研究工作中专门探讨此类时间变换随机微分方程的截断Milstein方法的稳定性问题.