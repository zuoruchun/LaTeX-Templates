% !TEX program = xelatex
% 使用 texlive 完整编译:
% xelatex -> bibtex -> xelatex -> xelatex
% SHNU-Thesis 上海师范大学研究生毕业论文 LaTeX 模板

\documentclass[master]{shnuthesis}
% master 硕士学位论文, 默认可省略
% doctor 博士学位论文, 不能省略
% arts 文科学位论文, 默认缺省为理科
% print 用于打印, 封面等生成空白页
% 提交给图书馆的电子版不要选 print

%----- 论文标题、作者、导师、日期等相关信息 -----
\title{一类高非线性非自治时间变换随机微分方程的Milstein型方法}
\author{*~*~*}   % 作者姓名
\date{二~~〇~~二~~四~~年~~三~~月 }  % 完成日期
\institute{数~~理~~学~~院}  % 学院名称
\major{计~~算~~数~~学}  % 专业名称
\research{随~机~微~分~方~程~数~值~解}  % 研究方向
\studentID{******}   % 学号
\advisor{*~*~*}  % 导师姓名

%----- 设置英文字体 -----
%\usepackage{newtxtext}  % New TX font for text
%\setmainfont{TeX Gyre Termes}  % Times New Roman 的开源复刻版本
%\setsansfont{TeX Gyre Heros}   % Helvetica 的开源复刻版本
%\setmonofont{TeX Gyre Cursor}  % Courier New 的开源复刻版本
\setmainfont{Times New Roman}
\setsansfont{Arial}
%\setmonofont{Courier New}

%----- 设置数学字体 -----
%\usepackage{newtxmath}
%\usepackage{mathptmx}

%----- 添加其他宏包 -----
\usepackage{listings}
\usepackage{subfig}
\usepackage{graphicx}
\usepackage{amsmath}
\usepackage{amsthm}
\usepackage{amsfonts}
\usepackage{amssymb}
\usepackage{latexsym}
\usepackage{marvosym}
\usepackage{enumitem}
\usepackage{verbatim}
\usepackage{lineno}
\usepackage{float}
\usepackage{epstopdf}
\usepackage{color}
\usepackage{amsmath} 
\allowdisplaybreaks[4]
\usepackage{lineno,hyperref}
\modulolinenumbers[1]
\usepackage{tcolorbox}

%----- 取消链接颜色和方框 -----
%\hypersetup{hidelinks}

%----- 参考文献格式 -----
%\bibliographystyle{plain} % abbrv, unsrt, siam
%\bibliographystyle{shnuthesis-numeric}
%\bibliographystyle{shnuthesis-author-year}
\bibliographystyle{shnuthesis-numeric-authors}

%----- 参考文献引用格式 -----
\citestyle{numbers}
%\citestyle{super}
%\citestyle{authoryear}

%----- 调整列表项的间距 -----
%\setlength{\itemsep}{3pt}

%----- 调整页面避免出现过大空白 -----
%\raggedbottom

%----- 定义符号描述命令  -----
\newcommand{\nameditem}[3][]{
\noindent\hspace{2em}\makebox[0.2\textwidth][l]{#2}{{#3}
\hfill\makebox[0.2\textwidth][l]{#1}\hspace*{2em}}\par}

%----- 微分算子 -----
\newcommand*{\dif}{\mathop{}\!\mathrm{d}}

%----- 自定义命令 -----
\newcommand{\CC}{\ensuremath{\mathbb{C}}}
\newcommand{\RR}{\ensuremath{\mathbb{R}}}
\newcommand{\A}{\mathcal{A}}
\newcommand{\bA}{\boldsymbol{A}}
\newcommand{\ii}{\mathrm{i}\,}
\newcommand{\abs}[1]{\lvert#1\rvert}
\newcommand{\norm}[1]{\left\lVert#1\right\rVert}
\newcommand{\dx}[1][x]{\mathop{}\!\mathrm{d}#1}
\newcommand{\red}[1]{\textcolor{red}{#1}}
\newcommand{\EE}{\mathbb{E}}
\newcommand{\PP}{\mathbb{P}}
\newcommand{\SM}{\mathbb{S}}
\newcommand{\NN}{\mathbb{N}}
\newcommand{\ZZ}{\mathbb{Z}}
\newcommand{\ind}{\mathbf{1}}
\newcommand{\LL}{\mathbb{L}}
\newcommand{\KK}{\mathcal{K}}
\def\F{{\cal F}}
\def\G{{\cal G}}
\def\P{{\cal P}}

\newtheorem{expl}[theorem]{Example}
\newtheorem{rmk}[theorem]{Remark}
\newtheorem{conj}[theorem]{Conjecture}
\newtheorem{Proof}[theorem]{Proof}

\newcommand\tq{{\scriptstyle{3\over 4 }\scriptstyle}}
\newcommand\qua{{\scriptstyle{1\over 4 }\scriptstyle}}
\newcommand\hf{{\textstyle{1\over 2 }\displaystyle}}
\newcommand\hhf{{\scriptstyle{1\over 2 }\scriptstyle}}
\newcommand\hei{\tfrac{1}{8}}

\newcommand{\eproof}{\indent\vrule height6pt width4pt depth1pt\hfil\par\medbreak}

\begin{document}
\def\a{\alpha}
\def\e{\varepsilon} \def\z{\zeta} \def\y{\eta} \def\o{\theta}
\def\vo{\vartheta} \def\k{\kappa} \def\l{\lambda} \def\m{\mu} \def\n{\nu}
\def\x{\xi}  \def\r{\rho} \def\s{\sigma}
\def\p{\phi} \def\f{\varphi}   \def\w{\omega}
\def\q{\surd} \def\i{\bot} \def\h{\forall} \def\j{\emptyset}

\def\be{\beta} \def\de{\h} \def\up{\upsilon} \def\eq{\equiv}
\def\ve{\vee} \def\we{\wedge}


\def\D{\h} \def\O{\Theta} \def\L{\Lambda}
\def\X{\Xi} \def\Si{\Sigma} \def\W{\Omega}
\def\M{\partial} \def\N{\nabla} \def\Ex{\exists} \def\K{\times}
\def\V{\bigvee} \def\U{\bigwedge}

\def\1{\oslash} \def\2{\oplus} \def\3{\otimes} \def\4{\ominus}
\def\5{\circ} \def\6{\odot} \def\7{\backslash} \def\8{\infty}
\def\9{\bigcap} \def\0{\bigcup} \def\+{\pm} \def\-{\mp}
\def\la{\langle} \def\ra{\rangle}


\def\tl{\tilde}
\def\trace{\hbox{\rm trace}}
\def\diag{\hbox{\rm diag}}
\def\for{\quad\hbox{for }}
\def\refer{\hangindent=0.3in\hangafter=1}
% 生成标题页
\maketitle

% 生成独创性与授权声明页, 此页可以放在最后
\makestatement
% 将签字扫描的声明页 statement.pdf 替换原始页面
%\makestatement[file=statement.pdf]

%%%%%%%%%%%%%%%%%%%%%%%%%%%%%%%%%%%%%%%%%%%%%%%

\frontmatter

%%%%%%%%%%%%%% 中文摘要内容和关键字  %%%%%%%%%%%%%


% 中文摘要和关键字

\begin{cnabstract}

研究Back-Euler-Maruyama (BEM)数值逼近一类经过Lamperti变换后,漂移系数线性增长、扩散系数是常数的时变随机微分方程.证明了BEM的强收敛性与逆从属的稳定指数之间的关系,并讨论收敛速度.并通过数值模拟验证了理论结果.


\cnkeywords{时间变换;等距离散;收敛阶;逆从属}

\end{cnabstract}



%%%%%%%%%%%%%% 英文摘要内容和关键字 %%%%%%%%%%%%%


%%%%%%%%%%%%% 英文摘要内容和关键字 %%%%%%%%%%%%

\begin{enabstract}
Time-changed stochastic differential equations (SDEs) are widely used in financial market modeling as an important tool to describe sub-diffusion phenomena. Therefore, research on numerical methods of time-changed stochastic differential equations has attracted much attention. Traditional solution methods mainly include classic explicit methods such as Euler-Maruyama method and Milstein method. Time-changed stochastic differential equations which coefficients satisfy global Lipschitz are usually numerically approximated by classical explicit methods that are simple, easy to implement, and highly computationally efficient. However stochastic differential equations with superlinear growth terms, the classic explicit methods tend to diverge.

This paper proposes a truncated Milstein method for a class of highly nonlinear non-autonomous stochastic differential equations with time-changed processes, using the idea of truncation to suppress superlinear growth terms. This method retains the advantages of the explicit method. The article proves strong convergence in a finite time interval properties, and obtained the convergence order. 

In a class of time-changed stochastic differential equations studied in this article, the spatial variables in the drift and diffusion coefficients satisfy the superlinear growth condition, and the temporal variables satisfies the H{\"o}lder continuity condition. In addition, the duality principle does not apply to these types of time-changed stochastic differential equations, so the connection between the classical stochastic differential equations and the time-changed stochastic differential equations cannot be established, so we can only directly construct a numerical method for the time-changed stochastic differential equations. In this paper, by constructing the truncated Milstein method,
and it is proven that the convergence order of this numerical method is ${\min\{\gamma_f,\gamma_g,(1-2\varepsilon)\}}$, where $\gamma_f$ and $\gamma_g$ represents the H{\"o}lder exponents of the temporal variables and $\varepsilon$ is an arbitrarily small number. This conclusion implies that the convergence order of the truncated Milstein method is related to the smoothness of the temporal variables. When the smoothness is poor, the convergence order of the numerical method is determined by the H{\"o}lder exponents of the temporal variables. When the smoothness is good, the convergence order of this method approaches 1. At the end of the paper, the theoretical results of numerical examples of one-dimensional and two-dimensional time-changed stochastic differential equations through Matlab were verified.

\enkeywords{time-changed stochastic differential equations; highly non-linear;  non-autonomous; Milstein-type method; strong convergence}

\end{enabstract}




%%%%%%%%%%%%%%%%%%%% 目录 %%%%%%%%%%%%%%%%%%%%%%

% 生成目录
\maketoc

% 生成插图清单, 如不需要可以注释
%\makelof

% 生成表格清单, 如不需要可以注释
%\makelot

%%%%%%%%%%%%%%%%%% 主要符号表 %%%%%%%%%%%%%%%%%%

% 如不需要可以注释
\input{part/denotation}


%%%%%%%%%%%%%%%%%%%%%%%%%%%%%%%%%%%%%%%%%%%%%%%

\mainmatter

%%%%%%%%%%%%%%%%%%% 正文内容 %%%%%%%%%%%%%%%%%%%

%----- 第1章 引言 -----

%%%%%%%%%%%%%%%%%%%% 引言 %%%%%%%%%%%%%%%%%%%%

\chapter{前言}\label{chap:Intro}

\section{研究背景}\label{sec:background}
近年来, 时间变换随机过程和时间变换随机微分方程(SDEs)作为描述次扩散过程的重要方法之一\cite{Umarov20181}, 受到了人们广泛的关注. Meerschaert和Scheffler在\cite{Meerschaert}中对时间变换的随机过程进行了详细的研究, 将具有无限均值等待时间的连续时间随机游走与时间变换的随机过程联系起来, 并提出了一个重要的极限定理. Deng和Schilling在\cite{Deng2017}中证明了时间变换的分数布朗运动的一些重要性质和基本不等式. Kobayashi在\cite{Kobayashi2011}中证明了时间变换随机微分方程的存在唯一性定理, 并提出了许多实用的分析方法. 在\cite{wu2016stability}中, Wu研究了时间变换布朗运动驱动的随机微分方程的稳定性. Nane和Li在\text{\cite{Nane20173085, Nane2018479}}中主要研究驱动噪声为时间变换的Lévy噪声的随机微分方程解的稳定性. Zhang和Yuan在\cite{Zhang2019689}中关注时间变换的随机泛函微分方程, 并探讨了此类方程的稳定性和收敛性问题. Yin等人在\cite{Yin20212338}中研究了一类具有脉冲效应的时间变换随机微分方程的稳定性. Shen等人在\cite{Shen2023}中讨论了由时间变换布朗运动驱动的分布依赖随机微分方程的存在唯一性和稳定性等性质. Li等人在\cite{Li2023}中研究了时间变换的McKean-Vlasov随机微分方程的一些理论结果, 该方程也是一种分布依赖的随机微分方程.

时间变换过程和时间变换随机微分方程被广泛应用于金融市场建模中. 在\cite{Magdziarz2009553}中Magdziarz通过使用经典的几何布朗运动和逆$\alpha$-稳定过程推导了 次扩散的Black-Scholes公式. 此外, Magdziarz等人在\cite{Magdziarz2011187}中提出了Bachelier模型的次扩散版本, 并深入研究了其在期权定价中的应用. Janczura等人在\cite{janczura2011subordinated}中研究了由$\alpha$-稳定过程驱动的时间变换的Ornstein-Uhlenbeck过程, 并探讨了该方法在金融数据建模和预测中的应用. 在\text{\cite{Chen2017168, Hahn2012262, Magdziarz20093238, Nane2016103}}中讨论了时间变换过程与各种确定性分数阶微分方程之间的联系.

时间变换随机微分方程的数值近似至关重要的原因主要有两点: 首先, 时间变换随机微分方程的解析解很少能显式地表达出来. 其次, 在实践中,时间变换随机微分方程模型的应用通常需要大量的样本路径进行统计学习, 如参数估计、检验和预测等. 在这种情况下, 即使某些类型的时间变换随机微分方程模型的解析解的显式表达式可用, 但在没有计算机模拟的帮助下进行这些计算也几乎是不可能的.

在随机过程中, 数值求解时间变换随机微分方程的一种思路是通过离散其对应的分数阶微分方程, 关于分数阶微分方程的数值方法已有众多的研究取得了显著的成果\text{\cite{Diethelm20023,Du2012667,Li20151,Wang20232125}}. 在本篇文章中, 我们将关注时间变换随机微分方程数值逼近的另一种方法, 即直接对时间变换随机微分方程进行离散. 在该方面, Kobayashi及其合作者在方程系数的空间变量上施加全局Lipschitz条件, 并研究了不同结构的时间变换随机微分方程的多种数值方法. Jum和Kobayashi在\cite{Jum2016201}中证明了一类时间变换随机微分方程的Euler–Maruyama(EM)方法的强收敛和弱收敛. 据我们所知, 这是关于时间变换随机微分方程解的样本路径模拟的首次研究. 近年, Jin和Kobayashi在\text{\cite{Jin2019619,Jin2021829}}中研究了更一般类型的时间变换随机微分方程的Euler型和Milstein型方法. 而\cite{Jum2016201}和\text{\cite{Jin2019619,Jin2021829}}在研究方法上的一个主要区别是\cite{Kobayashi2011}中建立的对偶原理在\cite{Jum2016201}中得到了应用, 而没有应用在\text{\cite{Jin2019619,Jin2021829}}中. 简而言之, 对偶原理揭示了经典随机微分方程与时间变换随机微分方程之间的关系, 提供了一种直接通过经典随机微分方程的数值方法来构造时间变换随机微分方程的数值方法的途径. 对于时间变换的McKean-Vlasov随机微分方程, Wen等人在\cite{Wen2023}中研究了相关的数值方法.

当随机微分方程中系数出现超线性项时, 经典的Euler型和Milstein型方法可能不会收敛\cite{Hutzenthaler20111563}, 而隐式方法和改进的显式方法通常是良好的选择. 当在时间变换随机微分方程漂移系数的空间变量上施加超线性增长条件时, Deng和Liu在\cite{Deng20201133}中研究了半隐式EM方法, Liu等人则在\cite{Liu202066}中借助对偶原理研究了截断EM方法. 另一方面, Li等人\cite{Li2023651}中在未采用对偶原理的前提下同样了研究了时间变换随机微分方程的截断EM方法. Liu等人在\cite{刘暐2020}中对随机方程的截断方法进行了综述和分析. Tang在\cite{汤婧雯2021}中对一类非自治随机微分方程的截断EM方法进行分析, 并通过对偶原理将截断EM方法的强收敛结论应用于时间变换随机微分方程的数值逼近研究中. Wu在\cite{吴硕2017}中对随机微分方程的截断θ方法进行了收敛性分析.

在本文中, 我们重点研究了具有超线性增长系数的时间变换随机微分方程的数值方法. 相较于过去的研究\text{\cite{Deng20201133, Liu202066,Li2023651}}, 我们提出了一种具有截断作用的Milstein型方法抑制超线性项. 该方法与Euler型方法相比具有更高的收敛阶, 更适用于在金融应用中流行的多层蒙特卡洛方法\text{\cite{giles2006,giles2018}}.

\section{结构安排}

本文的写作安排如下:

第一章主要介绍本文的研究背景和现状、研究目的以及文章的主要结构. 通过对时间变换随机微分方程的相关研究方法进行综述, 引出本文将要研究的一类特殊的高非线性非自治时间变换随机微分方程.

第二章首先给出进行本文研究所需的符号和假设条件, 接着在非自治系统中详细介绍时间变换随机微分方程的截断Milstein方法的构造过程, 然后列举出证明本文主要结论所需的重要引理, 对各个引理进行阐述说明并对其中缺少现有结论的引理进行详细证明.

第三章详细介绍本文的主要结论并对其进行证明. 通过使用前文所列的前提条件和假设, 本章证明了时间变换随机微分方程的截断Milstein方法的强收敛阶为
min$\{\gamma_f,\gamma_g,(1-2\varepsilon)\}$, 其中$\gamma_f$和$\gamma_g$分别是时间变换随机微分方程漂移项和扩散项中时间变量的H{\"o}lder连续指数, $\varepsilon$为任意小的数, 故得出此类时间变换随机微分方程的截断Milstein的收敛阶与时间变量的光滑性有关的结论.

第四章通过Matlab数值模拟对一维和二维时间变换随机微分方程算例的理论结果进行验证, 在两个时间变换随机微分方程中取不同的时间变量H{\"o}lder连续指数, 将数值模拟所得的误差阶和本文的理论结果进行对比分析.

第五章首先对本文的主要内容和结论进行总结, 然后在本文研究的一类时间变换随机微分方程重要性质的基础上, 进一步探讨未来的研究方向. 


%----- 第2章 LaTeX 常用环境 -----

% LaTeX 常用环境

\chapter{准备工作}\label{chap:LaTeXEnv}

\section{符号说明}
在这篇文章中,$(\Omega,\mathcal{F},\mathbb{P})$ 表示完备概率空间 , $D=(D_t)_{t\geq0}$ 表示具有Laplace指数$\psi$,从0开始的从属,其中$\psi$的被杀率是0且具有Lévy测度$\nu;$ 即$D$是具有开始于0的càdlàg路径的一维非减Lévy过程,其Laplace变换是:
$$\mathbb{E}[e^{-sD_t}]=e^{-t\psi(s)},\quad\text{其中}\quad\psi(s)=\int\limits_0^\infty(1-e^{-sy})\:\nu(\text{d}y),\quad s>0,$$
并且 $\int_0^\infty(y\wedge1)\nu(dy) < \infty$,
我们考虑Lévy测度$\nu$是无穷的情况,即$\nu ( 0, \infty ) = \infty$,这意味着复合泊松从属不在我们的考虑范围中.令 $E=(E(t))_{t\geq0}$是$D$的逆,即:
$$E(t):=\inf\{u>0;D_u>t\},\:t\geq0.$$
我们称$E$是逆从属,注意$E$是连续且非递减的.一般我们假设$B(t)$ 和 $D(t)$ 是相互独立的. 随机过程$B(E(t))$被称作时间变换的布朗运动.我们注意到$D(t)$的跳跃部分和$E(t)$的平坦部分相互对应的.又由于$E(t)$的平坦部分,导致$B(E(t))$在这一部分也是平坦的,因此$B(E(t))$可以被理解成是一种次扩散.

令 $S=(l,r)$, 其中 $-\infty\leq l<r\leq\infty$,函数 $a,b$是$S\to S$ 的连续可微函数. 考虑下面的SDE:
$$dy(t)=a(y(t))dE(t)+b(y(t))dB(E(t)),\quad t\geq0,\quad y(0)\in S$$
并且假设它在S中有唯一强解,即
$$\mathbb{P}(y(t)\in S,\:t\geq0)=1.$$
如果 $b(x)>0$ 对所有的 $x\in S$都成立, 那么我们可以使用Lamperti变换
\begin{equation}\label{Lamperti}
	F(x)=\lambda\int^x\frac1{b(y)}dy
\end{equation}
对于某些$\lambda>0.$并且$F^{-1}:F(S)\to S$ 是被良好定义的, 令$x(t)=F(y(t))$利用\cite{umarov2018beyond}中的时间变换It\^{o}公式可以得到:
$$dx(t)=f(x(t))dE(t) + \lambda dB(E(t)) \quad t\geq0,\quad x(0)\in F(S)$$
其中
$$f(x)=\lambda\left(\frac{a(F^{-1}(x))}{b(F^{-1}(x))}-\frac12b^{\prime}(F^{-1}(x))\right),\quad x\in F(S),$$
$F(D)=(F(l),F(r)).$ 这种变换可以将扩散项的非线性项转换到漂移项中

\section{主要假设和引理}

\begin{assumption}\label{moment}
	令$T>0$,假设漂移项系数$f$是二阶连续可微的并且满足:
	\begin{equation}
		\sup\limits_{t\in[0,T]}\mathbb{E}\left|f'(x(t))\right|+
		\sup\limits_{t\in[0,T]}\mathbb{E}\left|f(x(t))'f(x(t))+
		\frac{\sigma^2}2f''(x(t))\right|<\infty.
	\end{equation}
\end{assumption}

\begin{assumption}\label{ebound}
	在有限时间内, 对于随机变量$E(t)$, 我们假设$E(t)$可以被一次函数控制, 即:
	\begin{equation}
	E(t) \leq ct^{\alpha} \qquad a.s.
	\end{equation}
	
\end{assumption}

从\cite{umarov2018beyond}引入下面的三个关于时间变换的引理.
\begin{lemma}[第一变量变换公式]\label{first}
	令 B 是一维标准的布朗运动.
	如果 $H \in L(B(t), \mathcal{F}_t)$,则 $H_{E(t-)} \in L(B_{E(t)}, \mathcal{F}_{E_t})$.
	此外,对于所有 $t \geqslant 0$,几乎处处有
	$$
	\int_0^{E_t} H_s dB(s) = \int_0^t H_{E(s-)} dB_{E(s)}.
	$$
\end{lemma}
\begin{lemma}[第二变量变换公式]\label{second}
	令 B 是一维标准的布朗运动,设 $D$ 和 $E$ 是满足 $[D \longrightarrow E]$ 或 $[D \longleftarrow E]$ 的.
	假设 $B$ 与 $E$ 同步.如果 $K \in L(B_{E(t)}, \mathcal{F}_{E_t})$,则 $(K_{D(t-)}) \in L(B(t), \mathcal{F}_{E(D_t)})$.
	此外,对于所有 $t \geqslant 0$,几乎处处有
	$$
	\int_0^t K_s dB_{E(s)} = \int_0^{E_t} K_{D(s-)} dB(s).
	$$
\end{lemma}

\begin{lemma}[It\^{o}公式]\label{ito}
	令 B 是一维标准的布朗运动. 令 $D$ 和 $E$ 满足 $[  D\longrightarrow E ]$ 或 $[  D\longrightarrow E ] .$ X是由下述SDE定义的随机过程:
	$$X(t):=\int_0^tA(s)ds+\int_0^tF(s)dE(s)+\int_0^tG(s)dB(E(s))$$
	其中 $A(s)\in$ $L( t, \mathcal{F} _{E(t)})$, $F(s)\in$ $L( E(t), \mathcal{F} _{E(t)})$, 以及 $G(s)\in$ $L( B(E(s)), \mathcal{F} _{E(t)}) .$ 如果 $f\in$ $C^2( \mathbb{R} )$, 那么
	$f(X(t))$ 是 $\mathcal{F}_{E(t)}$-半鞅, 对于所有的t $\ge$ 0,都有
	$$\begin{aligned}
		&f(X(t))-f(0)=\int_{0}^{t}f^{\prime}(X(s))A(s)ds+\int_{0}^{E(t)}f^{\prime}\left(X(D(s-))\right)F(D(s-))ds\\
		&+\int_{0}^{E(t)}f^{\prime}\big(X(D(s-))\big)G(D(s-))dB(s)+\frac{1}{2}\int_{0}^{E(t)}f^{\prime\prime}\big(X(D(s-))\big)\big\{G(D(s-))\big\}^{2}ds.
	\end{aligned}$$
\end{lemma}

下面引入离散型Gronwall不等式, 
\begin{lemma}\label{gronwall}
	令$\Delta t > 0,g_n,\lambda _n \in \mathbb{R},\eta > 0,a_1=0$,再假设$1-\eta \Delta E_j > 0$,$1 + \lambda _n > 0,n \in \mathbb{N}$,那么如果
	\begin{equation*}
		a_{n+1} \leq a_n(1+\lambda _n)+\eta a_{n+1}\Delta E_n +g_{n+1}
	\end{equation*}
	则下面的不等式成立:
	\begin{equation}
		a_n \leq \sum\limits_{j=0}^{n-1}\prod_{i=j}^{n}(1-\eta\Delta E_i)g_{j+1}\prod\limits_{l=j+1}^{n-1}(1+\lambda _l)
	\end{equation}
\end{lemma}

下面的引理对于证明本文主要定理起着关键性的作用.

\begin{lemma}
	
		如果 $E$ 是从属过程 $D$ 的逆,其拉普拉斯指数 $\psi$ 在无穷大处的正则变化指数是 $\beta \in [0, 1)$ 。如果 $\beta = 0$,进一步假设 $\nu(0, \infty) = \infty$。固定 $\lambda > 0$, $t > 0$ 和 $r > 0$。
		\begin{enumerate}
			\item[(1)] 如果 $r < \frac{1}{1 - \beta}$,则 $\mathbb{E}\left[ e^{\lambda E_t^r} \right] < \infty$。
			\item[(2)] 如果 $r > \frac{1}{1 - \beta}$,则 $\mathbb{E}\left[ e^{\lambda E_t^r} \right] = \infty$。
		\end{enumerate}
	
	
\end{lemma}



\begin{lemma}\label{main lemma}
	对于任意给定的$0 = t_0 < ih < t_1 < t_2 < \ldots <t_n <(i+1)h$,都有:
	\begin{equation}
		\mathbb{E}\left[\int_{ih}^{(i+1)h}
		\int_{ih}^{t_n} \ldots \int_{ih}^{t_2} 1 dE(t_1) \ldots dE(t_{n-1})dE(t_n)\right] \le Ch^{1+(n-1)\beta}
	\end{equation}
	,其中C是与$h$无关的常数.
\end{lemma}


\begin{proof}    
	现在,在 $[0, \infty)$ 上引入随机测度 $\Pi$,定义为 $\Pi((s, t]) = E(t) - E(s)$,其中 $t > s \geq 0$.令 $\{C(t)\}_{t \geq 0}$ 为由 $\Pi$ 驱动的Cox过程,即在条件 $\Pi = \lambda$ 下,$\{C(t)\}$ 的分布与强度为 $\lambda$ 的非齐次泊松过程等价.注意,根据\cite{kingman1964doubly},$\{C(t)\}$ 是具有更新函数的更新过程:
	\begin{equation}
		u(t) = \mathbb{E}[C(t)] = \mathbb{E}[E(t)] = \frac{t^\alpha}{\Gamma(\alpha+1)}
	\end{equation}
	对于更新过程$C(t)$,参见\cite{daley2003introduction},可以得到
	\begin{equation*}
		\mathbb{E}[\mathrm{d}C(t_n)\ldots\mathrm{d}C(t_1)] = \prod_{i=1}^n u^{\prime}(t_i - t_{i-1})\mathrm{d}t_i
	\end{equation*}
	其中$0 = t_0 < t_1 < t_2 < \ldots <t_n$. 由于 Cox 过程$C(t)$的阶乘矩等于其驱动测度$\Pi$的普通矩,参见\cite{daley2003introduction}我们得到
	\begin{equation*}
		\mathbb{E}[\mathrm dE(t_n)\ldots\mathrm dE(t_1)]=\prod_{i=1}^nu'(t_i-t_{i-1})\mathrm dt_i.
	\end{equation*}
	因此:
	\begin{align*}
		I &= \mathbb{E}\left[\int_{ih}^{(i+1)h}
		\int_{ih}^{t_n}\int_{ih}^{t_{n-1}} \ldots \int_{ih}^{t_{2}} 1 dE(t_1) \ldots dE(t_{n-2})dE(t_{n-1})dE(t_n)\right] \\
		& = \int_{ih}^{(i+1)h}\int_{ih}^{t_n}\int_{ih}^{t_{n-1}}
		\ldots \int_{ih}^{t_{2}} 1 \mathbb{E}\left[dE(t_1) \ldots dE(t_{n-2})dE(t_{n-1})dE(t_n)\right] \\
		& = \frac{\alpha^n}{\Gamma^n(\alpha+1)}
		\int_{ih}^{(i+1)h}\int_{ih}^{t_n}\int_{ih}^{t_{n-1}} \ldots \int_{ih}^{t_{2}} \prod_{i=1}^{i=n}(t_i-t_{i-1})^{\alpha -1} dt_1 \ldots dt_{n-1}dt_n
	\end{align*}
	下面单独考虑积分项:
	\begin{equation*}
		I_{1}=\int_{ih}^{t_{2}} (t_{2}-t_1)^{\alpha -1} t_1^{\alpha - 1} dt_1 \\
	\end{equation*}
	做如下变换,令$t_{1} = ih + s_{1}h$,同时$t_2 = ih + s_2h ,$,其中$h$是步长,因此$s_1,s_{2} \in [0,1]$,注意由于我们不考虑时间在原点处,因此这里的$i=\frac{T}{h}$,这里的$T$是一个时间范围,于是
	\begin{align*}
		I_1 &= \int_{0}^{s_{2}} (s_{2}-s_{1})^{\alpha -1}h^{\alpha -1} (ih + s_1h)^{\alpha - 1}h ds_1 \\
		&= h^{\alpha}\int_{0}^{s_{2}} (s_{2}-s_{1})^{\alpha -1} (ih + s_1h)^{\alpha - 1} ds_1
	\end{align*}
	由于$(ih + s_1h)^{\alpha - 1}$关于$s_1$在$[0,1]$是单调递减的,并且积分$I_n$中,被积函数和积分区域都是正的,因此
	\begin{align*}
		I_1 &\le h^{\alpha}\int_{0}^{s_{2}} (s_{2}-s_{1})^{\alpha -1} (ih)^{\alpha - 1} ds_1 \\
		&=  T^{\alpha - 1}h^{\alpha}\int_{0}^{s_{2}} (s_{2}-s_{1})^{\alpha -1} ds_1
	\end{align*}
	令$w_1=s_{2}-s_{1}$,于是
	\begin{equation*}
		I_1\le T^{\alpha - 1}h^{\alpha}\int_{0}^{s_{2}} (s_{2}-s_{1})^{\alpha -1} ds_1
		=  T^{\alpha - 1}h^{\alpha}\int_{0}^{s_{n2}} w_1^{\alpha -1} dw_1
		=  \frac{T^{\alpha - 1}s_{2}^\alpha}{\alpha}h^{\alpha}
	\end{equation*}
	因此:
	\begin{equation*}
		I \le Ch^\alpha
		\int_{ih}^{(i+1)h}\int_{ih}^{t_n}\int_{ih}^{t_{n-1}} \ldots \int_{ih}^{t_{3}} 
		\prod_{i=3}^{n}(t_i-t_{i-1})^{\alpha -1} dt_{2} \ldots dt_{n-1}dt_n
	\end{equation*}
	同理,分析如下积分:
	\begin{equation*}
		I_{2} = \int_{ih}^{t_{3}}(t_{3}-t_{2})^{\alpha -1}
		dt_{2} \le Ch^\alpha 
	\end{equation*}
	因此:
	\begin{equation*}
		I \le Ch^{2\alpha}
		\int_{ih}^{(i+1)h}\int_{ih}^{t_n}\int_{ih}^{t_{n-1}} \ldots \int_{ih}^{t_{4}} 
		\prod_{i=4}^{n}(t_i-t_{i-1})^{\alpha -1} dt_{3} \ldots dt_{n-1}dt_n
	\end{equation*}
	如此进行迭代,我们得到:
	\begin{equation*}
		I \le Ch^{(n-1)\alpha}\int_{ih}^{(i+1)h} 1 dt_1 \le Ch^{1+(n-1)\alpha}
	\end{equation*}
\end{proof}



%----- 第3章 微分方程的数值方法 -----

%%%%%%%%%%%%%% 微分方程的数值方法 %%%%%%%%%%%%%

\chapter{截断Milstein方法的强收敛性}

本章我们研究以下形式的非自治时间变换随机微分方程:
\begin{align*}
    dY(t) = f(t,Y(t))dE(t) + g(t,Y(t))dW(E(t)),
\end{align*}
其中漂移项$f$和扩散项$g$中的空间变量满足超线性增长条件, 时间变量满足H{\"o}lder连续条件. 证明截断Milstein方法在有限时间内的强收敛性, 并获得收敛率.

\section{主要结论及推论}
\begin{theorem}
    \label{theorem3-2}
    若假设\ref{ass1}、\ref{ass2}、\ref{ass3} 和\ref{ass4} 对 $q>2(\a+1)p$成立, 那么对任意 $\bar{p}\in [2,p)$ 和 $h \in (0,1]$, $C$是一个常数且 $\l > 0$, 有
    \begin{align}
        \label{th321}
        &\EE \bigg( \sup_{0\leq t\leq T}|Y(t)-X(t)|^{\bar{p}} \bigg) \nonumber\\
        \leq& C\bigg(h^{\gamma_f\bar{p}}+h^{\gamma_g\bar{p}}+h^{\bar{p}}(\k(h))^{2\bar{p}}+(\m^{-1}(\k(h)))^{(\a+1)\bar{p}-q}\bigg)
    \end{align}
    和
    \begin{align}
        \label{th322}
        &\EE\bigg(\sup_{0\leq t\leq T}|Y(t)-\bar{X}(t)|^{\bar{p}}\bigg)\nonumber\\
        \leq& C\bigg(h^{\gamma_f\bar{p}}+h^{\gamma_g\bar{p}}+h^{\bar{p}}(\k(h))^{2\bar{p}}+(\m^{-1}(\k(h)))^{(\a+1)\bar{p}-q}\bigg).
    \end{align}
\end{theorem}
通过加强假设\ref{ass3}的条件, 并选择特定的$\mu(\cdot)$ 和 $\kappa (\cdot)$, 我们可以得到以下推论, 其更清晰地展示了截断Milstein方法的强收敛速率.
\begin{corollary}
    \label{theorem3-1}
    若假设\ref{ass1}、\ref{ass2} 和 \ref{ass4} 成立, 且假设\ref{ass3}对任意 $q>2$成立, 那么对任意 $\bar{p}\in [2,p)$ 和 $\varepsilon\in(0,\frac{1}{4}]$, 存在一个常数C对任意 $h \in (0,1]$和$\l > 0$, 有
    \begin{align}
        \label{th311}
        \EE\left( \sup_{0\leq t\leq T}|Y(t)-X(t)|^{\bar{p}}\right)\leq h^{\min\{\gamma_f\bar{p},\gamma_g\bar{p},(1-2\varepsilon)\bar{p}\}}
    \end{align}
    和
    \begin{align}
        \label{th312}
        \EE\left( \sup_{0\leq t\leq T}|Y(t)-\bar{X}(t)|^{\bar{p}}\right)\leq Ch^{\min\{\gamma_f\bar{p},\gamma_g\bar{p},(1-2\varepsilon)\bar{p}\}}.
    \end{align}
\end{corollary}
\begin{remark}
    推论\ref{theorem3-1}表明, 对于时间变量的某些光滑系数而言,截断Milstein方法的强收敛率可以无限接近于1, 但随着光滑系数的减小, 收敛速度也会相应减缓.
\end{remark}
\begin{remark}
    对此类高非线性非自治时间变换随机微分方程的研究, 相比于Li在\cite{Li2023651}中研究的截断Euler-Maruyama方法, 本文研究的截断Milstein方法提高了收敛阶.
\end{remark}
\section{强收敛率证明}
本节我们将证明主要定理$\ref{theorem3-2}$.
\begin{proof}
固定任意$\bar{p}\in [2,p)$和$h\in(0,1]$, 对$t\geq 0$令$e(t)=Y(t)-X(t)$, 对每个整数 $\ell> |Y(0)|$定义停时
\begin{align}
    \theta_{\ell}=\inf\{t\geq0:|Y(t)|\vee|X(t)|\geq \ell\},
\end{align}
这里, 我们设$\inf\emptyset=\infty$ (通常情况下$\emptyset$表示空集). 根据It\^o 公式, 对任意$0\leq t\leq T$有
\begin{align}
    \label{error}
    |e(t\wedge \theta_{\ell})|^{\bar{p}}&=\int^{t\wedge\theta_{\ell}}_{0}\bigg(\bar{p}|e(s)|^{\bar{p}-1}\left(f(s,Y(s))-f_{h}(\bar{\tau}(s),\bar{X}(s))\right)\nonumber\\
    &\quad +\frac{\bar{p}(\bar{p}-1)}{2}|e(s)|^{\bar{p}-2}\big|g(s,Y(s))-g_{h}(\bar{\tau}(s),\bar{X}(s))\nonumber\\
    &\quad
    -Lg_{h}(\bar{\tau}(s),\bar{X}(s))\Delta W(E_h(s))\big|^2 \bigg)dE(s)+M_{t\wedge \theta_{\ell}},
\end{align}
其中
\begin{align*}
    M_{t\wedge \theta_{\ell}}:= &\int^{t\wedge\theta_{\ell}}_{0}\bar{p}|e(s)|^{\bar{p}-1}\big|g(s,Y(s))-g_{h}(\bar{\tau}(s),\bar{X}(s))\\
    &-Lg_{h}(\bar{\tau}(s),\bar{X}(s))\Delta W(E_h(s))\big|dW(E(s)).
\end{align*}
由于随机积分$(M_t)_{t\geq 0}$是一个局部鞅, 其二次变差为
\begin{align*}
    [M,M]_{t\wedge \theta_{\ell}}=&\int^{t\wedge\theta_{\ell}}_{0}\bar{p}^2|e(s)|^{2\bar{p}-2}\big|g(s,Y(s))-g_{h}(\bar{\tau}(s),\bar{X}(s))\\
    &-Lg_{h}(\bar{\tau}(s),\bar{X}(s))\Delta W(E_h(s))\big|^2 dE(s).
\end{align*}
对$0\leq s\leq t\wedge\theta_{\ell}$有
\begin{align*}
    \begin{split}
        &\bar{p}^2|e(s)|^{2\bar{p}-2}\big|g(s,Y(s))-g_{h}(\bar{\tau}(s),\bar{X}(s))-Lg_{h}(\bar{\tau}(s),\bar{X}(s))\Delta W(E_h(s))\big|^2\\
        =&\bar{p}^2|e(s)|^{\bar{p}}|e(s)|^{\bar{p}-2}\big|g(s,Y(s))-g_{h}(\bar{\tau}(s),\bar{X}(s))-Lg_{h}(\bar{\tau}(s),\bar{X}(s))\Delta W(E_h(s))\big|^2\\
        \leq& \bar{p}^2(\sup_{0\leq r\leq t\wedge\theta_{\ell}}|e(r)|^{\bar{p}})|e(s)|^{\bar{p}-2}\big|g(s,Y(s))-g_{h}(\bar{\tau}(s),\bar{X}(s))\\
        &-Lg_{h}(\bar{\tau}(s),\bar{X}(s))\Delta W(E_h(s))\big|^2.
    \end{split}
\end{align*}
因此, 对任意$a,b>0$和$\l>0$,我们使用不等式$(ab)^{1/2}\leq a/\l+\l b$, 取 $\l=2\bar{p}$有
\begin{align}
    \label{mt}
    &([M,M]_{t\wedge \theta_{\ell}})^{1/2}\nonumber\\
    \leq &\bar{p}\bigg(\sup_{0\leq r\leq t\wedge\theta_{\ell}}|e(r)|^{\bar{p}}\int^{t\wedge\theta_{\ell}}_{0}|e(s)|^{\bar{p}-2}\big|g(s,Y(s))-g_{h}(\bar{\tau}(s),\bar{X}(s))\nonumber\\
    \quad&
    -Lg_{h}(\bar{\tau}(s),\bar{X}(s))\Delta W(E_h(s))\big|^2dE(s)\bigg)^{\frac{1}{2}}\nonumber\\
    \leq & \bar{p}\bigg(\frac{\sup_{0\leq r\leq t\wedge\theta_{\ell}}|e(r)|^{\bar{p}}}{2\bar{p}}+2\bar{p}\int^{t\wedge\theta_{\ell}}_{0}|e(s)|^{\bar{p}-2}\big|g(s,Y(s))\nonumber\\
    \quad&
    -g_{h}(\bar{\tau}(s),\bar{X}(s))-Lg_{h}(\bar{\tau}(s),\bar{X}(s))\Delta W(E_h(s))\big|^2dE(s)\bigg)\nonumber\\
    \leq &\frac{1}{2}\sup_{0\leq r\leq t\wedge\theta_{\ell}}|e(r)|^{\bar{p}}+2\bar{p}^{2}\int^{t\wedge\theta_{\ell}}_{0}\bigg(|e(s)|^{\bar{p}-2}\big|g(s,Y(s))\nonumber\\
    \quad&
    -g_{h}(\bar{\tau}(s),{X}(s))+\tilde{R}_{g_h}(s,X(s),\bar{X}(s))\big|^2\bigg)dE(s),
\end{align}
这里最后一个不等式使用了\eqref{the2_12}.接下来, 对\eqref{mt}取期望有
\begin{align}
    \EE_W(M_{t\wedge \theta_{\ell}})=&\EE_W([M,M]_{t\wedge \theta_{\ell}})^{\frac{1}{2}}\nonumber\\
    =&\EE_W\bigg(\frac{1}{2}\sup_{0\leq r\leq t\wedge\theta_{\ell}}|e(r)|^{\bar{p}}+2\bar{p}^{2}\int^{t\wedge\theta_{\ell}}_{0}|e(s)|^{\bar{p}-2}\big|g(s,Y(s))\nonumber\\
    &-g_{h}(\bar{\tau}(s),{X}(s))+\tilde{R}_{g_h}(s,X(s),\bar{X}(s))\big|^2dE(s)\bigg).
\end{align}
联立\eqref{error}和\eqref{mt}可得
\begin{align*}
    \label{err226}
    &\EE_W\left(\sup_{0\leq t\leq T}|e(t\wedge \theta_{\ell})|^{\bar{p}}\right)\\
    \leq&\EE_W\bigg(\sup_{0\leq t\leq T}\int^{t\wedge\theta_{\ell}}_{0}\bar{p}|e(s)|^{\bar{p}-2}\bigg(|e(s)|^{\mathrm{T}}\left(f(s,Y(s))-f_{h}(\bar{\tau}(s),\bar{X}(s))\right)\\
    \quad& +\frac{\bar{p}-1}{2}\big|g(s,Y(s))-g_{h}(\bar{\tau}(s),\bar{X}(s))-Lg_{h}(\bar{\tau}(s),\bar{X}(s))\Delta W(E_h(s))\big|^2 \bigg)dE(s)\\
    \quad&
    +\frac{1}{2}\sup_{0\leq r\leq t\wedge\theta_{\ell}}|e(r)|^{\bar{p}}+2\bar{p}^{2}\sup_{0\leq t\leq T}\int^{t\wedge\theta_{\ell}}_{0}|e(s)|^{\bar{p}-2}\big|g(s,Y(s))-g_{h}(\bar{\tau}(s),X(s))\\
    \quad&
    +\tilde{R}_{g_h}(s,X(s),\bar{X}(s))\big|^2dE(s)\bigg)\\
    \leq&\EE_W\bigg(\sup_{0\leq t\leq T}\int^{t\wedge\theta_{\ell}}_{0}\bar{p}|e(s)|^{\bar{p}-2}\bigg(|e(s)|^{\mathrm{T}}\left(f(s,Y(s))-f_{h}(\bar{\tau}(s),\bar{X}(s))\right)\\
    \quad& +\frac{\bar{p}-1}{2}\big|g(s,Y(s))-g_{h}(\bar{\tau}(s),X(s))+\tilde{R}_{g_h}(s,X(s),\bar{X}(s))\big|^2\bigg)dE(s)\\
    \quad&
    +\frac{1}{2}\sup_{0\leq r\leq t\wedge\theta_{\ell}}|e(r)|^{\bar{p}}+2\bar{p}^{2}\sup_{0\leq t\leq T}\int^{t\wedge\theta_{\ell}}_{0}|e(s)|^{\bar{p}-2}\big|g(s,Y(s))-g_{h}(\bar{\tau}(s),X(s))\\
    \quad&
    +\tilde{R}_{g_h}(s,X(s),\bar{X}(s))\big|^2dE(s)\bigg).
\end{align*}
这里第二项应用了\eqref{the2_12}. 然后整理上述方程可得
\begin{align}
    &\EE_W\left(\sup_{0\leq t\leq T}|e(t)\wedge \theta_{\ell})|^{\bar{p}}\right)\nonumber\\
    \leq&\EE_W\bigg(\sup_{0\leq t\leq T}\int^{t\wedge\theta_{\ell}}_{0}\bar{p}|e(s)|^{\bar{p}-2}\bigg(|e(s|^{\mathrm{T}}\left(f(s,Y(s))-f_{h}(\bar{\tau}(s),\bar{X}(s))\right)\nonumber\\
    \quad& +(\bar{p}-1)|g(s,Y(s))-g_{h}(\bar{\tau}(s),X(s))|^{2}\nonumber\\
    \quad &+(\bar{p}-1)
    |\tilde{R}_{g_h}(s,X(s),\bar{X}(s))|^2\bigg)dE(s)+\frac{1}{2}\sup_{0\leq r\leq t\wedge\theta_{\ell}}|e(r)|^{\bar{p}}\nonumber\\
    \quad &+\bar{p}|e(s)|^{\bar{p}-2}\sup_{0\leq t\leq T}\int^{t\wedge\theta_{\ell}}_{0}\bigg(4\bar{p}|g(s,Y(s))-g_{h}(\bar{\tau}(s),X(s))|^{2}\nonumber\\
    \quad&
    +4\bar{p}|\tilde{R}_{g_h}(s,X(s),\bar{X}(s))|^2\bigg)dE(s)\bigg)\nonumber\\
    \leq&\EE_W\bigg(\sup_{0\leq t\leq T}\int^{t\wedge\theta_{\ell}}_{0}\bar{p}|e(s)|^{\bar{p}-2}\bigg(|e(s)|^{\mathrm{T}}\left(f(s,Y(s))-f_{h}(\bar{\tau}(s),\bar{X}(s))\right)\nonumber\\
    \quad& +(5\bar{p}-1)|g(s,Y(s))-g_{h}(\bar{\tau}(s),X(s))|^{2}\bigg)dE(s)+\frac{1}{2}\sup_{0\leq r\leq t\wedge\theta_{\ell}}|e(r)|^{\bar{p}}\nonumber\\
    \quad&
    +\bar{p}(5\bar{p}-1)\sup_{0\leq t\leq T}\int^{t\wedge\theta_{\ell}}_{0}|e(s)|^{\bar{p}-2} |\tilde{R}_{g_h}(s,X(s),\bar{X}(s))|^2dE(s)\bigg).   
\end{align}
上式推导中, 最后两项通过使用基本不等式$(a+b)^{2}\leq2(a^{2}+b^{2})$并进行合并得到.
接下来, 我们通过整理方程, 对任意$a,b\geq 0$和$ \varepsilon>0$使用Young不等式$(a+b)^{2}\leq (1+\varepsilon)a^{2}+(1+1/\varepsilon)b^{2}$, 并在第二项中选择$ \varepsilon =(5p-5\bar{p})/(5\bar{p}-1)$, 故能从\eqref{err226}中得到
\begin{align}
    \label{err2}
    &\EE_W\left(\sup_{0\leq t\leq T}|e(t\wedge \theta_{\ell})|^{\bar{p}}\right)\nonumber\\
    \leq&\EE_W\bigg(\sup_{0\leq t\leq T}\int^{t\wedge\theta_{\ell}}_{0}\bar{p}|e(s)|^{\bar{p}-2}\bigg(|e(s)|^{\mathrm{T}}\left(f(s,Y(s))-f_{h}(\bar{\tau}(s),\bar{X}(s))\right)\nonumber\\
    \quad& +(5\bar{p}-1)\big|g(s,Y(s))-g(s,X(s))+g(s,X(s))\nonumber\\
    \quad&
    -g_{h}(\bar{\tau}(s),X(s))\big|^{2}\bigg)dE(s)+\frac{1}{2}\sup_{0\leq r\leq t\wedge\theta_{\ell}}|e(r)|^{\bar{p}}\nonumber\\
    \quad&+(5{\bar{p}}^{2}-\bar{p})\sup_{0\leq t\leq T}\int^{t\wedge\theta_{\ell}}_{0}|e(s)|^{\bar{p}-2}|\tilde{R}_{g_h}(s,X(s),\bar{X}(s))|^2dE(s)\bigg)\nonumber\\
    \leq&\EE_W\bigg(\sup_{0\leq t\leq T}\int^{t\wedge\theta_{\ell}}_{0}\bar{p}|e(s)|^{\bar{p}-2}\bigg(|e(s)|^{\mathrm{T}}\left(f(s,Y(s))-f_{h}(\bar{\tau}(s),\bar{X}(s))\right)\nonumber\\
    \quad& +(5\bar{p}-1)\bigg((1+\dfrac{5p-5\bar{p}}{5\bar{p}-1})|g(s,Y(s))-g(s,X(s))|^{2}\nonumber\\
    \quad&
    +(1+\dfrac{5\bar{p}-1}{5p-5\bar{p}})|g(s,X(s))-g_{h}(\bar{\tau}(s),X(s))|^{2}\bigg)dE(s)\nonumber\\
    \quad&
    +\frac{1}{2}\sup_{0\leq r\leq t\wedge\theta_{\ell}}|e(r)|^{\bar{p}}+(5{\bar{p}}^{2}-\bar{p})\sup_{0\leq t\leq T}\int^{t\wedge\theta_{\ell}}_{0}|e(s)|^{\bar{p}-2}\nonumber\\
    \quad& \times|\tilde{R}_{g_h}(s,X(s),\bar{X}(s))|^2dE(s)\bigg)\nonumber\\
    \leq&\EE_W\bigg(\sup_{0\leq t\leq T}\int^{t\wedge\theta_{\ell}}_{0}\bar{p}|e(s)|^{\bar{p}-2}\bigg(|e(s)|^{\mathrm{T}}\left(f(s,Y(s))-f_{h}(\bar{\tau}(s),\bar{X}(s))\right)\nonumber\\
    \quad& +(5p-1)|g(s,Y(s))-g(s,X(s))|^{2}
    +\dfrac{5p-1}{5p-5\bar{p}}|g(s,X(s))\nonumber\\
    \quad&
    -g_{h}(\bar{\tau}(s),X(s))|^{2}\bigg)dE(s)+\frac{1}{2}\sup_{0\leq r\leq t\wedge\theta_{\ell}}|e(r)|^{\bar{p}}\nonumber\\
    \quad&+(5{\bar{p}}^{2}-\bar{p})\sup_{0\leq t\leq T}\int^{t\wedge\theta_{\ell}}_{0}|e(s)|^{\bar{p}-2}|\tilde{R}_{g_h}(s,X(s),\bar{X}(s))|^2dE(s)\bigg).
\end{align}
根据不等式的基本性质, 从\eqref{err2}可得
\begin{align}
    \label{sup}
    &\EE_W\left(\sup_{0\leq t\leq T}|e( t\wedge\theta_{\ell})|^{\bar{p}}\right)\nonumber\\
    \leq& \frac{1}{2}\sup_{0\leq r\leq t\wedge\theta_{\ell}}|e(r)|^{\bar{p}}
    +\EE_W\sup_{0\leq t\leq T}\int^{t\wedge\theta_{\ell}}_{0}\bar{p}|e(s)|^{\bar{p}-2}\bigg(e^{\mathrm{T}}(s)\big(f(s,Y(s))\nonumber\\
    &-f(s,X(s))\big)
    +(5p-1)|g(s,Y(s))-g(s,X(s))|^2\bigg)dE(s)\nonumber\\
    &+\EE_W\sup_{0\leq t\leq T}\int^{t\wedge\theta_{\ell}}_{0}\bar{p}|e(s)|^{\bar{p}-2}\bigg(e^{\mathrm{T}}(s)\big(f(s,X(s))-f_{h}(\bar{\tau}(s),X(s))\big)\nonumber\\
    \quad &+\dfrac{5p-1}{5p-5\bar{p}}|g(s,X(s))-g_{h}(\bar{\tau}(s),X(s))|^2\bigg)dE(s)\nonumber\\
    &+\EE_W\sup_{0\leq t\leq T}\int^{t\wedge\theta_{\ell}}_{0}(5\bar{p}^{2}-p)|e(s)|^{\bar{p}-2}|\tilde{R}_{g_h}(s,X(s),\bar{X}(s))|^2dE(s)\nonumber\\
    \leq& \frac{1}{2}\sup_{0\leq r\leq t\wedge\theta_{\ell}}|e(r)|^{\bar{p}}+[J_1]+[J_2]+[J_3],
\end{align}
其中
\begin{align*}
    J_1&:=\EE_W \bigg(\sup_{0\leq t\leq T}\int^{t\wedge\theta_{\ell}}_{0}\bar{p}|e(s)|^{\bar{p}-2}\bigg(e^{\mathrm{T}}(s)\big(f(s,Y(s))-f(s,X(s))\big)\\
    &\quad+(5p-1)|g(s,Y(s))-g(s,X(s))|^2\bigg)dE(s)\bigg),
\end{align*}
\begin{align*}
    J_2&:=\EE_W\bigg(\sup_{0\leq t\leq T}\int^{t\wedge\theta_{\ell}}_{0}\bar{p}|e(s)|^{\bar{p}-2}\bigg(e^{\mathrm{T}}(s)\left(f(s,X(s))-f_{h}(\bar{\tau}(s),\bar{X}(s))\right)\\
    &\quad+\frac{5p-1}{5p-5\bar{p}}|g(s,X(s))-g_{h}(\bar{\tau}(s),X(s))|^2\bigg)dE(s)\bigg),
\end{align*}
\begin{align*}
    \begin{split}
        J_3&:=\EE_W\bigg(\sup_{0\leq t\leq T}\int^{t\wedge\theta_{\ell}}_{0}(5\bar{p}^{2}-p)|e(s)|^{\bar{p}-2}|\tilde{R}_{g_h}(s,X(s),\bar{X}(s))|^2dE(s)\bigg).
    \end{split}
\end{align*}
根据假设\ref{ass2}可推导出
\begin{align}
    \label{J1}
    J_1\leq H_1\int^{T}_{0}\EE_W|e(s)|^{\bar{p}}dE(s),
\end{align}
其中$H_1=\bar{p}K$. 接下来, 我们处理 $J_{2}$.
\begin{align}
    \label{lemm330}
    J_2&= \EE_W\bigg(\sup_{0\leq t\leq T}\int^{t\wedge\theta_{\ell}}_{0}\bar{p}|e(s)|^{\bar{p}-2}\bigg(e^{T}(s)\big(f(s,X(s))-f_h(\bar{\tau}(s),\bar{X}(s))\big)\nonumber\\
    &\quad+\dfrac{5p-1}{5p-5\bar{p}}|g(s,X(s))-g_h(\bar{\tau}(s),X(s))|^2\bigg)dE(s)\bigg)\nonumber\\
    &\leq \EE_W\bigg(\sup_{0\leq t\leq T}\int^{t\wedge\theta_{\ell}}_{0}\bar{p}|e(s)|^{\bar{p}-2}\bigg(e^{T}(s)\big(f(s,X(s))-f(\bar{\tau}(s),X(s))\big)\nonumber\\
    &\quad+\dfrac{5p-1}{5p-5\bar{p}}|g(s,X(s))-g(\bar{\tau}(s),X(s))|^2\bigg)dE(s)\nonumber\\
    &\quad+\sup_{0\leq t\leq T}\int^{t\wedge\theta_{\ell}}_{0}\bar{p}|e(s)|^{\bar{p}-2}\bigg(e^{T}(s)\big(f(\bar{\tau}(s),X(s))-f_h(\bar{\tau}(s),\bar{X}(s))\big)\nonumber\\
    &\quad+\dfrac{5p-1}{5p-5\bar{p}}|g(\bar{\tau}(s),X(s))-g_h(\bar{\tau}(s),X(s))|^2\bigg)dE(s)\bigg)\nonumber\\
    &\leq J_{21}+J_{22},
\end{align}
通过使用假设\ref{ass4}和基本不等式的性质, Young不等式即对任意$0\leq t\leq t\wedge\theta_{\ell}\leq T $,
\begin{align*}
    a^{p-2}b\leq \frac{p-2}{p}a^p + \frac{2}{p}b^{p/2},\quad \forall a,b \geq 0.
\end{align*}
可以推出
\begin{align}
    \label{lemm331}
    J_{21}& =
    \EE_W\bigg(\sup_{0\leq t\leq T}\int^{t\wedge\theta_{\ell}}_{0}\bar{p}|e(s)|^{\bar{p}-2}\bigg(\frac{1}{2}|e(s)|^2+\frac{1}{2}|f(s,X(s))-f(\bar{\tau}(s),X(s))|^2\nonumber\\
    &\quad+\dfrac{5p-1}{5p-5\bar{p}}|g(s,X(s))-g(\bar{\tau}(s),X(s))|^2\bigg)dE(s)\bigg)\nonumber\\
    &\leq
    C\bigg(\EE_W\sup_{0\leq t\leq T}\bigg(\int^{t\wedge\theta_{\ell}}_{0}|e(s)|^{\bar{p}}dE(s)+ \int^{t\wedge\theta_{\ell}}_{0}|f(s,X(s))-f(\bar{\tau}(s),X(s))|^{\bar{p}}dE(s)\nonumber\\
    &\quad +\int^{t\wedge\theta_{\ell}}_{0}|g(s,X(s))-g(\bar{\tau}(s),X(s))|^{\bar{p}}\bigg)dE(s)\bigg)\nonumber\\
    &\leq
    C\bigg(\EE_W\int^{T}_{0}|e(s)|^{\bar{p}}dE(s)
    +\EE_W\int^{T}_{0}H_1^{\bar{p}}(1+|X(s)|^{(1+\a)\bar{p}})h ^{\gamma_f\bar{p}}dE(s)\nonumber\\
    &\quad  +\EE_W\int^{T}_{0}H_2^{\bar{p}}(1+|X(s)|^{(1+\a)\bar{p}})h^{\gamma_g\bar{p}})dE(s)\bigg)\nonumber\\
    &\leq
    C\bigg(\EE_W\int^{T}_{0}|e(s)|^{\bar{p}}dE(s)
    +h^{\gamma_f\bar{p}}E(T)+h^{\gamma_g\bar{p}}E(T)\bigg).
\end{align}
其中最后一个不等式通过使用引理\ref{lemma3}得到. 接着, 我们根据不等式的基本性质处理$J_{22}$项, 可得
\begin{align}
    \label{lem332}
    J_{22}&= \EE_W\bigg(\sup_{0\leq t\leq T}\int^{t\wedge\theta_{\ell}}_{0}\bar{p}|e(s)|^{\bar{p}-2}\bigg(e^{T}(s)\big(f(\bar{\tau}(s),X(s))-f(\bar{\tau}(s),\bar{X}(s))\big)\bigg)dE(s)\nonumber\\
    &\quad+
    \sup_{0\leq t\leq T}\int^{t\wedge\theta_{\ell}}_{0}\bar{p}|e(s)|^{\bar{p}-2}\bigg(e^{T}(s)\big(f(\bar{\tau}(s),\bar{X}(s))-f_{h}(\bar{\tau}(s),\bar{X}(s))\big)\nonumber\\
    &\quad+\dfrac{5p-1}{5p-5\bar{p}}|g(\bar{\tau}(s),X(s))-g_h(\bar{\tau}(s),X(s))|^2\bigg)dE(s)\bigg)\nonumber\\
    &\leq I_1+ I_2.
\end{align}
对$I_1$, 根据\eqref{le210}和Young不等式可推导出
\begin{align}
    \label{333}
    I_{1}&=\EE_W\bigg(\sup_{0\leq t\leq T}\int^{t\wedge\theta_{\ell}}_{0}\bar{p}|e(s)|^{\bar{p}-2}\bigg(e^{T}(s)\big(f(\bar{\tau}(s),X(s))-f(\bar{\tau}(s),\bar{X}(s))\big)\bigg)dE(s)\bigg)\nonumber\\
    &\leq  \EE_W\bigg(\sup_{0\leq t\leq T}\int^{t\wedge\theta_{\ell}}_{0}\bar{p}|e(s)|^{\bar{p}-2}\bigg(e^{T}(s)\big(f^{'}(\bar{\tau}(s),x)|_{x=\bar{X}(s)}\nonumber\\
    &\quad \times \int^{s}_{0}g_h(\bar{\tau}(s_{1}),\bar{X}(s_{1}))dW(E(s_{1}))+\tilde{R}_{f}(s,X(s),\bar{X}(s))\big)\bigg)dE(s)\bigg)\nonumber\\
    &\leq 
    H_{21}\EE_W\bigg(\sup_{0\leq t\leq T}\int^{t\wedge\theta_{\ell}}_{0}\bigg(|e(s)|^{\bar{p}}+\big|e(s)^{T}(f^{'}(\bar{\tau}(s),x)|_{x=\bar{X}(s)}\nonumber\\
    &\quad \times \int^{s}_{0}g_h(\bar{\tau}(s_{1}),\bar{X}(s_{1}))dW(E(s_{1}))\big|^{\frac{\bar{p}}{2}}+|e(s)^{T}\tilde{R}_{f}(s,X(s),\bar{X}(s))|^{{\frac{\bar{p}}{2}}}\bigg)dE(s)\bigg)\nonumber\\
    &\leq  H_{21}\bigg(\EE_W\sup_{0\leq t\leq T}\int^{t\wedge\theta_{\ell}}_{0}\bigg(|e(s)|^{\bar{p}}dE(s)+\big|e(s)^{T}(f^{'}(\bar{\tau}(s),x)|_{x=\bar{X}(s)}\nonumber\\
    &\quad \times \int^{s}_{0}g_h(\bar{\tau}(s_{1}),\bar{X}(s_{1}))dW(E(s_{1}))\big|^{\frac{\bar{p}}{2}}dE(s)+|\tilde{R}_{f}(s,X(s),\bar{X}(s))|^{\bar{p}}dE(s)\bigg).   
\end{align}
再通过应用类似文献\cite{Wang2013466}中(3.35)的方法, 并结合\eqref{333}和引理 \ref{lemma5}可以得到
\begin{align}
    \label{335}
    I_{1}&\leq H_{21}\bigg(\EE_{W}\int^{T}_{0}|e(s)|^{\bar{p}}dE(s)+\EE_{W}\int^{T}_{0}|\tilde{R}_{f}(s,X(s),\bar{X}(s))|^{\bar{p}}dE(s)+h^{\bar{p}}\bigg)\nonumber\\
    &\leq 
    H_{21}\bigg(\EE_{W}\int^{T}_{0}|e(s)|^{\bar{p}}dE(s)+\int^{T}_{0}\EE_{W}|\tilde{R}_{f}(s,X(s),\bar{X}(s))|^{\bar{P}}dE(s)+h^{\bar{p}}\bigg)\nonumber\\
    &\leq 
    H_{21}\bigg(\EE_{W}\int^{T}_{0}|e(s)^{\bar{p}}dE(s))+h^{\bar{p}}(k(h))^{2\bar{p}}+h^{\bar{p}}\bigg).
\end{align} 
然后对$I_2$应用Young不等式、假设\ref{ass1}和 H{\"o}lder 不等式可得
 \begin{align*}
     I_{2}&=\EE_{W}\bigg(\sup_{0\leq t\leq T}\int^{t\wedge\theta_{\ell}}_{0}\bar{p}|e(s)|^{\bar{p}-2}\bigg(e^{T}(s)\big(f(\bar{\tau}(s),\bar{X}(s))-f_h(\bar{\tau}(s),\bar{X}(s))\big)\\
     &\quad+\dfrac{5p-1}{5p-5\bar{p}}|g(\bar{\tau}(s),X(s))-g_h(\bar{\tau}(s),X(s))|^2\bigg)dE(s)\bigg)\\
     &\leq 
     H_{22}\bigg(\EE_{W}\sup_{0\leq t\leq T}\int^{t\wedge\theta_{\ell}}_{0}|e(s)|^{\bar{p}}dE(s)+\EE_{W}\sup_{0\leq t\leq T}\int^{t\wedge\theta_{\ell}}_{0}|f(\bar{\tau}(s),\bar{X}(s))\\
     &\quad-f_h(\bar{\tau}(s),\bar{X}(s))|^{\bar{p}}+|g(\bar{\tau}(s),X(s))-g_h(\bar{\tau}(s),X(s))|^{\bar{p}}dE(s)\bigg)\\
     &\leq 
     H_{22}\bigg(\EE_{W}\sup_{0\leq t\leq T}\int^{t\wedge\theta_{\ell}}_{0}|e(s)|^{\bar{p}}dE(s)+\EE_{W}\sup_{0\leq t\leq T}\int^{t\wedge\theta_{\ell}}_{0}\big(1+|\bar{X}(s)|^{\a\bar{p}}\\
     &\quad +\big||\bar{X}(s)|\wedge \mu ^{-1}(k(h))\big|^{\a\bar{p}}\big) \bigg|\bar{X}(s)-\big(|\bar{X}(s)|\wedge \mu ^{-1}(k(h))\big)\frac{\bar{X}(s)}{|\bar{X}(s)|}\bigg|^{\bar{p}}dE(s)\\
     &\quad + \EE_{W}\sup_{0\leq t\leq T}\int^{t\wedge\theta_{\ell}}_{0}(1+|X(s)|^{\a\bar{p}}+\big||X(s)|\wedge \mu ^{-1}(k(h))\big|^{\a\bar{p}})\\
     &\quad \times \bigg|X(s)-\big(|X(s)|\wedge \mu ^{-1}(k(h))\big)\frac{X(s)}{|X(s)|}\bigg|^{\bar{p}}dE(s) \bigg)\\
     &\leq 
     H_{22}\bigg(\EE_{W}\int^{T}_{0}|e(s)|^{\bar{p}}dE(s)+\int^{T}_{0}\bigg(\EE_{W}\bigg[1+|\bar{X}(s)|^{q}+\big||\bar{X}(s)|\wedge \mu ^{-1}(k(h))\big|^{q}\bigg]\bigg)^{\frac{\a\bar{p}}{q}}\\
     &\quad \times \bigg[\EE_{W}\big|\bar{X}(s)-\big(|\bar{X}(s)|\wedge \mu ^{-1}(k(h))\big)\frac{\bar{X}(s)}{|\bar{X}(s)|}\big|^{\frac{q\bar{p}}{q-\a\bar{p}}}\bigg]^{\frac{q-\a\bar{p}}{q}}dE(s)\\
     &\quad + \int^{T}_{0}\bigg(\EE_{W}\bigg[1+|X(s)|^{q}+\big||X(s)|\wedge \mu ^{-1}(k(h))\big|^{q}\bigg]\bigg)^{\frac{\a\bar{p}}{q}}\\
     &\quad \times\bigg[\EE_{W} \big|X(s)-\big(|X(s)|\wedge \mu ^{-1}(k(h))\big)\frac{X(s)}{|X(s)|}\big|^{\frac{q\bar{p}}{q-\a\bar{p}}}\bigg]^{\frac{q-\a\bar{p}}{q}}dE(s) \bigg),
 \end{align*}
上述证明中引理\ref{lemma3}被使用到. 再通过H{\"o}lder不等式和Chebyshev不等式$ \PP(|x|\geqslant a) \leqslant a^{-q} \EE|x|^{q}$(其中$a>0,q>0$), 可得
\begin{align}
    \label{336}
    I_{2}&\leq 
    H_{22}\bigg(\EE_{W}\int^{T}_{0}|e(s)|^{\bar{p}}dE(s)\nonumber\\
    &\quad+\int^{T}_{0}\bigg(\EE_{W}\big|I\left\{|\bar{X}(s)|>\mu^{-1}(k(h))\right\}|\bar{X}(s)|^{\frac{q\bar{p}}{q-\a\bar{p}}}\big|\bigg)^{\frac{q-\a\bar{p}}{q}}dE(s)\nonumber\\
    &\quad+\int^{T}_{0}\bigg(\EE_{W}\big|I\left\{|X(s)|>\mu^{-1}(k(h))\right\}|X(s)|^{\frac{q\bar{p}}{q-\a\bar{p}}}\big|\bigg)^{\frac{q-\a\bar{p}}{q}}dE(s)\bigg)\nonumber\\
    &\leq 
    H_{22}\bigg(\EE_{W}\int^{T}_{0}|e(s)|^{\bar{p}}dE(s)\nonumber\\
    &\quad+\int^{T}_{0}\bigg(\big[P\left\{|\bar{X}(s)|>\mu^{-1}(k(h))\right\}\big]^{\frac{q-\a\bar{p}-\bar{p}}{q-\a\bar{p}}} \big[\EE|\bar{X}(s)|^{q}\big]^{\frac{\bar{p}}{q-\a\bar{p}}}\bigg)^{\frac{q-\a\bar{p}}{q}}dE(s)\nonumber\\
    &\quad+\int^{T}_{0}\bigg(\big[P\left\{|X(s)|>\mu^{-1}(k(h))\right\}\big]^{\frac{q-\a\bar{p}-\bar{p}}{q-\a\bar{p}}}[\EE|X(s)|^{q}]^{\frac{\bar{p}}{q-\a\bar{p}}}\bigg)^{\frac{q-\a\bar{p}}{q}}dE(s)\bigg)\nonumber\\
    &\leq 
    H_{22}\bigg(\EE_{W}\int^{T}_{0}|e(s)|^{\bar{p}}dE(s)+\int^{T}_{0}\bigg(\frac{\EE_{W}|\bar{X}(s)|^{q}}{|\mu^{-1}(k(h))|^{q}}\bigg)^{\frac{q-\a\bar{p}-\bar{p}}{q}}dE(s)\nonumber\\
    &\quad+\int^{T}_{0}\bigg(\frac{\EE_{W}|X(s)|^{q}}{|\mu^{-1}(k(h))|^{q}}\bigg)^{\frac{q-\a\bar{p}-\bar{p}}{q}}dE(s)\bigg)\nonumber\\
    &\leq 
    H_{22}\bigg(\EE_{W}\int^{T}_{0}|e(s)|^{\bar{p}}dE(s)+\big(\mu^{-1}(k(h))\big)^{(\a+1)\bar{p}-q}\bigg).
\end{align}
将\eqref{335} 和 \eqref{336} 代入\eqref{lem332} 可得
\begin{align}
    \label{lemm337}
    J_{22}\leq  H_{22}\bigg(\EE_{W}\int^{T}_{0}|e(s)|^{\bar{p}}dE(s)+\big(\mu^{-1}(k(h))\big)^{(\a+1)\bar{p}-q}+h^{\bar{p}}(k(h))^{2\bar{p}}+h^{\bar{p}}\bigg).
\end{align} 
对于$J_3$的处理, 我们通过应用Young不等式和引理\ref{lemma5}可推出
\begin{align}
    \label{J3}
    J_3&= \EE_{W}\sup_{0\leq t\leq T}\int^{t\wedge\theta_{\ell}}_{0}(5\bar{p}^{2}-\bar{p})|e(s)|^{\bar{p}-2}|\tilde{R}_{g_h}(s,X(s),\bar{X}(s))|^2dE(s)\nonumber\\
    &\leq 
    H_{3}\EE_{W}\sup_{0\leq t\leq T}\int^{t\wedge\theta_{\ell}}_{0}(|e(s)|^{\bar{p}}+|\tilde{R}_{g_h}(s,X(s),\bar{X}(s))|^{\bar{p}})dE(s)\nonumber\\
    &\leq 
    H_{3}\bigg(\EE_{W}\int^{T}_{0}|e(s)|^{\bar{p}}dE(s)+\int^{T}_{0}\EE_{W}|\tilde{R}_{g_h}(s,X(s),\bar{X}(s))|^{\bar{p}}dE(s)\bigg)\nonumber\\
    &\leq 
    H_{3}\bigg(\EE_{W}\int^{T}_{0}|e(s)|^{\bar{p}}dE(s)+h^{\bar{p}}(k(h))^{2\bar{p}}\bigg).
\end{align}
其中$H_{21}$、$H_{22}$、$H_{3}$ 和下面的C都是与$h$无关的正常数, 其值可能会在每行之间发生变化. 最后联立方程\eqref{sup}、\eqref{J1}、\eqref{lemm330}、\eqref{lemm331}、 \eqref{lemm337}、\eqref{J3}可将原始式子写为
\begin{align*}
    \EE_W\left(\sup_{0\leq t\leq T}|e( t\wedge\theta_{\ell})|^{\bar{p}}\right)\leq& \frac{1}{2}\sup_{0\leq r\leq t\wedge\theta_{\ell}}|e(r)|^{\bar{p}}+[J_1]+[J_2]+[J_3]\\
    \leq&
    2([J_1]+[J_2]+[J_3])\\
    \leq  &
    C\bigg(\EE_{W}\int^{T}_{0}|e(s)|^{\bar{p}}dE(s)+h^{\gamma_f\bar{p}}+h^{\gamma_g\bar{p}}\\
    &\quad+h^{\bar{p}}(k(h))^{2\bar{p}}+h^{\bar{p}}+\big(\mu^{-1}(k(h))\big)^{(\a+1)\bar{p}-q}\bigg)\\
    \leq &
    C\bigg(\int^{T}_{0}\EE_{W}\sup_{0\leq u\leq s}|e(u\wedge\theta_{\ell} )|^{\bar{p}}dE(s)+h^{\gamma_f\bar{p}}+h^{\gamma_g\bar{p}}\\
    &\quad+h^{\bar{p}}(k(h))^{2\bar{p}}+\big(\mu^{-1}(k(h))\big)^{(\a+1)\bar{p}-q}\bigg).
\end{align*}
再应用Gronwall不等式可得
\begin{align*}
    \EE_W(\sup_{0\leq t\leq T}|e( t\wedge\theta_{\ell})|^{\bar{p}})
    &\leq 
    C\bigg(h^{\gamma_f\bar{p}}+h^{\gamma_g\bar{p}}
    +h^{\bar{p}}(k(h))^{2\bar{p}}+(\mu^{-1}(k(h)))^{(\a+1)\bar{p}-q}\bigg)e^{\l E(T)},
\end{align*}
通过令$n\to \infty$, 根据Fatou引理可得
\begin{align*}
    \EE_W(\sup_{0\leq t\leq T}|e(t)|^{\bar{p}})
    &\leq 
    C\bigg(h^{\gamma_f\bar{p}}+h^{\gamma_g\bar{p}}
    +h^{\bar{p}}(k(h))^{2\bar{p}}+\big(\mu^{-1}(k(h))\big)^{(\a+1)\bar{p}-q}\bigg)e^{\l E(T)},
\end{align*}
其中$C$与$h$独立, 且$\l > 0$. 接着我们对两边同时取$\EE_D$得
 \begin{align}
     \label{EEW}
    \EE\left( \sup_{0\leq t\leq T}|Y(t)-X(t)|^{\bar{p}}\right)\leq C\bigg(h^{\gamma_f\bar{p}}+h^{\gamma_g\bar{p}}
    +h^{\bar{p}}(k(h))^{2\bar{p}}+\big(\mu^{-1}(k(h))\big)^{(\a+1)\bar{p}-q}\bigg).
\end{align}
再联立引理\ref{lemma2}和\eqref{EEW}可得
\begin{align*}
    \EE\left( \sup_{0\leq t\leq T}|Y(t)-\bar{X}(t)|^{\bar{p}}\right)\leq C\bigg(h^{\gamma_f\bar{p}}+h^{\gamma_g\bar{p}}
    +h^{\bar{p}}(k(h))^{2\bar{p}}+\big(\mu^{-1}(k(h))\big)^{(\a+1)\bar{p}-q}\bigg).
\end{align*}
最后通过选择适当$\mu^{-1}(\cdot)$和$\kappa(\cdot)$, 完成证明.
\end{proof}



%----- 第4章 插图环境 -----

% 插图环境

\chapter{数值模拟}

\section{模拟$D(s)$和$E(t)$}
设定等距步长 $\delta \in (0,1)$ 及时间区间 $T > 0$.为了在区间 $[0,T]$ 上逼近逆从属过程 $E$,我们遵循 \cite{magdziarz2009stochastic} 中提出的方法.具体来说,首先通过模拟具有独立且平稳增量的从属过程 $D$ 的样本路径来进行逼近.设定 $D_0 = 0$,然后遵循规则 $D_{i\delta} := D_{(i-1)\delta} + Z_i, i=1,2,3,\ldots$,其中 $\{Z_i\}_{i \in \mathbb{N}}$ 为独立同分布的序列,且 $Z_i \stackrel{d}{=} D_{\delta}$.我们在找到整数 $N$ 使得 $T \in [D_N\delta, D_{(N+1)\delta})$ 时停止该过程.请注意,$\mathbb{N}\cup\{0\}$ 值的随机变量 $N$ 确实存在,因为 $D_t \to \infty$ 随着 $t \to \infty$ 几乎必然成立.可以通过下面的算法生成随机变量 $\{Z_i\}$,
\begin{align*}
	Z(i)=\delta^{1/\alpha}\xi_{i}
\end{align*}
其中$\xi_i$是独立同分布的完全偏斜的$\alpha$稳定随机变量,$\xi_i$的实现过程如下:
\begin{align*}
	\xi_i=\frac{\sin(\alpha(V+c_1))}{\left(\cos(V)\right)^{1/\alpha}}\Big(\frac{\cos(V-\alpha(V+c_1))}{W}\Big)^{(1-\alpha)/\alpha}
\end{align*}
其中$c_1 = \frac{\pi}{2}$,随机变量$V$是$(-\frac{\pi}{2},\frac{\pi}{2})$上的均匀分布,$W$是均值为$1$的指数分布.
然后,令
$$
E_t^\delta := \left(\min\{n \in \mathbb{N}; D_{n\delta} > t\} - 1\right)\delta, \quad t \in [0, T].
$$
过程 $E^\delta = (E_t^\delta)_{t \geq 0}$ 的样本路径是具有恒定跳跃大小 $\delta$ 的单调递增阶梯函数,第 $i$ 个等待时间为 $Z_i = D_{i\delta} - D_{(i-1)\delta}$.过程 $E^\delta$ 有效地逼近 $E$;实际上,几乎必然地,
$$
E_t - \delta \leq E_t^\delta \leq E_t \quad \text{对于所有} \quad t \in [0, T].
$$

在\cite{jin2019strong}中对$\Delta E$的处理时,t每次跳$D_{n\delta} - D_{(n-1)\delta}$,于是$E$每次对应改变$\delta$.然而,在我们的离散格式中,选择对t做等距离散,让t每次跳跃的长度是固定的长度h,于是$E$在第i次跳跃对应的变化就是$E_{ih} - E_{(i-1)h}$,这样的离散会导致出现$\Delta E=0$,这是得到收敛阶是$\alpha$的关键.

\section{时间变换的布朗运动驱动的CIR过程}\label{TCCIR}
回忆时间变换的CIR过程
\begin{equation}\label{CIR}
	dy(t)=\kappa(\theta-y(t))dE(t)+\sigma\sqrt{y(t)}dB(E(t)),\quad t\geq0,\quad y(0)>0.
\end{equation}
如果 $2\kappa\theta\geq\sigma^{2}$, 那么 $D=(0,\infty)$ 并且\cref{assum1} 在 $(\alpha,\beta)=$
$(0,\infty)$是成立的. 
另外, 使用It\^{o}公式$X(t)=F((y(t))$,其中$F$是由\cref{Lamperti}定义,即对时间变换的CIR过程进行Lamperti变换可以得到
\begin{equation}
	dX(t)=f(X(t))dE(t)+\frac12\sigma dB(E(t)),\quad t\geq0,\quad X(0)=\sqrt{y(0)}
\end{equation}
其中
\begin{equation}
	f(X)=\dfrac{1}{2}\kappa\left(\theta_vX^{-1}-X\right),\quad X>0
\end{equation}
其中 $\theta_v=\theta-\frac{\sigma^2}{4\kappa}$ ,并且 BEM数值格式如下
\begin{equation}
	X_{t_{i+1}}=X_{t_{i}}+f(X_{t_{i+1}})\Delta E_i+\frac{1}{2}\sigma\Delta B_{E_i},\quad k=0,1,\dots 
\end{equation}
观察到
\begin{equation}
	f'(X)=-\frac{1}{2}\kappa(\theta_vX^{-2}+1)
\end{equation}
以及
\begin{equation}
	f(X)f'(X)+\frac{\sigma^2}{2}f''(X)=-\frac{\kappa^2}{4}(\theta_v^2X^{-3}-X)+\frac{1}{2}\kappa\theta_vX^{-3}\sigma^2.
\end{equation}
因此为了满足\cref{assum3},只需要满足
\begin{equation}
	\sup_{0\leq t\leq T}\mathbb{E}[X(t)^{-3}]=\sup_{0\leq t\leq T}\mathbb{E}[y(t)^{-\frac{3}{2}}] < \infty.
\end{equation}
对于时间变换的CIR过程y(t)的矩有界,即
\begin{equation}
	\sup\limits_{0\leq t\leq T}\mathbb{E}[y(t)^q]<\infty\quad\mathrm{for}\quad q>-\frac{2k\theta}{\sigma^2},
\end{equation}
下面验证,对于由时间变换的的布朗运动驱动的CIR过程\cref{CIR}的精确解矩有界.
\begin{proposition}
	对于由时间变换的布朗运动驱动的CIR过程\cref{CIR},其中$y_0>0$,$1<p<\frac{2K\theta}{\sigma^2}-1$,都存在一个常数C使得
	\begin{equation*}
		\sup\limits_{t\in[0,T]}\mathbb{E}\left[\left(y(t)\right)^{-p}\right]\leq C(1+y(0)^{-p})
	\end{equation*}
\end{proposition}
\begin{proof}
	定义停时$\tau_{n}=\mathrm{inf}\{0<s\leq T;y(s)\leq1/n\}$,通过It\^{o}公式,我们可以得到
	$$\begin{aligned}
		\mathbb{E}_B\left[(y(t\wedge\tau_{n}))^{-p}\right] &=y(0)^{-p}-p\mathbb{E}_B\left[\int_{0}^{t\wedge\tau_{n}}\frac{K(\theta-y(s))}{(y(s))^{p+1}}dE(s)\right]\\
		&+p(p+1)\frac{\sigma^{2}}{2}\mathbb{E}_B\left[\int_{0}^{t\wedge\tau_{n}}\frac{1}{(y(s))^{p+1}}dE(s)\right] \\
		&\leq y(0)^{-p}+pK\int_{0}^{t}\mathbb{E}_B\left(\frac{1}{(y(s\wedge\tau_{n}))^{p}}
		\right)dE(s) \\
		&+\mathbb{E}_B\left[\int_0^{t\wedge\tau_n}\frac{p\left(\frac{(p+1)\sigma^2}{2}-K\theta\right)}{(y(s))^{p+1}}dE(s)\right]
	\end{aligned}$$
	通过计算可以找到正数$\underline C$使得, 当$\frac{(p+1)\sigma^2}{2}-K\theta<0$时,对于任意的 $y(0)=x>0$,都有
	$$\frac{p\left(\frac{(p+1)\sigma^2}{2}-K\theta\right)}{x^{p+1}}\leq \underline C$$
	因此
	$$\mathbb{E}_B\left[(y(t\wedge\tau_n))^{-p}\right]\leq y(0)^{-p}+\underline{C}E(T)+pK\int_0^t\sup_{r\in[0,s]}\mathbb{E}_B\left[(y(r\wedge\tau_n))^{-p}\right]dE(s)$$
	于是由Gronwall不等式,可以得到
	$$\sup\limits_{t\in[0,T]}\mathbb{E}_B\left[(y(t\wedge\tau_n)^x)^{-p}\right]\leq\left(y(0)^{-p}+\underline{C}E(T)\right)\exp(pKE(T))$$
	两边同时取$\mathbb{E}_D$并使用Cauchy-Schwarz不等式,得到
	$$\begin{aligned}
		\sup\limits_{t\in[0,T]}\mathbb{E}\left[(y(t\wedge\tau_n)^x)^{-p}\right]&\leq\mathbb{E}\left[\left(y(0)^{-p}+\underline{C}E(T)\right)\exp(pKE(T))\right]\\
		&\leq\sqrt{\mathbb{E}\left[\left(y(0)^{-p}+\underline{C}E(T)\right)^2\right]\mathbb{E}\left[\exp(2pKE(T))\right]}
	\end{aligned}$$
	从\cite{jum2014strong}可以得到
	\begin{equation}
		\mathbb{E}[E^n(t)]=\frac{n!}{\Gamma(n\alpha+1)}t^{n\alpha}
	\end{equation}
	\begin{equation}
		\mathbb{E}[e^{\lambda E(t)}]<\infty
	\end{equation}
	其中$\lambda \in \mathbb{R},t>0$.最后,让 $n\to+\infty$,我们完成了这个证明.
\end{proof}
\begin{remark}
	在这里只是证明了当$1<p<\frac{2K\theta}{\sigma^2}-1$的时候,矩的存在性,实际上更可以证明$p<\frac{2K\theta}{\sigma^2}$,但是证明起来过于复杂,这里就不在说明,对于我们的结果已经够用了.
\end{remark}
对于\cref{assum3},可以验证只需要保证$1 < \frac{4}{3}\frac{k\theta}{\sigma^2}$成立即可,而这个区间可以被$1<p<\frac{2K\theta}{\sigma^2}-1$包含在内,因此只需要保证p在后者这个区间即可.至于\cref{assum2},在$(0,\infty)$中很容易可以验证存在这样的$\kappa$使之成立.因此由\cref{main th}可以得到,对于时间变换的CIR过程,使用BEM数值格式的强收敛阶是$\alpha$
\section{数值实验}
在我们的数值实验中,我们关注端点$T = 1$处的$L_1$误差,因此我们令
\begin{align*}
	e_T^{i}=\mathbb{E}\left|X_T^{\delta _{15}}-X_T^{\delta _i}\right|
\end{align*}
其中$X_T^{\delta _i}$是步长为$\delta _i$时T处的模拟值,$\delta _i = 2^{-i}$,对于我们的数值实验,取$\theta=0.125,\kappa=2$以及$\sigma=0.5$,采用蒙德卡洛方法,
\begin{align*}
	e_{T}^i\approx\frac{1}{10^3}\sum_{j=1}^{10^3}\left|X_T^{\delta _{15}}-X_T^{\delta _i}\right|.
\end{align*}
选择步长为$2^{-15}$作为参考,通过${2^{-11},2^{-10},2^{-9},2^{-8}}$的步长来估计$L_1$误差.
\section{时间变换的布朗运动驱动的CEV过程}
另外一个比较常用的金融模型是CEV模型,下面是由时变布朗运动驱动的CEV模型
\begin{equation}\label{CEV}
	dy(t)=\kappa(\theta-y(t))dE(t)+\sigma y(t)^\alpha dB_{E(t)}
\end{equation}
其中 $0.5<\alpha<1,\kappa,\theta,\sigma>0.$ 通过变换$X(t)=F(y(t))$的变换之后,其中$F$是由\cref{Lamperti}定义,我们可以得到\cref{assum2}在$(\alpha,\beta)=(0,\infty)$下,是成立的,此时	$$dX(t)=f(X(t))dE(t)+(1-\alpha)\sigma dB(E(t))$$
其中
$$f(X)=(1-\alpha)\left(\kappa\theta X^{-\frac\alpha{1-\alpha}}-\kappa X-\frac{\alpha\sigma^2}2X^{-1}\right),\quad X>0.$$
同时我们需要验证另一个\cref{assum3}.因为 $\alpha>0.5$, 于是$\frac{1}{1-\alpha}>2$,因此
$$f^{\prime}(X)=-\alpha\kappa\theta X^{-\frac1{1-\alpha}}-(1-\alpha)\kappa+(1-\alpha)\frac{\alpha\sigma^2}2X^{-2},\quad X>0$$
同时我们又有
$$f^{\prime\prime}(X)=\frac\alpha{1-\alpha}\kappa\theta X^{-\frac{2-\alpha}{1-\alpha}}-(1-\alpha)\alpha\sigma^2X^{-3}.$$
下面验证,对于由时间变换的的布朗运动驱动的CEV过程\cref{CEV}的精确解矩有界.
\begin{proposition}
	对于由时间变换的布朗运动驱动的CEV过程\cref{CEV},其中$X_0>0$,对于任意的$\frac{1}{2}<\alpha<1$和任意的$p>0$,都存在一个常数C使得
	\begin{equation*}
		\sup\limits_{t\in[0,T]}\mathbb{E}\left[\left(y(t)\right)^{-p}\right]\leq C(1+y(0)^{-p})
	\end{equation*}
\end{proposition}
\begin{proof}
	定义停时$\tau_{n}=\mathrm{inf}\{0<s\leq T;y(s)\leq1/n\}$,通过It\^{o}公式,我们可以得到
	$$\begin{aligned}
		\mathbb{E}_B\left[(y(t\wedge\tau_{n}))^{-p}\right] &=y(0)^{-p}-p\mathbb{E}_B\left[\int_{0}^{t\wedge\tau_{n}}\frac{K(\theta-y(s))}{(y(s))^{p+1}}dE(s)\right]\\
		&+p(p+1)\frac{\sigma^{2}}{2}\mathbb{E}_B\left[\int_{0}^{t\wedge\tau_{n}}\frac{1}{(y(s))^{p+2(1-\alpha)}}dE(s)\right] \\
		&\leq y(0)^{-p}+pK\int_{0}^{t}\mathbb{E}_B\left(\frac{1}{(y(s\wedge\tau_{n}))^{p}}
		\right)dE(s) \\
		&+\mathbb{E}_B\left[\int_0^{t\wedge\tau_n}\left(p(p+1)\frac{\sigma^2}{2}\frac{1}{(y(s))^{p+2(1-\alpha)}}-p\frac{K\theta}{(y(s))^{p+1}}\right)dE(s)\right]
	\end{aligned}$$
	可以找到正数$C$使得, 对于任意的 $y(0)=x>0$,都有
	$$\left(p(p+1)\frac{\sigma^2}{2}\frac{1}{x^{p+2(1-\alpha)}}-p\frac{K\theta}{x^{p+1}}\right)\leq C$$
	通过计算可以得到,$\underline C=p(2\alpha-1)\frac{\sigma^2}{2}\left[(p+2(1-\alpha))\frac{\sigma^2}{2K\theta}\right]^{\frac{p+2(1-\alpha)}{2\alpha-1}}$ 是最小的上界. 因此
	$$\mathbb{E}_B\left[(y(t\wedge\tau_n))^{-p}\right]\leq y(0)^{-p}+\underline{C}E(T)+pK\int_0^t\sup_{r\in[0,s]}\mathbb{E}_B\left[(y(r\wedge\tau_n))^{-p}\right]dE(s)$$
	于是由Gronwall不等式,可以得到
	$$\sup\limits_{t\in[0,T]}\mathbb{E}_B\left[(y(t\wedge\tau_n)^x)^{-p}\right]\leq\left(y(0)^{-p}+\underline{C}E(T)\right)\exp(pKE(T))$$
	两边同时取$\mathbb{E}_D$并使用Cauchy-Schwarz不等式,得到
	$$\begin{aligned}
		\sup\limits_{t\in[0,T]}\mathbb{E}\left[(y(t\wedge\tau_n)^x)^{-p}\right]&\leq\mathbb{E}\left[\left(y(0)^{-p}+\underline{C}E(T)\right)\exp(pKE(T))\right]\\
		&\leq\sqrt{\mathbb{E}\left[\left(y(0)^{-p}+\underline{C}E(T)\right)^2\right]\mathbb{E}\left[\exp(2pKE(T))\right]}
	\end{aligned}$$
	从\cite{jum2014strong}可以得到
	\begin{equation}
		\mathbb{E}[E^n(t)]=\frac{n!}{\Gamma(n\alpha+1)}t^{n\alpha}
	\end{equation}
	\begin{equation}
		\mathbb{E}[e^{\lambda E(t)}]<\infty
	\end{equation}
	其中$\lambda \in \mathbb{R},t>0$.最后,让 $n\to+\infty$,我们完成了这个证明.
\end{proof}


由Lamperti变换可知,$X(t)$的逆阶矩可以$y(t)$的逆阶矩控制,于是\cref{assum3}成立.因此根据\cref{main th}可以得到由时间变换布朗运动驱动的CEV过程的收敛阶是$\alpha$
\

\begin{figure}[htp!]
	\centering
	\includegraphics[width=0.45\linewidth]{BEMalpha=0.3.eps}
	\hfill
	\includegraphics[width=0.45\linewidth]{BEMalpha=0.8.eps}
	\caption{时间变换CIR过程的数值解与解析解之间的绝对误差估计.左图是$\alpha=0.3$,右图是$\alpha=0.8$}
	\label{fig:image}
	\vspace{-2ex}
\end{figure}

%	\section{Present Job}
%	我们想证明的是对于一个漂移项和扩散项都是非线性的时间变换SDE,将它进行Lamperti变换,变成漂移项是非线性,扩散项是常数的时间变换SDE,然后使用数值方法进行离散,得到与模拟代码一致的强收敛阶$\alpha$.截止当前的证明,我们已经证明了显式EM在$L^1$误差下的强收敛阶是$\alpha$,在证明$L^p$时,若使用之前的方法,得到的强收敛阶不仅与模拟的结果不一致,并且与p有关,强收敛阶与模拟代码不一致.证明的显示EM的强收敛阶,看起来对我们的预期没有什么意义,至少暂时我还没有找到对于原始非线性的漂移项扩散项经过Lamperti变换之后,得到的时变SDE漂移项满足全局Lipschitz,进而能够使用显示EM得到强收敛阶.下面对于变换后非线性的SDE,采用截断EM或者BEM,都会出现一个难以处理的东西,我们必须计算$L^2$误差,也就是$e^2$,这将导致不得不使用Gronwall不等式来处理这个式子,此时也会出现完全与p相关的收敛阶,那么我们可能需要换一个方法了.
%	由于无法处理$L^2$误差,因此对于单调Lipschitz条件出现的平方项就无法使用.
%	\textcolor{red}{至于Gronwall不等式,他在积分方面是几乎无法使用的,如果你选择先期望再Gronwall,会发现期望无法进入积分里面,这就导致了Gronwall条件不成立.然而如果先Gronwall再期望,则会发现离Gronwall条件相差十万八千里}

%----- 第5章 表格环境 -----

% 表格环境

\chapter{结论与展望}




%%%%%%%%%%%%%%%%%%%% 参考文献 %%%%%%%%%%%%%%%%%%

% 生成参考文献, 两种方式任选一种

% 第一种方式, 使用 bib 文件
%\nocite{*}  % 可以显示全部参考文献, 包括未引用的
\bibliography{mybib}


%---------------------------------------------%

% 第二种方式, 手动添加文献信息
%
%%%%%%%%%%%%%% 手动添加参考文献  %%%%%%%%%%%%%%%

\begin{thebibliography}{99}

    \providecommand{\natexlab}[1]{#1}
    \providecommand{\url}[1]{#1}
    \expandafter\ifx\csname urlstyle\endcsname\relax\else
    \urlstyle{same}\fi
    \expandafter\ifx\csname href\endcsname\relax
    \DeclareUrlCommand\doi{\urlstyle{rm}}
    \def\eprint#1#2{#2}
    \else
    \def\doi#1{\href{https://doi.org/#1}{\nolinkurl{#1}}}
    \let\eprint\href
    \fi
    
    \bibitem[Umarov et~al.(2018)Umarov, Hahn, and Kobayashi]{Umarov20181}
    Umarov~S, Hahn~M, Kobayashi~K.
    \newblock Beyond the triangle: Brownian motion, {Ito} calculus, and
    {Fokker-Planck} equation - fractional generalizations\allowbreak[M].
    \newblock Hackensack: World Scientific Publishing Co.Pte.Ltd, 2018.
    
    \bibitem[Meerschaert et~al.(2004)Meerschaert and Scheffler]{Meerschaert}
    Meerschaert~M-M, Scheffler~H-P.
    \newblock Limit theorems for continuous-time random walks with infinite mean
    waiting times\allowbreak[J].
    \newblock Journal of Applied Probability, 2004, 41\allowbreak (3):\allowbreak
    623-638.
    
    \bibitem[Deng et~al.(2017)Deng and Schilling]{Deng2017}
    Deng~C~S, Schilling~R~L.
    \newblock Harnack inequalities for {SDEs} driven by time-changed fractional
    brownian motions\allowbreak[J].
    \newblock Electronic Journal of Probability, 2017, 22:\allowbreak 1-23.
    
    \bibitem[Kobayashi(2011)]{Kobayashi2011}
    Kobayashi~K.
    \newblock Stochastic calculus for a time-changed semimartingale and the
    associated stochastic differential equations\allowbreak[J].
    \newblock Journal of Theoretical Probability, 2011, 24\allowbreak
    (3):\allowbreak 789-820.
    
    \bibitem[Wu(2016)]{wu2016stability}
    Wu~Q.
    \newblock Stability analysis for a class of nonlinear time-changed
    systems\allowbreak[J].
    \newblock Cogent Mathematics, 2016, 3\allowbreak (1):\allowbreak 1228273.
    
    \bibitem[Nane et~al.(2017)Nane and Ni]{Nane20173085}
    Nane~E, Ni~Y.
    \newblock Stability of the solution of stochastic differential equation driven
    by time-changed {Lévy} noise\allowbreak[J].
    \newblock Proceedings of the American Mathematical Society, 2017,
    145\allowbreak (7):\allowbreak 3085-3104.
    
    \bibitem[Nane et~al.(2018)Nane and Ni]{Nane2018479}
    Nane~E, Ni~Y.
    \newblock Path stability of stochastic differential equations driven by
    time-changed {Lévy} noises\allowbreak[J].
    \newblock Alea-latin American Journal of Probability and Mathematical
    Statistics, 2018, 15\allowbreak (1):\allowbreak 479-507.
    
    \bibitem[Zhang et~al.(2019)Zhang and Yuan]{Zhang2019689}
    Zhang~X, Yuan~C.
    \newblock Razumikhin-type theorem on time-changed stochastic functional
    differential equations with {Markovian} switching\allowbreak[J].
    \newblock Open Mathematics, 2019, 17\allowbreak (1):\allowbreak 689-699.
    
    \bibitem[Yin et~al.(2021)Yin, Xu, and Shen]{Yin20212338}
    Yin~X, Xu~W, Shen~G.
    \newblock Stability of stochastic differential equations driven by the
    time-changed {Lévy} process with impulsive effects\allowbreak[J].
    \newblock International Journal of Systems Science, 2021, 52\allowbreak
    (11):\allowbreak 2338-2357.
    
    \bibitem[Shen et~al.(2023)Shen, Zhang, Song, and Wu]{Shen2023}
    Shen~G, Zhang~T, Song~J, Wu~J.
    \newblock On a class of distribution dependent stochastic differential
    equations driven by time-changed brownian motions\allowbreak[J].
    \newblock Applied Mathematics and Optimization, 2023, 88\allowbreak
    (2):\allowbreak 1432-0606.
    
    \bibitem[Li et~al.(2023{\natexlab{a}})Li, Xu, and Yan]{Li2023}
    Li~Z, Xu~L, Yan~L.
    \newblock Mckean-vlasov stochastic differential equations driven by the
    time-changed brownian motion\allowbreak[J].
    \newblock Journal of Mathematical Analysis and Applications,
    2023{\natexlab{a}}, 527\allowbreak (1):\allowbreak 127336.
    
    \bibitem[Magdziarz(2009{\natexlab{a}})]{Magdziarz2009553}
    Magdziarz~M.
    \newblock Black-scholes formula in subdiffusive regime\allowbreak[J].
    \newblock Journal of Statistical Physics, 2009{\natexlab{a}}, 136\allowbreak
    (3):\allowbreak 553-564.
    
    \bibitem[Magdziarz et~al.(2011)Magdziarz, Orzeł, and Weron]{Magdziarz2011187}
    Magdziarz~M, Orzeł~S, Weron~A.
    \newblock Option pricing in subdiffusive {Bachelier} model\allowbreak[J].
    \newblock Journal of Statistical Physics, 2011, 145\allowbreak (1):\allowbreak
    187-203.
    
    \bibitem[Janczura et~al.(2011)Janczura, Orzeł, and
    Wyłomańska]{janczura2011subordinated}
    Janczura~J, Orzeł~S, Wyłomańska~A.
    \newblock Subordinated $\alpha$-stable ornstein--uhlenbeck process as a tool
    for financial data description\allowbreak[J].
    \newblock Physica A: Statistical Mechanics and its Applications, 2011,
    390\allowbreak (23-24):\allowbreak 4379-4387.
    
    \bibitem[Chen(2017)]{Chen2017168}
    Chen~Z~Q.
    \newblock Time fractional equations and probabilistic
    representation\allowbreak[J].
    \newblock Chaos, Solitons and Fractals, 2017, 102:\allowbreak 168-174.
    
    \bibitem[Hahn et~al.(2012)Hahn, Kobayashi, and Umarov]{Hahn2012262}
    Hahn~M, Kobayashi~K, Umarov~S.
    \newblock {SDEs} driven by a time-changed {Lévy} process and their associated
    time-fractional order pseudo-differential equations\allowbreak[J].
    \newblock Journal of Theoretical Probability, 2012, 25\allowbreak
    (1):\allowbreak 262-279.
    
    \bibitem[Magdziarz(2009{\natexlab{b}})]{Magdziarz20093238}
    Magdziarz~M.
    \newblock Stochastic representation of subdiffusion processes with
    time-dependent drift\allowbreak[J].
    \newblock Stochastic Processes and their lications, 2009{\natexlab{b}},
    119\allowbreak (10):\allowbreak 3238-3252.
    
    \bibitem[Nane et~al.(2016)Nane and Ni]{Nane2016103}
    Nane~E, Ni~Y.
    \newblock Stochastic solution of fractional {Fokker-Planck} equations with
    space-time-dependent coefficients\allowbreak[J].
    \newblock Journal of Mathematical Analysis and Applications, 2016,
    442\allowbreak (1):\allowbreak 103-116.
    
    \bibitem[Diethelm et~al.(2002)Diethelm, Ford, and Freed]{Diethelm20023}
    Diethelm~K, Ford~N~J, Freed~A~D.
    \newblock A predictor-corrector approach for the numerical solution of
    fractional differential equations\allowbreak[J].
    \newblock Nonlinear Dynamics, 2002, 29\allowbreak (1-4):\allowbreak 3-22.
    
    \bibitem[Du et~al.(2012)Du, Gunzburger, Lehoucq, and Zhou]{Du2012667}
    Du~Q, Gunzburger~M, Lehoucq~R, et~al.
    \newblock Analysis and approximation of nonlocal diffusion problems with volume
    constraints\allowbreak[J].
    \newblock SIAM Review, 2012, 54\allowbreak (4):\allowbreak 667-696.
    
    \bibitem[Li et~al.(2015)Li and Zeng]{Li20151}
    Li~C, Zeng~F.
    \newblock {Numerical Methods for Fractional Calculus}\allowbreak[M].
    \newblock Chapman and Hall: Numerical Methods for Fractional Calculus, 2015.
    
    \bibitem[Wang et~al.(2023)Wang and Zou]{Wang20232125}
    Wang~D, Zou~J.
    \newblock {Mittag–Leffler} stability of numerical solutions to time
    fractional {ODEs}\allowbreak[J].
    \newblock Numerical Algorithms, 2023, 92\allowbreak (4):\allowbreak 212-2159.
    
    \bibitem[Jum et~al.(2016)Jum and Kobayashi]{Jum2016201}
    Jum~E, Kobayashi~K.
    \newblock A strong and weak approximation scheme for stochastic differential
    equations driven by a time-changed {Brownian} motion\allowbreak[J].
    \newblock Probability and Mathematical Statistics, 2016, 36\allowbreak
    (2):\allowbreak 201-220.
    
    \bibitem[Jin et~al.(2019)Jin and Kobayashi]{Jin2019619}
    Jin~S, Kobayashi~K.
    \newblock Strong approximation of stochastic differential equations driven by a
    time-changed {Brownian} motion with time-space-dependent
    coefficients\allowbreak[J].
    \newblock Journal of Mathematical Analysis and Applications, 2019,
    476\allowbreak (2):\allowbreak 619-636.
    
    \bibitem[Jin et~al.(2021)Jin and Kobayashi]{Jin2021829}
    Jin~S, Kobayashi~K.
    \newblock Strong approximation of time-changed stochastic differential
    equations involving drifts with random and non-random
    integrators\allowbreak[J].
    \newblock BIT Numerical Mathematics, 2021, 61\allowbreak (3):\allowbreak
    829-857.
    
    \bibitem[Wen et~al.(2023)Wen, Li, and Xu]{Wen2023}
    Wen~X, Li~Z, Xu~L.
    \newblock Strong approximation of non-autonomous time-changed {McKean–Vlasov}
    stochastic differential equations\allowbreak[J].
    \newblock Communications in Nonlinear Science and Numerical Simulation, 2023,
    119:\allowbreak 107122.
    
    \bibitem[Hutzenthaler et~al.(2011)Hutzenthaler, Jentzen, and
    Kloeden]{Hutzenthaler20111563}
    Hutzenthaler~M, Jentzen~A, Kloeden~P~E.
    \newblock Strong and weak divergence in finite time of {Euler's} method for
    stochastic differential equations with non-globally {Lipschitz} continuous
    coefficients\allowbreak[J].
    \newblock Proceedings of the Royal Society A: Mathematical, Physical and
    Engineering Sciences, 2011, 467\allowbreak (2130):\allowbreak 1563-1576.
    
    \bibitem[Deng et~al.(2020)Deng and Liu]{Deng20201133}
    Deng~C~S, Liu~W.
    \newblock Semi-implicit {Euler–Maruyama} method for non-linear time-changed
    stochastic differential equations\allowbreak[J].
    \newblock BIT Numerical Mathematics, 2020, 60\allowbreak (4):\allowbreak
    1133-1151.
    
    \bibitem[Liu et~al.(2020)Liu, Mao, Tang, and Wu]{Liu202066}
    Liu~W, Mao~X, Tang~J, Wu~Yue.
    \newblock Truncated {Euler-Maruyama} method for classical and time-changed
    non-autonomous stochastic differential equations\allowbreak[J].
    \newblock Applied Numerical Mathematics, 2020, 153:\allowbreak 66-81.
    
    \bibitem[Li et~al.(2023{\natexlab{b}})Li, Liao, Liu, and Xing]{Li2023651}
    Li~X, Liao~J, Liu~W, et~al.
    \newblock Convergence and stability of an explicit method for autonomous
    time-changed stochastic differential equations with super-linear
    coefficients\allowbreak[J].
    \newblock Advances in Applied Mathematics and Mechanics, 2023{\natexlab{b}},
    15\allowbreak (3):\allowbreak 651-683.
    
    \bibitem[刘暐\ 等(2020)刘暐 and 毛学荣]{刘暐2020}
    刘暐, 毛学荣.
    \newblock 随机方程的截断方法综述\allowbreak[J].
    \newblock 安徽工程大学学报, 2020, 35\allowbreak (1):\allowbreak 1-11.
    
    \bibitem[汤婧雯(2021)]{汤婧雯2021}
    汤婧雯.
    \newblock
    非自治随机微分方程的截断欧拉方法及其应用\allowbreak[D].
    \newblock 上海师范大学, 2021.
    
    \bibitem[Giles(2008)]{giles2006}
    Giles~M.
    \newblock {Improved Multilevel Monte Carlo Convergence Using the Milstein
        Scheme}\allowbreak[M]//\allowbreak
    Berlin, Heidelberg: Springer Berlin Heidelberg, 2008.
    
    \bibitem[Giles et~al.(2018)Giles and Szpruch]{giles2018}
    Giles~M~B, Szpruch~L.
    \newblock Multilevel monte carlo methods for applications in
    finance\allowbreak[J].
    \newblock High-Performance Computing in Finance, 2018:\allowbreak 197-247.
    
    \bibitem[Applebaum(2009)]{applebaum2009}
    Applebaum~D.
    \newblock {L{\'e}vy Processes and Stochastic Calculus}\allowbreak[M].
    \newblock Cambridge University Press, 2009.
    
    \bibitem[Sato(1999)]{ken1999}
    Sato~K~I.
    \newblock L{\'e}vy {Processes and Infinitely Divisible
        Distributions}\allowbreak[M].
    \newblock Cambridge university press, 1999.
    
    \bibitem[Hu et~al.(2018)Hu, Li, and Mao]{Hu2018274}
    Hu~L, Li~X, Mao~X.
    \newblock Convergence rate and stability of the truncated {Euler–Maruyama}
    method for stochastic differential equations\allowbreak[J].
    \newblock Journal of Computational and Applied Mathematics, 2018,
    337:\allowbreak 274-289.
    
    \bibitem[Li et~al.(2021)Li, Liu, and Tang]{li2021}
    Li~X, Liu~W, Tang~T.
    \newblock {Truncated Euler-Maruyama} method for time-changed stochastic
    differential equations with super-linear state variables and {H\"older's}
    continuous time variables\allowbreak[J].
    \newblock arXiv preprint arXiv:2110.02819, 2021.
    
    \bibitem[Wang et~al.(2013)Wang and Gan]{Wang2013466}
    Wang~X, Gan~S.
    \newblock The tamed {Milstein} method for commutative stochastic differential
    equations with non-globally {Lipschitz} continuous
    coefficients\allowbreak[J].
    \newblock Journal of Difference Equations and Applications, 2013, 19\allowbreak
    (3):\allowbreak 466-490.
    
\end{thebibliography}




%%%%%%%%%%%%%%%%%%% 附录 %%%%%%%%%%%%%%%%%%%%%%

% 添加附录, 如不需要可以注释
%\input{part/appendix}


%%%%%%%%%%%%%%%%%%%%%%%%%%%%%%%%%%%%%%%%%%%%%%

\backmatter  % 结束章节自动编号

%%%%%%%%%%%% 攻读学位期间的研究成果  %%%%%%%%%%%%

\input{part/publications}


%%%%%%%%%%%%%%%%%%% 致谢 %%%%%%%%%%%%%%%%%%%%%%

%
%%%%%%%%%%%%%%%%%%% 致谢 %%%%%%%%%%%%%%%%%%%%%

\begin{acknowledgement}
%\setlength{\baselineskip}{24pt}
行文至此, 竟迟迟难以落笔, 思绪万千......

在晚冬的寒意还未褪去的三月初, 上海经过连续几天的下雨后终于迎来了晴天, 上天好像安排了一场见面会一样, 我毫无征兆地走进了一教215, 后知后觉想起这是我备考研究生的自习室, 在这一瞬间, 我突然觉得人生某一阶段的结束并不是消散, 而在命运的齿轮里, 研究生阶段的充实、成长和精彩已经为我的人生留下了浓墨重彩的一笔. 现回首研究生学习生涯, 皆是幸运与感恩!

得遇良师, 人之幸事. 记得第一次加上刘老师的联系方式是在本科阶段, 由于班级工作需要班导师的签名, 刘老师的礼貌和耐心给我留下了谦谦君子的印象. 进入研究生, 我非常有幸成为了刘老师的学生, 同时也成为了大家羡慕的对象, 一位性格温暖、亦师亦友、指导有方的导师成为了我研究生的指明灯. 万千言语,终将汇为一句由衷的感谢!尊敬的刘老师, 感谢您对我学术上孜孜不倦的指导和包容; 感谢您鼓励我勇敢参加学术报告提升自己; 感谢您在我在面临毕业选择时的理解和支持; 感谢您的认真负责和悉心教导让我顺利完成研究生学业......桃李不言, 下自成蹊, 感谢研究生阶段遇到的每一位老师, 愿恩师们万事胜意, 桃李芬芳.

学贵得师, 亦贵得友. 很幸运在读研期间结交了很多志同道合的好友, 有爱的同学、温暖的室友、永远默默付出的同门和师门们以及陪伴我整个青春的人, 我们一起分享着彼此的青春, 感谢你们让平淡如水的日子熠熠生辉, 愿今后的我们继续以另一种方式陪伴着彼此, 继续相伴而行. 春晖寸草, 山高海深. 我最爱的家人们, 感谢你们的陪伴与支持, 感谢你们让我永远相信即使失败也会有靠山. 亲爱的爸爸妈妈、爷爷奶奶, 一直以来辛苦您们啦, 感谢你们为我创建的幸福家庭, 我一直在享受你们无私的爱, 未来, 我们一起去看遍万千山水吧!亲爱的弟弟, 还是觉得这辈子最幸运的事是成为你的姐姐, 感谢你给我搭建的堡垒, 感谢你永远出现在我最脆弱的时侯, 感谢你带来的一切......

最后, 感谢一路坚持的自己, 蜗牛慢慢爬也会当终点的. 做好自己的事, 对生活真诚、简单和快乐, 人生终究是幸福烂漫的!



\end{acknowledgement}




\end{document}

