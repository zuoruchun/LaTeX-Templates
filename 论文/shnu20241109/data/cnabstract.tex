
% 中文摘要和关键字

\begin{cnabstract}

%研究Back-Euler-Maruyama (BEM)数值逼近一类经过Lamperti变换后,漂移系数线性增长、扩散系数是常数的时变随机微分方程.证明了BEM的强收敛性与逆从属的稳定指数之间的关系,并讨论收敛速度.并通过数值模拟验证了理论结果.

本研究提出了一种新的数值方法,专门针对一类具有高度非线性的时间变换随机微分方程(SDE)。这些SDE的漂移项和扩散项系数,满足超线性增长条件。通过应用Lamperti变换,我们将这些方程中的乘性噪声转换为加性噪声,从而简化了数值解的计算并提高了收敛阶。本研究不仅探讨了变换后的SDEs的强收敛性,还详细分析了收敛阶。此外,我们还通过数值模拟验证了理论结果的有效性,展示了该方法在提高收敛阶方面的显著优势,为时间变换SDEs的数值分析提供了新的视角和工具。



\cnkeywords{时间变换;等距离散;收敛阶;逆从属;EM型数值方法;Lamperti变换;强收敛}

\end{cnabstract}

