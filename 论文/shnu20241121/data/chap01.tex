
% 引言

\chapter{引言}\label{chap:Intro}

\section{研究背景}\label{sec:background}

%近年来, 由时间变换的布朗运动驱动的随机微分方程收到了广泛关注, 他们在金融, 生物和物理等方面有着广泛的应用. Meerschaert和Scheffler 在\cite{meerschaert2004limit}中给出了从属和逆从属的重要性质, 这为后面对时间变换的研究做了大量铺垫. Kobayashi在\cite{kobayashi2011stochastic}中研究了一类可对偶化的时间变换SDE, 与原始SDE对比, 揭示了逆从属微分项的作用. Magdziarz在\cite{magdziarz2009stochastic}中研究了一类漂移项完全依赖于时间, 并且满足全局Lipschitz条件, 扩散项是加性噪声的时间变换SDE, 通过离散逆从属, 得到强收敛阶是0.5的数值方法. Jum和Kobayashi在\cite{jum2014strong}中, 研究了一类漂移项和扩散项都是依赖于状态和逆从属过程, 并且满足全局Lipschitz条件和Holder连续性条件的时间变换SDE, 利用对偶原则, 将时间变换SDE转换成一般的SDE, 利用一般SDE已有的收敛结果, 得到时间变换SDE的Euler-Maruyama(EM)的强收敛阶是0.5. Deng和Liu在\cite{deng2020semi}中, 研究了一类漂移项是依赖于状态和逆从属过程, 并且满足单边Lipschitz条件和Holder连续性条件的时间变换SDE, 同样利用对偶原则得到semi-implicit Euler-Maruyama的强收敛阶是0.5. Jin和Kobayashi在\cite{jin2021strong}中, 研究了一类漂移项和扩散项都是依赖于状态和时间, 并且满足全局Lipschitz条件和Holder连续性条件的时间变换SDE, 在\cite{magdziarz2009stochastic}的离散格式基础之上, 得到变步长的EM数值方法的强收敛阶是0.5. Shen等人在\cite{shen2023class}中, 研究了一类由时间变换的布朗运动驱动的分布依赖SDE, 其中漂移项和扩散项满足单边Lipschitz和Holder连续性条件, 给出了解的存在唯一性和稳定性. Li等人在\cite{li2023mckean}中, 研究了时间变换McKean-Vlasov SDE, 并给出了解的存在唯一性和稳定性的证明. He等人在\cite{he2024eta}中, 研究了漂移项和扩散项满足全局Lipschitz条件的时间变换SDE, 证明了解的p阶矩是$\eta$稳定的. Li等人在\cite{li2023transportation}中, 研究了时间变换SDE和时间变换脉冲SDE, 得到了这些方程解的规律满足二次传输不等式的结果. Wu等人\cite{wu2024mean}中, 研究了分步theta数值方法在非线性时间变换SDE中的均方稳定性, 并证明了该方法的强收敛性以及在有限时间内达到1的收敛率. Wen等人在\cite{wen2023strong}中, 研究了非自治时间变换McKean-Vlasov SDE, 通过交互粒子系统探讨了EM方法的强收敛性和收敛率, 并证明了该方法的收敛率是0.5. 

%近年来, 由时间变换的布朗运动驱动的随机微分方程收到了广泛关注, 他们在金融, 生物和物理等方面有着广泛的应用. 
%Meerschaert和Scheffler在他们的文章\cite{meerschaert2004limit}中,深入研究了连续时间随机行走(CTRWs)的极限行为,特别是在更新时间间隔具有无限均值的情况下。他们的文章不仅揭示了CTRWs的缩放极限是一个由经典稳定从属过程的逆过程所支配的Lévy运动,而且还证明了这一逆过程具有自相似性。这些发现对于理解异常扩散现象具有重要意义,并且为后续关于时间变换和分数阶动力学的研究提供了坚实的理论基础。
%
%Magdziarz在\cite{magdziarz2009stochastic}中探讨了一类具有时间依赖漂移项和加性噪声的次扩散过程的随机表示。这类过程的数学描述通常涉及分数阶福克-普朗克方程(fractional Fokker–Planck equations)。作者构建了一个随机过程,其概率密度函数是具有时间依赖漂移项的分数阶福克-普朗克方程的解。该研究提出了一个强收敛的近似方案,允许使用蒙特卡洛方法来近似这些方程的解。在证明过程中,对于由逆α-稳定从属过程驱动的随机积分的矩的结果至关重要。文章中,Magdziarz通过离散化逆从属过程,得到了一个强收敛阶为0.5的数值方法。该研究不仅提供了次扩散过程的随机表示,而且还通过蒙特卡洛模拟展示了该过程的数值解,为研究具有时间依赖漂移的次扩散过程提供了新的视角和工具。
%
%Kei Kobayashi在其研究论文\cite{kobayashi2011stochastic}中深入探讨了时间变换随机微分方程(SDEs)的特性,特别是那些涉及半鞅和时间变化的特定条件。他证明了在一定条件下,任何由时间变换半鞅驱动的随机积分都可以表示为由原始半鞅驱动的时间变换随机积分。这一发现直接导致了Itô公式的一个专门形式的推导。当原始半鞅是标准布朗运动时,经典的Itô随机微分方程,即由布朗运动驱动的带有漂移的方程,扩展到了涉及连续路径的时间变换的更广泛的随机微分方程。
%
%Jum和Kobayashi在他们的文章\cite{jum2014strong}中,进一步拓展了时间变换随机微分方程(SDEs)的研究领域。他们专注于一类特殊的SDEs,这些SDEs的漂移项和扩散项不仅依赖于状态变量,还依赖于一个逆从属过程。这些SDEs通常在全局Lipschitz条件和Holder连续性条件下进行研究。作者巧妙地运用对偶原则,将这些复杂的时间变换SDEs转化为更易于处理的标准SDEs形式。通过利用标准SDEs的已知收敛结果,他们成功地证明了时间变换SDEs的Euler-Maruyama(EM)近似方案具有0.5的强收敛阶。这一成果不仅为时间变换SDEs的数值模拟提供了理论支持,也为理解和模拟由时间变化过程驱动的复杂动态系统开辟了新的道路。
%
%Jin和Kobayashi在\cite{jin2019strong}中研究了时间变换布朗运动驱动的随机微分方程(SDEs)的强逼近问题。与之前Jum和Kobayashi在\cite{jum2014strong}中的研究不同,他们考虑的SDEs的系数依赖于时间变量$t$而不是逆从属$E(t)$。这种改变使得直接应用之前的方法变得困难。为了克服这一难题,Jin和Kobayashi采用了涉及随机驱动的Gronwall型不等式来控制误差过程的矩,并建立了随机时间变换指数矩存在的有用准则,以确保最终导出的误差界限是有限的。他们的工作不仅理论上重要,而且在数值分析复杂系统中表现出异常动态行为时具有实际意义,特别是在金融、物理和生物科学等领域。
%
%Deng和Liu在他们的研究\cite{deng2020semi}中,探讨了一类非线性时间变换随机微分方程(SDEs),这些方程的漂移系数依赖于状态变量和逆从属过程,并且满足单边Lipschitz条件和Holder连续性条件。他们提出了半隐式Euler-Maruyama(EM)方法来近似这类SDEs,并证明了该方法的强收敛性,收敛速率为0.5。这项工作不仅在理论上对时间变换SDEs的数值方法进行了深入分析,而且还在实践中展示了如何保持这类方程的渐近性质,即均方多项式稳定性。通过数值模拟,Deng和Liu进一步验证了他们的理论结果,为理解和模拟复杂动态系统提供了新的视角和工具。
%
%Jin和Kobayashi在他们的研究\cite{jin2021strong}中,针对一类具有随机时间变换的随机微分方程(SDEs)进行了深入研究。这些SDEs的独特之处在于它们不仅包含由随机时间变化驱动的漂移项,还包含由常规非随机时间变量驱动的漂移项。文章中,作者们面对的主要挑战是处理这两个漂移项的同时出现。Jin和Kobayashi采用了一种新的Gronwall型不等式方法,有效克服了这一挑战。此外,他们还探讨了系数的时变Lipschitz条件,这是对标准Lipschitz假设的重要扩展。在数值方法方面,他们提出了Milstein型逼近方案,并详细分析了强收敛速率。这些结果不仅在理论上推动了对时间变化SDEs数值方法的理解,而且在实际应用中,为模拟具有复杂动态行为的系统提供了重要的参考。
%
%Shen等人在他们的研究\cite{shen2023class}中,深入探讨了一类由时间变换布朗运动驱动的分布依赖随机微分方程(DDSDEs)。这些方程的特点是漂移项和扩散项不仅依赖于状态变量,还依赖于其概率分布。文章中,作者们证明了在单边Lipschitz和Holder连续性条件下,这类DDSDEs解的存在唯一性,并进一步研究了其稳定性。特别是,他们提供了充分的条件来保证解在不同意义上的稳定性,包括随机稳定性、随机渐近稳定性和全局随机渐近稳定性。此外,Shen等人还展示了如何通过相关的\textcolor{red}{平均}场随机微分方程在均方收敛的意义上近似DDSDEs的解。这项工作不仅丰富了分布依赖SDEs的理论,也为研究由时间变换过程驱动的复杂动力系统提供了新的视角和工具。
%
%Li等人在他们的研究\cite{li2023mckean}中,深入探讨了一类由时间变换布朗运动驱动的McKean-Vlasov随机微分方程(SDEs)。这类SDEs的特点是其漂移项和扩散项不仅依赖于状态变量,还依赖于其概率分布。文章证明了这类SDEs解的存在唯一性以及相对于初始数据和系数的稳定性。特别地,他们证明了在适当的条件下,这些SDEs的解在均方意义下可以由相关平均SDEs的解来近似。这项工作不仅丰富了McKean-Vlasov SDEs的理论,也为研究具有时间变换特性的复杂动力系统提供了新的视角和工具。
%
%He等人在他们的文章\cite{he2024eta}中,深入研究了一类由时间变换布朗运动驱动的随机泛函微分方程(SDEs)。这些SDEs的漂移项和扩散项满足全局Lipschitz条件,他们利用Lyapunov方法捕获了一些充分条件,以确保所考虑方程的解在p阶矩意义下是$\eta$-稳定的。随后,他们通过时间变换的伊藤公式和反证法,提出了一些新的$\eta$-稳定性均方标准。最后,他们提供了一些例子来展示他们主要结果的有效性。这项工作不仅为时间变换SDEs的稳定性分析提供了新的视角,而且对于理解和模拟具有时间变化特性的复杂动态系统具有重要意义。通过引入$\eta$-稳定性的概念,He等人的研究扩展了对随机系统稳定性类型的理解,包括多项式、指数和对数稳定性等。他们的发现为进一步探索时间变化SDEs的长期行为和应用提供了坚实的理论基础。
%
%Li等人在他们的文章\cite{li2023transportation}中,深入研究了由时间变换布朗运动驱动的随机微分方程(SDEs)以及由时间变换布朗运动驱动的脉冲随机微分方程。通过建立时间变换延迟Gronwall类不等式,并运用Girsanov变换论证,他们成功地建立了这些方程解的二次传输不等式(quadratic transportation inequalities)。这些传输不等式的结果不仅为理解时间变换SDEs的规律性提供了新的视角,而且对于研究异常扩散现象以及相关金融、物理和生物科学中的应用具有重要意义。特别是,他们对于时间变换SDEs和脉冲SDEs的解的传输不等式的研究成果,为进一步探索这些方程在复杂动态系统中的行为奠定了理论基础。
%
%Wu等人在他们的文章\cite{wu2024mean}中,深入研究了分步theta(SST)数值方法在非线性时间变换随机微分方程(SDEs)中的应用。这些SDEs的漂移系数可以超线性增长,而扩散系数满足全局Lipschitz条件。他们证明了SST方法的强收敛性,并展示了该方法在有限时间内达到了经典的1/2收敛率。此外,他们还探讨了时间变换SDEs的均方稳定性,并证明了数值解能够保持这一性质。通过两个具体的例子,Wu等人展示了他们理论结果的一致性。这项工作不仅为时间变换SDEs的数值近似提供了新的视角,而且对于理解和模拟具有复杂动态行为的系统提供了重要的理论支持。
%
% Wen等人在他们的文章\cite{wen2023strong}中,深入研究了非自治时间变换McKean-Vlasov随机微分方程(MV-SDEs)。这些方程包含了两个漂移项和一个扩散项,且漂移项之一由随机时间变化$E(t)$驱动,另一个由常规的非随机时间变量$t$
% t驱动。通过构建交互粒子系统,他们探讨了Euler-Maruyama(EM)方法在这些方程中的应用,并证明了该方法的强收敛性。此外,他们还确定了EM方法的收敛率为0.5,即经典的1/2收敛率。这一发现对于理解和模拟由时间变化过程驱动的复杂动态系统具有重要意义,尤其是在金融、物理和生物科学等领域的应用。Wen等人的工作不仅在理论上推动了对时间变化SDEs数值方法的理解,而且为实际应用中模拟具有复杂动态行为的系统提供了重要的参考。
%
%关于可Lamperti变换的SDE数值方法的收敛性研究,已有众多学者进行了深入探讨。
%
%Neuenkirch和Szpruch在他们的文章\cite{neuenkirch2014first}中,深入研究了可Lamperti变换的随机微分方程(SDEs)的数值方法收敛性。他们提出了一种针对定义在域内的、具有单边Lipschitz系数的一维SDEs的强近似方法。通过运用Lamperti变换,作者将原始SDE转化为具有加性噪声的SDE,并采用后向Euler-Maruyama(BEM)方案进行数值近似。他们证明了该方法在保持SDE定义域的同时,能够实现一阶强收敛率。这一成果不仅为SDEs的数值近似提供了新的视角,而且对于理解和模拟具有复杂动态行为的系统提供了重要的理论支持。
%
%Alfonsi在他们的文章\cite{Alfonsi2013602}中独立于Neuenkirch等人,研究了一大类经过Lamperti变换后漂移项满足单调条件的加性噪声随机微分方程(SDE)。这些SDEs在金融数学和生物数学等领域中具有重要的应用价值。Alfonsi通过构建隐式漂移的欧拉(drift implicit Euler)数值方案,证明了该方案对于所研究的SDEs具有一阶的强收敛性。
%
%Chen、Gan和Wang在他们的文章\cite{chen2021first}中提出了一种新的显式时间步进方案,称为Lamperti平滑截断(LST)方案,用于强逼近随机SIS流行病模型。该方案结合了Lamperti变换与显式截断方法,能够有效地保持原始随机微分方程(SDEs)的定义域。研究表明,LST方案在保持该定义域的同时,具有一阶均方收敛率。这一成果为流行病模型的数值近似提供了新的视角,尤其是在处理具有非全局单调性条件的SDEs时,展现了良好的收敛性能。Chen等人的研究不仅丰富了流行病学模型的数值方法理论,也为实际应用中的疾病传播动态模拟提供了有效的工具。
%
%Yang和Huang在他们的文章\cite{yang2021first}中,通过结合对数变换和Euler-Maruyama(EM)方法,为SIS流行病模型构建了一种保持正性的数值方法。这种方法不仅保持了原始随机微分方程(SDE)的定义域,而且在有限时间区间内对于所有p>0的p阶矩收敛速率为一阶。这一成果为理解和改进SDE数值方法提供了重要的理论基础和实践指导,特别是在处理具有正性保持需求的流行病模型时。Yang和Huang的研究强调了在数值模拟中保持模型属性的重要性,并为开发高效、准确的数值方案提供了新的视角。


近年来, 由时间变换的布朗运动驱动的随机微分方程收到了广泛关注, 通过引入时间变换过程来描述复杂的动态系统行为,为研究次扩散现象提供了重要工具,它们在金融, 生物和物理等方面有着广泛的应用. 
Meerschaert和Scheffler在\cite{meerschaert2004limit}中,深入研究了连续时间随机行走(CTRWs)的极限行为, 不仅揭示了CTRWs的缩放极限是一个由经典稳定从属过程的逆过程所支配的Lévy运动,而且还证明了这一逆过程具有自相似性。这些发现对于理解异常扩散现象具有重要意义,并且为后续关于时间变换和分数阶动力学的研究提供了坚实的理论基础。
Magdziarz在\cite{magdziarz2009stochastic}中探讨了一类具有时间依赖漂移项和加性噪声的次扩散过程的随机表示,这类过程的数学描述通常涉及分数阶福克-普朗克方程。Magdziarz通过离散化逆从属过程,提出了一个强收敛的近似方案,得到了一个强收敛阶为0.5的数值方法。
Kei Kobayashi在\cite{kobayashi2011stochastic}中研究了时变SDE的特性,特别是那些涉及半鞅和时间变换的特定条件, 他证明了在一定条件下,任何由时间变换半鞅驱动的随机积分都可以表示为由原始半鞅驱动的时间变换随机积分。
Jum和Kobayashi在\cite{jum2014strong}中,研究一类SDEs的漂移项和扩散项不仅依赖于状态变量,还依赖于一个逆从属过程。在全局Lipschitz条件和Holder连续性条件下, 运用对偶原则,将这些复杂的时间变换SDEs转化为更易于处理的标准SDEs形式. 通过利用标准SDEs的已知收敛结果,得到时间变换SDEs的Euler-Maruyama(EM)数值方法具有0.5的强收敛阶。
Jin和Kobayashi在\cite{jin2019strong}不同于Jum和Kobayashi在\cite{jum2014strong}中的研究,他们考虑的SDEs的系数依赖于时间变量而不是逆从属,这种改变使得应用对偶原则变得困难。为了克服这一难题,Jin和Kobayashi采用了随机驱动的Gronwall型不等式来控制误差过程的矩,并建立了随机时间变换指数矩存在的有用准则,最终得到EM数值方法具有0.5的强收敛阶
Deng和Liu在\cite{deng2020semi}中,研究了一类非线性时变SDEs,在漂移系数依赖于状态变量和逆从属过程,并且满足单边Lipschitz和Holder连续性条件下,提出了半隐式EM数值方法来近似这类SDEs,并证明了该方法的强收敛率为0.5。同时提出了如何保持这类方程的均方多项式稳定性。
Jin和Kobayashi在\cite{jin2021strong}中,研究了一类不仅包含由逆从属驱动的漂移项,还包含由常规时间变量驱动的漂移项,在系数满足时变Lipschitz条件下,提出了Milstein型逼近方案,并详细分析了强收敛速率。
Shen等人在\cite{shen2023class}中,研究了一类由时间变换布朗运动驱动的分布依赖随机微分方程(DDSDEs),这些方程的特点是漂移项和扩散项不仅依赖于状态变量,还依赖于其概率分布。在单边Lipschitz和Holder连续性条件下,得到这类DDSDEs解的存在唯一性,并进一步研究了其稳定性。
Li等人在\cite{li2023mckean}中,研究了一类由时间变换布朗运动驱动的McKean-Vlasov随机微分方程(SDEs),在适当的条件下,得到这类SDEs解的存在唯一性以及相对于初始数据和系数的稳定性,并且这些SDEs的解在均方意义下可以由相关平均SDEs的解来近似。
Li等人在\cite{li2023transportation}中,研究了时变SDEs和时变脉冲SDEs,利用时变延迟Gronwall类不等式和Girsanov变换,得到解的二次传输不等式。
Wu等人在\cite{wu2024mean}中,研究了非线性时变SDEs,在具有超线性增长的漂移系数和全局Lipschitz的扩散系数条件下,得到分步theta数值方法在有限时间内能够达到了经典的0.5强收敛率。
Wen等人在\cite{wen2023strong}中,研究了非自治时变McKean-Vlasov SDE,其中漂移项之一由逆从属驱动,另一个由常规时间变量驱动。通过构建交互粒子系统,得到EM数值方法的强收敛率为0.5。

截止目前对于时变SDE的研究,一方面对于逆从属的离散都是非等距的,这种格式的好处是逆从属在相邻的两个离散点的差值是非随机的,在后续处理时,可以直接将逆从属的增量转化成固定增量,这将大大简化证明步骤。坏处是在这种非等距的离散下,我们看不到收敛阶和逆从属的稳定指数的关系,为了保证逆从属的原始性质,我们渴望得到与稳定指数相关的收敛阶。另一方面对于EM型数值方法的收敛阶的研究中,在以往的工作大多都是得到0.5阶就够了,我们希望得到在一些特定情况下,EM型数值方法在时变SDE能够得到更高的收敛阶,在经典的SDE的研究中,Lamperti变换完美的解决了这一点。

关于Lamperti变换的SDE数值方法的收敛性研究,已有众多学者进行了深入探讨。
Neuenkirch和Szpruch在\cite{neuenkirch2014first}中,通过运用Lamperti变换,将原始SDE转化为漂移项满足单边Lipschitz的加性噪声SDE,并采用后向Euler-Maruyama(BEM)数值方法进行近似,得到该方法在保持SDE定义域的同时,能够实现一阶强收敛率。
Alfonsi在\cite{Alfonsi2013602}中独立于Neuenkirch等人,研究了一大类经过Lamperti变换后漂移项满足单调条件的加性噪声SDE,通过构建隐式EM数值方法,实现了一阶强收敛率。
Chen、Gan和Wang在\cite{chen2021first}中提出了Lamperti平滑截断(LST)方案,用于强逼近随机SIS流行病模型。该方案结合了Lamperti变换与显式截断方法,证明了在保持原始SDE定义域的同时,得到一阶均方收敛率。
Yang和Huang在他们的文章\cite{yang2021first}中,通过结合对数变换和EM数值方法,为SIS流行病模型构建了一种保持正性的数值方法。证明了在保持原始SDE定义域的同时,在有限时间区间内收敛速率为一阶。

在本文中,我们将研究Lamperti变换后的时变SDE,重点关注EM型数值方法的强收敛阶。相较于以往的EM型数值方法,我们不仅可以得到与经典SDE一致的1阶强收敛,另外通过对逆从属的等距离散,我们还建立了EM型数值方法的收敛阶和逆从属稳定指数的关系。

\section{主要结论}\label{sec:mainResults}

主要结论


\section{结构安排}

本文接下来的写作安排如下:


