
% 引言

\chapter{引言}\label{chap:Intro}

\section{研究背景}\label{sec:background}

近年来, 由时间变换的布朗运动驱动的随机微分方程收到了广泛关注, 他们在金融, 生物和物理等方面有着广泛的应用. Meerschaert和Scheffler 在\cite{meerschaert2004limit}中给出了从属和逆从属的重要性质, 这为后面对时间变换的研究做了大量铺垫. Kobayashi在\cite{kobayashi2011stochastic}中研究了一类可对偶化的时间变换SDE, 与原始SDE对比, 揭示了逆从属微分项的作用. Magdziarz在\cite{magdziarz2009stochastic}中研究了一类漂移项完全依赖于时间, 并且满足全局Lipschitz条件, 扩散项是加性噪声的时间变换SDE, 通过离散逆从属, 得到强收敛阶是0.5的数值方法. Jum和Kobayashi在\cite{jum2014strong}中, 研究了一类漂移项和扩散项都是依赖于状态和逆从属过程, 并且满足全局Lipschitz条件和Holder连续性条件的时间变换SDE, 利用对偶原则, 将时间变换SDE转换成一般的SDE, 利用一般SDE已有的收敛结果, 得到时间变换SDE的Euler-Maruyama(EM)的强收敛阶是0.5. Deng和Liu在\cite{deng2020semi}中, 研究了一类漂移项是依赖于状态和逆从属过程, 并且满足单边Lipschitz条件和Holder连续性条件的时间变换SDE, 同样利用对偶原则得到semi-implicit Euler-Maruyama的强收敛阶是0.5. Jin和Kobayashi在\cite{jin2021strong}中, 研究了一类漂移项和扩散项都是依赖于状态和时间, 并且满足全局Lipschitz条件和Holder连续性条件的时间变换SDE, 在\cite{magdziarz2009stochastic}的离散格式基础之上, 得到变步长的EM数值方法的强收敛阶是0.5. Shen等人在\cite{shen2023class}中, 研究了一类由时间变换的布朗运动驱动的分布依赖SDE, 其中漂移项和扩散项满足单边Lipschitz和Holder连续性条件, 给出了解的存在唯一性和稳定性. Li等人在\cite{li2023mckean}中, 研究了时间变换McKean-Vlasov SDE, 并给出了解的存在唯一性和稳定性的证明. He等人在\cite{he2024eta}中, 研究了漂移项和扩散项满足全局Lipschitz条件的时间变换SDE, 证明了解的p阶矩是$\eta$稳定的. Li等人在\cite{li2023transportation}中, 研究了时间变换SDE和时间变换脉冲SDE, 得到了这些方程解的规律满足二次传输不等式的结果. Wu等人\cite{wu2024mean}中, 研究了分步theta数值方法在非线性时间变换SDE中的均方稳定性, 并证明了该方法的强收敛性以及在有限时间内达到1的收敛率. Wen等人在\cite{wen2023strong}中, 研究了非自治时间变换McKean-Vlasov SDE, 通过交互粒子系统探讨了EM方法的强收敛性和收敛率, 并证明了该方法的收敛率是0.5. 

关于可Lamperti变换的SDE数值方法的收敛性研究,已有众多学者进行了深入探讨。Neuenkirch和Szpruch在他们的文章\cite{neuenkirch2014first}中,提出了一种针对定义在域内的单边Lipschitz系数的一维SDEs的强近似方法。他们通过Lamperti变换将原始SDE转换成具有加性噪声的SDE,并应用backward Euler-Maruyama(BEM)方案,证明了该方法在保持SDE定义域的同时,具有一阶强收敛率。Alfonsi在\cite{Alfonsi2013602}中独立于Neuenkirch等人,研究了一大类Lamperti变换后漂移项满足单调条件的加性噪声SDE,并得到一阶的强收敛阶。进一步地,Chen、Gan和Wang在\cite{chen2021first}中提出了一种新的显式时间步进方案,称为Lamperti平滑截断(LST)方案,用于强逼近SIS流行病模型。该方案基于Lamperti变换与显式截断方法的结合,能够在保持原始SDEs定义域的同时,证明具有一阶均方收敛率。最后,Yang和Huang在\cite{yang2021first}中,通过结合对数变换和Euler-Maruyama(EM)方法,为SIS流行病模型构建了一种保持正性的数值方法,并证明了该方法不仅保持了原始SDE的定义域,而且在有限时间区间内对于所有\( p > 0 \)的\( p \)阶矩收敛速率为一阶。这些研究为理解和改进SDE数值方法提供了重要的理论基础和实践指导。

%在这篇文章中,我们研究的随机微分方程是:
%\begin{equation}\label{basic SDE}
%	dX(s)=f(X(s))dE(s)+\sigma dB(E(s))
%\end{equation}
%这样离散,会引入一个必须面临的困难,在取期望的时候,不能再像之前的离散那样,将微分项的$dt$拿出来,这就引入了不得不解决的麻烦,对于$dE(t)$的分析.
%\cite{daley2003introduction}和\cite{magdziarz2009stochastic}中对于Cox这个更新过程的描述,对研究$E(t)$的期望起到了关键性作用.

\section{主要结论}\label{sec:mainResults}

主要结论


\section{结构安排}

本文接下来的写作安排如下:


