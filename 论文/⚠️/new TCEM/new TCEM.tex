% -*- coding: utf-8 -*-
% !TEX program = xelatex

%\documentclass[11pt,a4paper]{article}
\documentclass[12pt,final]{article}

\usepackage[UTF8]{ctex}
\usepackage{amsmath,amsthm,amssymb}
\usepackage{mathrsfs}
\usepackage{graphicx}
\usepackage{subfig}
%\usepackage[english]{babel}
\usepackage{color,xcolor}
\usepackage{enumitem}
\usepackage{lipsum} % 用于生成示例文本
\usepackage{float}
\usepackage{pifont}
\usepackage{tabularx}
\usepackage{booktabs}
\usepackage{array}
\usepackage{multirow,multicol}
\usepackage{longtable}
\usepackage{makecell}
\usepackage{anyfontsize}
%\usepackage{microtype}
\usepackage{geometry}
\usepackage{relsize} % For smaller command
%\geometry{left=1.25in,right=1.25in,top=1in,bottom=1in}
\geometry{left=3cm,right=3cm,top=3cm,bottom=3cm}
\setlength{\headheight}{18pt}
\setlength{\headsep}{18pt}
\setlength{\footskip}{25pt}

%----- 设置超链接 -----
\usepackage{hyperref}
\usepackage{cleveref}
\hypersetup{
	colorlinks=true,
	linkcolor=black,
	citecolor=blue,
	filecolor=blue,
	urlcolor=blue
}

% 允许多行公式跨页显示
\allowdisplaybreaks

%----- 定义页眉-----
\makeatletter
\newcommand\@header{}
\newcommand\header[1]{\def\@header{#1}}
\makeatother

%----- 页眉页脚 -----
\usepackage{fancyhdr}
\makeatletter
\pagestyle{fancy}
\fancyhf{}
\fancyhead[C]{\small\kaishu\@header}
\fancyfoot[C]{\small\thepage}
\renewcommand{\headrulewidth}{0.5pt}
\makeatother

%----- 设置英文字体 -----
%\usepackage[no-math]{fontspec}
%\usepackage{newtxtext}  % New TX font for text
%\setmainfont{TeX gyre Termes}  % Times New Roman 的开源复刻版本
%\setsansfont{TeX gyre Heros}   % Helvetica 的开源复刻版本
%\setmonofont{TeX gyre Cursor}  % Courier New 的开源复刻版本
%\setmainfont{Times New Roman}
%\setsansfont{Arial}
%\setmonofont{Courier New}

%----- 设置数学字体 -----
%\usepackage{newtxmath}
%\usepackage{mathptmx}

%----- 设置编号格式 -----
\numberwithin{equation}{section}
\numberwithin{figure}{section}
\numberwithin{table}{section}

%----- 重新设置图表公式 autoref -------
\renewcommand{\figureautorefname}{图}
\renewcommand{\tableautorefname}{表}
\renewcommand{\equationautorefname}{公式}

%----- 设置各种间距 -----
%\renewcommand{\baselinestretch}{1.35}
%\setlength{\parindent}{2em}
%\ziju{0.1}  % 控制中文字间距
%\setlength{\parskip}{3pt plus1pt minus1pt}

%----- 算法环境 -----
\usepackage{algorithm}
\usepackage{algpseudocode}
\floatname{algorithm}{算法}
\algrenewcommand\algorithmicrequire{\textbf{输入:}}
\algrenewcommand\algorithmicensure{\textbf{输出:}}

%----定义列表项的样式 -----
\usepackage{enumitem}
\setlist{nolistsep}

%-----设置图片的路径 -----
\graphicspath{{./figure/}{./figures/}}

%----- 使用 tabularx库并定义新的左右中格式 -----
\newcolumntype{L}{X}
\newcolumntype{C}{>{\centering \arraybackslash}X}
\newcolumntype{R}{>{\raggedleft \arraybackslash}X}
\newcolumntype{P}[1]{>{\centering \arraybackslash}p{#1}}

%----- 数学定理设置 -----
\theoremstyle{plain}
%\newtheorem{Definition}{定义}[section]
%\newtheorem{Proposition}{命题}[section]
%\newtheorem{Lemma}{引理}[section]
%\newtheorem{Theorem}{定理}[section]
%\newtheorem{Example}{例}
%\newtheorem{Corollary}{推论}[section]
%\newtheorem{Remark}{注}[section]
\renewcommand{\proofname}{proof}
%\newtheorem{Assumption}[Theorem]{假设} % [Theorem] 表示与定理共享编号
% 定义定理环境
\newtheorem{Theorem}{Theorem}[section]   % [section] 表示按章节编号
\newtheorem{Lemma}[Theorem]{Lemma}      % [Theorem] 表示与定理共享编号
\newtheorem{Corollary}[Theorem]{Corollary}  % [Theorem] 表示与定理共享编号
\newtheorem{Proposition}[Theorem]{Proposition} % [Theorem] 表示与定理共享编号
\newtheorem{Assumption}[Theorem]{Assumption} % [Theorem] 表示与定理共享编号
% 定义定义环境
\theoremstyle{Definition}
\newtheorem{Definition}[Theorem]{Definition}  % [Theorem] 表示与定理共享编号
\newtheorem{Example}[Theorem]{Example}       % [Theorem] 表示与定理共享编号
% 定义注释环境
\theoremstyle{Remark}
\newtheorem{Remark}[Theorem]{Remark}      % [Theorem] 表示与定理共享编号

% 自定义标签名称
\crefname{Theorem}{Theorem}{Theorems}
\Crefname{Theorem}{Theorem}{Theorems}
\crefname{Lemma}{Lemma}{Lemmas}
\Crefname{Lemma}{Lemma}{Lemmas}
\crefname{Corollary}{Corollary}{Corollaries}
\Crefname{Corollary}{Corollary}{Corollaries}
\crefname{Proposition}{Proposition}{Propositions}
\Crefname{Proposition}{Proposition}{Propositions}
\crefname{Assumption}{Assumption}{Assumptions}
\Crefname{Assumption}{Assumption}{Assumptions}
\crefname{Definition}{Definition}{Definitions}
\Crefname{Definition}{Definition}{Definitions}
\crefname{Example}{Example}{Examples}
\Crefname{Example}{Example}{Examples}
\crefname{Remark}{Remark}{Remarks}
\Crefname{Remark}{Remark}{Remarks}


\makeatletter
\renewenvironment{proof}[1][\proofname]{\par
	\pushQED{\qed}%
	\normalfont \topsep6\p@\@plus6\p@\relax
	\trivlist\item[\hskip\labelsep
	\bfseries #1\@addpunct{\,:\,}]\ignorespaces
}{%
	\popQED\endtrivlist\@endpefalse
}
\makeatother

%----- 参考文献格式 -----
%\bibliographystyle{plain} % abbrv, unsrt, siam
\bibliographystyle{thuthesis-numeric}
%\bibliographystyle{thuthesis-author-year}

%----- 参考文献引用格式 -----
\usepackage[numbers,sort&compress]{natbib}
%\usepackage[numbers,super,square,sort&compress]{natbib}
\def\bibfont{\small}  % 修改参考文献字体
\setlength{\bibsep}{7pt plus 3pt minus 3pt}  % 调整参考文献间距

%----- 微分符号 -----
\newcommand{\dif}{\mathop{}\!\mathrm{d}}

%----- 定义新命令 -----
\newcommand{\CC}{\ensuremath{\mathbb{C}}}
\newcommand{\RR}{\ensuremath{\mathbb{R}}}
\newcommand{\abs}[1]{\lvert#1\rvert}
\newcommand{\norm}[1]{\lVert#1\rVert}
\newcommand{\dx}[1][x]{\mathop{}\!\mathrm{d}#1}
\newcommand{\ii}{\mathrm{i}\mkern1mu} % imaginary
\newcommand{\refe}[2]{(\ref{#1})--(\ref{#2})}
\newcommand{\A}{\mathcal{A}}
\newcommand{\bA}{\boldsymbol{A}}
\newcommand{\red}[1]{\textcolor{red}{#1}}

%----- 论文信息 -----
\header{毕业论文}

\title{毕业论文}
\author{左如春}
\date{\today}


\begin{document}
	
	\maketitle
	
	
	\section{EM}
	由Taylor定理可知
	$$\begin{aligned}
		f(X(r))&=f(X(s))+f'(X(s))\big(X(r)-X(s)\big)\\&+\int_{0}^{1}(1-h)g''f\big(X(s)+h(X(r)-X(s))\big)\big(X(r)-X(s)\big)^{2}\:dh,
	\end{aligned}$$
	可以写成
	$$f(X(r))=f(X(s))+R_f(r;s,X(s)),\quad0\leq s\leq r,$$
	其中
	$$\begin{aligned}
		R_{f}(r;s,X(s))&:=f'(X(s))\big(X(r)-X(s)\big)\\&+\int_{0}^{1}(1-h)g''f\big(X(s)+h(X(r)-X(s))\big)\big(X(r)-X(s)\big)^{2}\:dh.\\
	\end{aligned}$$
	由于扩散项 $g$是常数,因此不做改变。
	将$f$ 的Taylor展开代入到下面的积分方程
	\begin{equation}\label{int}
		X(t)=X(s)+\int_s^tf(X(r))\:dr+\int_s^tg(X(r))\:dW(r),
	\end{equation}
	于是
	\begin{equation}\label{Mil}
		X(t)=X(s)+f(X(s))(t-s)+g(X(s))\:\int_{s}^{t}dW(r)+\int_{s}^{t}R_{f}(r;s,X(s))\:dr
	\end{equation}
	将这里的$R_M$去掉得到的就是Milstein方法
	
	\textcolor{red}{这个引理需要验证,应该是可以的。真实解的收敛性}
	\begin{Lemma}
		假设$f$和$g$满足全局Lipschitz条件,并且  X( t) 是原SDE的真实解. 那么对于任意的T>0和
		${X}_{0}\in \mathbb{R}$, 都存在 $K> 0$ 使得对于任意的 $0\leq s, t\leq T$都有
		\begin{equation}
			||X(t)-X(s)||_{L^2(\Omega,\mathbb{R}^d)}\leq K|t-s|^{1/2}.
		\end{equation}
	\end{Lemma}
	
	\begin{proof}
		由8.3
		$$X(t)-X(s)=\:\int_{s}^{t}f(X(r))\:dr+\int_{s}^{t}g(X(r))\:dW(r).$$
		Then, using (A.10),
		$$\mathbb{E}\left[\left|X(t)-X(s)\right|^2\right]\leq2\mathbb{E}\left[\left|\int_s^tf(X(r))\:dr\right|^2\right]+2\mathbb{E}\left[\left|\int_s^tg(X(r))\:dW(r)\right|^2\right].$$
		Following the argument for deriving (8.29), we find
		$$\mathbb{E}\Big[\left|X(t)-X(s)\right|^2\Big]\leq2L^2\big((t-s)+1\big)(t-s)\bigg(1+\sup\limits_{t\in[0,T]}\mathbb{E}\Big[\left|X(t)\right|^2\bigg]\bigg).$$
		As $X\in\mathcal{H}_{2,T}$, this implies (8.55).
	\end{proof} 
	
	\begin{Proposition}
		假设$f$和$g$满足全局Lipschitz条件,并且X(t) 是原SDE的真实解. 对于每个
		$T> 0$ 和 $X_0\in \mathbb{R} ^d,$ 存在 $K> 0$ 对于任意的 $0\leq s, t\leq T:$,使得下述成立
		(i) 对于$R_f$
		\begin{equation}
			\mathbb{E}\left[\left|R_{f}(t;s,X(s))\right|^2\right]\leq\:K|t-s|
		\end{equation}
		(ii) 对于$R_{f}$的积分,其中$t_{j}=j\Delta t$,
		$$\mathbb{E}\left[\left|\sum_{t_j<t} \int_{t_j}^{t\wedge t_{j+1}}R_{f}(r;t_j,X(t_j))\:dr \right|^2\right]\leq K\Delta t^2.$$
	\end{Proposition}
	
	\begin{proof}
		(i) 在全局Lipschitz的条件下,该结论可以直接得到\\
		(ii) 由$R_f$ 的定义可知
		$$\begin{aligned}R_{f}(r;s,X(s))&=f'(X(s))\big(X(r)-X(s)\big)\\
			&+\:\int_{0}^{1}(1-h)g''f\big(X(s)+h(X(r)-X(s))\big)\big(X(r)-X(s)\big)^{2}\:dh\\
			&=f'(X(s))\Bigg(g(X(s))\int_{s}^{r}dW(p)\Bigg)+f'(X(s))\left(f(X(s))(t-s)+\int_{s}^{t}R_{f}(r;s,X(s))\:dr\right)\\
			&+\:\int_{0}^{1}(1-h)g''f\big(X(s)+h(X(r)-X(s))\big)\big(X(r)-X(s)\big)^{2}\:dh.\end{aligned}$$
		令
		\begin{equation}
			\Theta_{j}:= \int_{t_{j}}^{t_{j+1}}f'(X(t_{j}))\biggl(g(X(t_{j}))\int_{t_{j}}^{r} dW(p)\biggr) dr 
		\end{equation}
		于是
		$$\begin{gathered}
			\mathbb{E}\left[\left|\sum_{j=0}^{k-1}\int_{t_{j}}^{t_{j+1}}R_{f}(r;t_{j},X(t_{j})) \, dr\right|^2\right] \leq 3\mathbb{E}\left[\left|\sum_{j=0}^{k-1}\Theta_{j}\right|^2\right] \\
			+ 3\mathbb{E}\left[\left|\sum_{j=0}^{k-1}\int_{t_j}^{t_{j+1}}f'(X(t_j))\left(f(X(t_j))(r-t_j)+\int_{t_j}^{r}R_{f}(p;t_j,X(t_j))\:dp\right) \, dr\right|^2\right] \\
			+ 3\mathbb{E}\left[\left|\sum_{j=0}^{k-1}\int_{t_j}^{t_{j+1}}\int_0^1(1-h)f''\big(X(t_j)+h(X(r)-X(t_j))\big)\big(X(r)-X(t_j)\big)^2 dh \, dr\right|^2\right].
		\end{gathered}$$
		利用Cauchy-Schwarz不等式以及前面的结论,得到第二项和第三项是$\Delta t^{2}$.对于第一项,对于 $k>j$,我们有 $\mathbb{E}[\langle\Theta_j,\Theta_k\rangle]=$
		$\mathbb{E}[\langle\Theta_j,\mathbb{E}[\Theta_k\mid\mathcal{F}_{t_k}]\rangle]=0$ a.s. 于是
		$$\mathbb{E}\left[\left|\sum_{j=0}^{k-1}\Theta_j\right|^2\right]=\sum_{j=0}^{k-1}\mathbb{E}\left[\left|\int_{t_j}^{t_{j+1}}f'(X(t_j))\left(g(X(t_j))\:\int_{t_j}^r\:dW(p)\right)dr\right|^2\right].$$
		由于$ \int _{t_{j}}^{t_{j+ 1}}\int _{t_{j}}^{r}dW( p)$ $dr\sim \operatorname { N} ( 0, \frac 13\Delta t^{3}I_{m})$, 求和中的每一项都是 $\mathcal{O}(\Delta t^{3})$
		于是最终的和是 $\mathcal{O}(\Delta t^{2})$ .
		
	\end{proof}
	
	
	下面我们来证明Milstein的强收敛阶. 为了方便证明, 我们定义一个连续版本的过程 $X_\Delta t(t)$ ,它和估计$X_n$ 在 $t=t_n.$是相等的, 为此, 介绍变量 $\hat{t}:=t_n$ 当$t_n\leq t<t_{n+1}$ 并且令
	$$\begin{aligned}
		X_{\Delta t}(t)&=X_{\Delta t}(\hat{t})+f(X_{\Delta t}(\hat{t}))\int_{\hat{t}}^{t}ds+G(X_{\Delta t}(\hat{t}))\int_{\hat{t}}^{t}dW(r)
	\end{aligned}$$
	注意,当$\hat{t}=t_n$时,	$X_{\Delta t}(\hat{t})=X_n$
	\begin{equation}\label{Milnum}
		X_{\Delta t}(t)=X_{\Delta t}(t_{0})+\int_{t_{0}}^{t}f(X_{\Delta t}(\hat{s}))ds+\int_{t_{0}}^{t}G(X_{\Delta t}(\hat{s}))dW(s)
	\end{equation}
	
	\begin{Theorem}
		假设满足全局Lipschitz条件.
		令$X_n$ 是精确解$X(t)$的Milstein估计. 对于任意的 $T\geq 0$, 存在 $K> 0$ 使得
		$$\sup\limits_{0\leq t_n\leq T}\left|X(t_n)-X_n\right|\leq K\Delta t.$$
	\end{Theorem}
	\begin{proof}
		为了简化符号, 令 $D(s):=f(X_{\Delta t}(\hat{s}))-f(X(\hat{s}))$ 以及
		$$M(s):=g(X_{\Delta t}(\hat{s}))-g(X(\hat{s}))$$
		数值解\ref{Milnum}和真实解\ref{Mil}相减,可以得到
		\begin{align}
			u_{\Delta t}(t)-X(t)& = \int_{0}^{t}D(s) ds+\int_{0}^{t}M(s) dW(s) \\
			&+\sum_{t_{j}<t}\int_{t_{j}}^{t\wedge t_{j+1}}R_{f}(r;t_{j},X(t_{j})) \, dr+\int_{t_n}^{t}R_{f}(r;t_n,X(t_{n})) \, dr. 
		\end{align}
		令$ e(\hat{t} )=X_{\Delta t}(\hat{t} )-X(\hat{t} ).  $于是,
		\begin{align}
			\mathbb{E}\left[\left|e(\hat{t})\right|^2\right]& \leq3\mathbb{E}\left[\left|\int_0^{\hat{t}}D(s) ds\right|^2\right]+3\mathbb{E}\left[\left|\int_0^{\hat{t}}M(s) dW(s)\right|^2\right] \\
			&+3\mathbb{E}\left[\left|\sum_{t_{j}<\hat{t}}\int_{t_{j}}^{t\wedge t_{j+1}}R_{f}(r;t_{j},X(t_{j})) \, dr\right|^2\right].
		\end{align}
		对于第一项,由Lipschitz条件和Cauchy-Schwarz不等式可以得到
		\begin{equation}
			\mathbb{E}\left[\left|\int_{0}^{\hat{t}}D(s) ds\right|^2\right]\leq\hat{t}\int_{0}^{\hat{t}}\mathbb{E}\left[\left|D(s)\right|^2\right]ds\leq\hat{t}\int_{0}^{\hat{t}}L^{2}\mathbb{E}\left[\left|e(s)\right|^{2}\right]ds.
		\end{equation}
		对于第二项,通过等距性质:
		\begin{equation}
			\mathbb{E}\left[\left|\int_{0}^{\hat{t}}M(s) dW(s)\right|^2\right]= \int_{0}^{\hat{t}}\mathbb{E}\left[\left|M(s)\right|_{\mathrm{F}}^{2}\right]ds.  
		\end{equation}
		由于前面的等式,可以得到:
		\begin{equation}
			\mathbb{E}\Big[\left|M(s)\right|_{\mathrm{F}}^{2}\Big] \leq 2\mathbb{E}\Big[\left|g(X_{\Delta t}(\hat{s}))-g(X(\hat{s}))\right|_{\mathrm{F}}^{2}\Big] \le
			\leq2(L+L_{2}) \mathbb{E}\left[\left|e(s)\right|^2\right]
		\end{equation}
		因此存在独立于 $\Delta t$的$K_3>0$使得 ,
		$$\mathbb{E}\left[\left|\int_0^{\hat{t}}M(s)\:dW(s)\right|^2\right]\leq K_3\int_0^{\hat{t}}\mathbb{E}\left[\left|e(s)\right|^2\right]ds.$$
		对于第三项, 由Proposition(ii)可知,存在独立于$\Delta t$的$K_2>0$使得
		$$\mathbb{E}\left[\left|\sum_{j=0}^{n-1}\int_{t_{j}}^{t\wedge t_{j+1}}R_{f}(r;t_{j},X(t_{j})) \, dr\right|^2\right]\leq K_2\Delta t^2.$$
		将这些估计放在一起, 于是对于 $t\in[0,T]$
		$$\mathbb{E}\Big[\left|e(\hat{t})\right|^2\Big]\leq3\big(TL^2+K_3\big)\int_0^{\hat{t}}\mathbb{E}\Big[\left|e(s)\right|^2\Big]\:ds+3K_2\Delta t^2.$$
		最终由Gronwall不等式即可完成这个证明.
		
	\end{proof}
	
	
	
	
	
	
	
	
	
	
	
	
\end{document}

