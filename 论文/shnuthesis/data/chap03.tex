
% 微分方程的数值方法

\chapter{BEM方法的强收敛性}

对于随机微分方程\cref{basic SDE},它的BEM数值格式是:
\begin{equation}\label{eq:1}
	X_{t_{i+1}}=X_{t_i}+f(X_{t_{i+1}})\Delta E_{i}+\sigma\Delta B_{E_{i}},\quad i=0,1,2,\ldots,\qquad X_0=X(0)
\end{equation}
其中$\Delta E_{i}=E(t_{i+1})-E(t_i)$以及$\Delta B_{E_{i}}=B(E{(t_{i+1})})-B(E({t_i}))$.
\begin{theorem}\label{main th}
	对于任意的$\epsilon>0$,令$\epsilon < T1 < T2$,在\cref{assum1},\cref{assum2}和\cref{assum3}的条件下,存在常数C,使得下面的不等式成立:
	$$\mathbb{E}[|X({t_i})-X_{t_i}|]\le C\Delta t^\alpha,\quad i=\lceil T1/\Delta t \rceil,\lceil T1/\Delta t \rceil+1 \ldots \lceil T2/\Delta t \rceil$$
\end{theorem}
\begin{proof}
	由于第一变量变换和第二变量变换公式\cref{first},\cref{second}的成立,使得我们可以考虑\cref{basic SDE}在$[t_i,t_{i+1})$的积分:
	\begin{align}
		\int_{t_i}^{t_{i+1}}dX(s)=\int_{t_i}^{t_{i+1}}f(X(s))dE(s)+\int_{t_i}^{t_{i+1}}\sigma dB(E(s))
	\end{align}
	等价于
	\begin{align}
		\int_{t_i}^{t_{i+1}}dX(s))=\int_{E_{t_i}}^{E_{t_{i+1}}}f(X(D(s-)))ds+\int_{E_{t_i}}^{E_{t_{i+1}}}\sigma dB(s)
	\end{align}
	针对于漂移项$f(X(D(s-)))$,下面等式恒成立:
	\begin{align}\label{eq:ito}
		\int_{E(t_i)}^{E(t_{i+1})} f(X(D(t_{i+1}-)))) - f(X(D(t-))) dt = \int_{E(t_i)}^{E(t_{i+1})} \int^{D(t_{i+1}-)}_{D(t-)} df(X(s)) dt
	\end{align}
	对于$df(X(s))$,由\cref{ito}的时间变换It\^{o}公式:
	\begin{align*}
		\begin{gathered}
			f(X(t))-f(0)=\int_{0}^{E(t)}f(X(D(s-)))f^{\prime}\left(X(D(s-))\right)+\frac{\sigma^{2}}{2}f^{\prime\prime}\big(X(D(s-))\big)ds \\
			+\int_{0}^{E(t)}\sigma f^{\prime}\big(X(D(s-))\big)dB(s)
		\end{gathered}
	\end{align*}
	于是\cref{eq:ito}变成
	\begin{equation}\label{eq:ito1}
		\begin{aligned}
			&\quad\int_{E(t_i)}^{E(t_{i+1})} f(X(D(t_{i+1}-))) - f(X(D(t-))) dt \\
			&= \int_{E(t_i)}^{E(t_{i+1})} \int_{t}^{t_{i+1}} \left( f(X(D(s))) f^{\prime}(X(D(s))) + \frac{1}{2} \sigma^2 f^{\prime\prime}(X(D(s))) \right) ds \, dt\\
			&\quad + \int_{E(t_i)}^{E(t_{i+1})} \int_{t}^{t_{i+1}} \sigma f^{\prime}(X(D(s))) \, dB(s) \, dt .
		\end{aligned}
	\end{equation}
	由\cref{basic SDE}与\cref{eq:ito1},以及\cite[Theorem 3.1]{kobayashi2011stochastic}可以得到
	\begin{align*}
		X(t_{i+1}) 
		&= X(t_i) + \int_{E(t_i)}^{E(t_{i+1})} f(X({D(t_{i+1})})) \, dt + \int_{t_i}^{t_{i+1}} \sigma \, dB(E(t)) \\
		&\quad + \int_{E(t_i)}^{E(t_{i+1})} \int_{t}^{t_{i+1}}\left( f(X(D(s))) f^{\prime}(X(D(s))) + \frac{1}{2} \sigma^2 f^{\prime\prime}(X(D(s))) \right) ds \, dt \\
		&\quad + \int_{E(t_i)}^{E(t_{i+1})} \int_{t}^{t_{i+1}}\sigma f^{\prime}(X(D(s)))) \, dB(s) \, dt \\
		&= X(t_i) + \int_{t_i}^{t_{i+1}} f(X({t_i)}) \, dE(t) + \int_{t_i}^{t_{i+1}} \sigma \, dB(E(t)) \\
		&\quad + \int_{t_i}^{t_{i+1}} \int_{E(t)}^{E(t_{i+1})} \left( f(X(D(s))) f^{\prime}(X(D(s))) + \frac{1}{2} \sigma^2 f^{\prime\prime}(X(D(s))) \right) ds \, dE(t) \\
		&\quad + \int_{t_i}^{t_{i+1}} \int_{E(t)}^{E(t_{i+1})}\sigma f^{\prime}(X(D(s))) \, dB(s) \, dE(t)\\
		&= X(t_i) + \int_{t_i}^{t_{i+1}} f(X({t_i})) \, dE(t) + \int_{t_i}^{t_{i+1}} \sigma \, dB(E(t)) \\
		&\quad + \int_{t_i}^{t_{i+1}} \int_{t}^{t_{i+1}} \left( f(X(s)) f^{\prime}(X(s)) + \frac{1}{2} \sigma^2 f^{\prime\prime}(X(s)) \right) dE(s) \, dE(t) \\
		&\quad + \int_{t_i}^{t_{i+1}} \int_{t}^{t_{i+1}}\sigma f^{\prime}(X(s)) \, dB(E(s)) \, dE(t)
	\end{align*}
	因此
	\begin{align}\label{eq:2}
		X(t_{i+1})
		&= X(t_i) + \int_{t_i}^{t_{i+1}} f(X({t_{i+1})}) \, dE(t) + \int_{t_i}^{t_{i+1}} \sigma \, dB(E(t)) + R_i
	\end{align}
	其中
	\begin{align*}
		R_i = &-\int_{t_i}^{t_{i+1}} \int_{t}^{t_{i+1}} \left( f(X(s)) f^{\prime}(X(s)) + \frac{1}{2} \sigma^2 f^{\prime\prime}(X(s)) \right) dE(s) \, dE(t)\\
		&-\int_{t_i}^{t_{i+1}} \int_{t}^{t_{i+1}} \sigma f^{\prime}(X(s)) \, dB(E(s)) \, dE(t)
	\end{align*}
	将$R_i$分解成$R_i = R_i^{(1)} + R_i^{(2)}$,其中:
	\begin{align*}
		& R_s^{(1)} = -\int_{t_i}^{t_{i+1}} \int_{t}^{t_{i+1}} \left( f(X(s)) f^{\prime}(X(s)) + \frac{1}{2} \sigma^2 f^{\prime\prime}(X(s)) \right) dE(s) \, dE(t).\\
		& R_s^{(2)} = -\int_{t_i}^{t_{i+1}} \int_{t}^{t_{i+1}} \sigma f^{\prime}(X(s)) \, dB(E(s)) \, dE(t)  
	\end{align*}
	和离散格式相减,即\cref{eq:2}-\cref{eq:1}得到:
	\begin{equation}
		X({t_{i+1}})-X_{t_{i+1}}=X({t_i})-X_{t_i}+(f{(X({t_{i+1}}))}-f{(X_{t_{i+1}})})\Delta E_{i}+R_{i}
	\end{equation}
	令$e_i = X({t_i})-X_{t_i}$
	由\cref{assum2}得到:
	\begin{equation}
		(1-K_1\Delta E_s)e_{s+1}\leq e(s)+R_{s},\quad\text{其中}s=\lceil T1/\Delta t \rceil,\lceil T1/\Delta t \rceil+1 \ldots \lceil T2/\Delta t \rceil
	\end{equation}
	定义$\gamma_l = 1-K_1\Delta E_l$,$N_1 = \lceil T1/\Delta t \rceil$以及$N_2 = \lceil T2/\Delta t \rceil$,\textcolor{red}{由$E$的次线性性质,我们可以取$(K_1\Delta E_l)_{max} = K'\Delta E \le K'' \Delta t$},从而可以得到
	\begin{equation}\label{bound}
		\prod\limits_{l=N_1}^{N_2}\gamma_l^{-1} < \frac{1}{(1-C\Delta t)^{N_2-N_1}} < \infty
	\end{equation}
	由\cref{lemma:1},我们可以得到
	$$e_{N_2} \leq \sum\limits_{j=N_1}^{N_2}R_{j}\prod\limits_{l=j}^{N_2}\gamma_l^{-1} \le C \sum\limits_{j=N_1}^{N_2}R_{j}$$
	于是结合\cref{bound},我们可以得到
	$$\mathbb{E}  |e_{N_2}| \leq C\mathbb{E}\left|\sum\limits_{j=N_1}^{N_2}R_{j}^{(1)} \right| + C\mathbb{E}\left|\sum\limits_{j=N_1}^{N_2}R_{j}^{(2)} \right|$$.
	下面先考虑第一项,由\cref{lemma:2}和\cref{assum3},可以得到
	\begin{equation}
		\mathbb{E} |R_j^{(1)}| \le C\mathbb{E} \left[ \int_{t_i}^{t_{i+1}}\int_{t_i}^{t}1dEsdEt \right] \le C\Delta t^{1+\alpha}
	\end{equation}
	于是
	\begin{equation}
		\mathbb{E}\left|\sum\limits_{j=N_1}^{N_2}R_{j}^{(1)}\right|\leq C\sum\limits_{N_1}^{N_2}\mathbb{E}\left|R_{j}^{(1)}\right| \leq
		C\sum\limits_{j=N_1}^{N_2}\Delta t^{1+\alpha} \le C\Delta t^\alpha
	\end{equation}
	因为
	\begin{align*}
		\mathbb{E}\left[ R_{k}^{(2)}|\mathcal{F}_{k\Delta t} \right] &= \mathbb{E}_D\mathbb{E}_B\left[ \int_{t_i}^{t_{i+1}} \int_{t_i}^{t} \sigma f^{\prime}(X(s)) \, dB(E(s)) \, dE(t) \right]\\
		&=\mathbb{E}_D\left[ \int_{t_i}^{t_{i+1}} \mathbb{E}_B\int_{t_i}^{t} \sigma f^{\prime}(X(s)) \, dB(E(s)) \, dE(t) \right]\\
		&= 0
	\end{align*}
	因此
	\begin{align*}
		\sum_{j=N_1}^{N_2}R_{j}^{(2)} 
	\end{align*}
	是鞅.我们知道由BDG不等式和\cref{lemma:2}可以得到:
	\begin{equation}
		\mathbb{E}[dB_EdE]^2=\mathbb{E}[(dB_E)^2(dE)^2]=\mathbb{E}_D[(dE)^2\mathbb{E}_B(dB_E)^2]\leq
		C\mathbb{E}_{D}[dE]^3\leq C\Delta t ^{1+2\alpha}
	\end{equation}
	于是对于第二项,由BDG不等式和Cauchy-Schwarz不等式,可以得到:
	\begin{equation*}
		\mathbb{E}\left|\sum_{j=N_1}^{N_2}R_{j}^{(2)}\right|  \le C\mathbb{E} \left|\sum_{j=N_1}^{N_2}(R_{j}^{(2)})^2\right|^{\frac{1}{2}} \le C\sqrt{\sum_{j=N_1}^{N_2}\mathbb{E}(R_{j}^{(2)})^2}
		\le C\sqrt{\sum_{j=N_1}^{N_2}\Delta t^{1+2\alpha}} \le C\Delta t^{\alpha}
	\end{equation*}
	综上所述,
	\begin{equation*}
		\mathbb{E} [e_n] \leq C\Delta t^\alpha
	\end{equation*}
\end{proof}
%	tips:We have also proved that $\mathbb{E} [\Delta E \Delta B_E ] \le C\Delta t ^{1+\frac{\alpha}{2}}$,however it seems that it's nothing help here.