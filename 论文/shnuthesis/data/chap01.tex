
% 引言

\chapter{引言}\label{chap:Intro}

\section{研究背景}\label{sec:background}

受到\cite{Alfonsi2013602}的启发,本文的目的是推导出一个有效的数值逼近一类一维随机微分方程.这类随机微分方程可以通过Lamperti变换,将漂移项变换成满足单调条件的,见\cite{iacus2008simulation},通过BEM对变换后的时变SDE进行逼近,再将其变换成原始时变SDE的逼近格式.

对于Lamperti变换的随机微分方程, 截止目前已经有了很多结果.\cite{neuenkirch2014first}研究了一大类Lamperti变换后随机微分方程的漂移项是局部Lipschitz的,并得到1阶的强收敛阶. \cite{yang2021first}研究了Lamperti变换作用在SIS模型中,并且得到1阶强收敛阶

对于时间变换的随机微分方程,有关收敛阶的研究,截止目前已经有而很多结果.\cite{wen2023strong}
研究了非自治的时间变换McKean-Vlasov随机微分方程,并给出EM方法的强收敛性和收敛阶.\cite{liu2020truncated}.研究截断EM方法来逼近一类具有Hölder连续性和超线性增长的非自治随机微分方程,并证明了强收敛性和收敛阶.\cite{jin2021strong}
研究一类具有时间变换Lipschitz界限下,包含随机和非随机积分项的时间变换随机微分方程,并讨论EM方法和Ito-Taylor方法的强收敛性和收敛阶.\cite{li2023convergence}
研究一类具有超线性增长的,包含随机和非随机积分项的时间变换随机微分方程,并讨论截断EM方法的强收敛性和收敛阶.在\cite{jum2014strong}中证明了对于漂移项和扩散项都满足全局Lipschitz时,可以采用对偶原则,将时间变换随机微分方程转换成一般随机微分方程,通过对一般随机微分方程使用EM数值格式进行逼近,进而得到时间变换随机微分方程的强收敛阶.在\cite{deng2020semi}中将这一思想运用在漂移项满足非全局Lipschitz条件时,使用半隐式EM得到强收敛阶,并研究了其稳定性.这解决了一大类可以对偶化的时间变换微分方程的数值格式收敛性问题.在\cite{jin2019strong},对于不能使用对偶原则的时间变换随机微分方程,采用一种非等距的离散格式,并得到强收敛阶.对于时间变换的随机微分方程,现在已经有了很多的数值格式,但是到目前为止这些数值格式,大多采用的是非等距离散,对于$E(t)$的离散,通过对$t$非等距离散,使之变成实际上是对$E(t)$的等距离散.而在这篇文章中,我们将对$t$采用等距离散来研究随机微分方程:
\begin{equation}\label{basic SDE}
	dX(s)=f(X(s))dE(s)+\sigma dB(E(s))
\end{equation}
这样离散,会引入一个必须面临的困难,在取期望的时候,不能再像之前的离散那样,将微分项的$dt$拿出来,这就引入了不得不解决的麻烦,对于$dE(t)$的分析.
\cite{daley2003introduction}和\cite{magdziarz2009stochastic}中对于Cox这个更新过程的描述,对研究$E(t)$的期望起到了关键性作用.


\section{主要结论}\label{sec:mainResults}

主要结论


\section{结构安排}

本文接下来的写作安排如下:


