% -*- coding: utf-8 -*-
% !TEX program = xelatex

%\documentclass[11pt,a4paper]{article}
\documentclass[12pt,final]{article}

\usepackage[UTF8]{ctex}
\usepackage{amsmath,amsthm,amssymb}
\usepackage{mathrsfs}
\usepackage{graphicx}
\usepackage{subfig}
%\usepackage[english]{babel}
\usepackage{color,xcolor}
\usepackage{enumitem}
\usepackage{float}
\usepackage{pifont}
\usepackage{tabularx}
\usepackage{booktabs}
\usepackage{array}
\usepackage{multirow,multicol}
\usepackage{longtable}
\usepackage{makecell}
\usepackage{anyfontsize}
%\usepackage{microtype}
\usepackage{geometry}

\geometry{left=1.25in,right=1.25in,top=1in,bottom=1in}
%\geometry{left=3cm,right=3cm,top=3cm,bottom=3cm}
\setlength{\headheight}{18pt}
\setlength{\headsep}{18pt}
\setlength{\footskip}{25pt}

%----- 设置超链接 -----
\usepackage{hyperref}
\hypersetup{
	colorlinks=true,
	linkcolor=black,
	citecolor=blue,
	filecolor=blue,
	urlcolor=blue
}

% 允许多行公式跨页显示
\allowdisplaybreaks

%----- 定义页眉-----
\makeatletter
\newcommand\@header{}
\newcommand\header[1]{\def\@header{#1}}
\makeatother

%----- 页眉页脚 -----
\usepackage{fancyhdr}
\makeatletter
\pagestyle{fancy}
\fancyhf{}
\fancyhead[C]{\small\kaishu\@header}
\fancyfoot[C]{\small\thepage}
\renewcommand{\headrulewidth}{0.5pt}
\makeatother

%----- 设置英文字体 -----
%\usepackage[no-math]{fontspec}
%\usepackage{newtxtext}  % New TX font for text
%\setmainfont{TeX Gyre Termes}  % Times New Roman 的开源复刻版本
%\setsansfont{TeX Gyre Heros}   % Helvetica 的开源复刻版本
%\setmonofont{TeX Gyre Cursor}  % Courier New 的开源复刻版本
%\setmainfont{Times New Roman}
%\setsansfont{Arial}
%\setmonofont{Courier New}

%----- 设置数学字体 -----
%\usepackage{newtxmath}
%\usepackage{mathptmx}

%----- 设置编号格式 -----
\numberwithin{equation}{section}
\numberwithin{figure}{section}
\numberwithin{table}{section}

%----- 重新设置图表公式 autoref -------
\renewcommand{\figureautorefname}{图}
\renewcommand{\tableautorefname}{表}
\renewcommand{\equationautorefname}{公式}

%----- 设置各种间距 -----
%\renewcommand{\baselinestretch}{1.35}
%\setlength{\parindent}{2em}
%\ziju{0.1}  % 控制中文字间距
%\setlength{\parskip}{3pt plus1pt minus1pt}

%----- 算法环境 -----
\usepackage{algorithm}
\usepackage{algpseudocode}
\floatname{algorithm}{算法}
\algrenewcommand\algorithmicrequire{\textbf{输入:}}
\algrenewcommand\algorithmicensure{\textbf{输出:}}

%----定义列表项的样式 -----
\usepackage{enumitem}
\setlist{nolistsep}

%-----设置图片的路径 -----
\graphicspath{{./figure/}{./figures/}}

%----- 使用 tabularx库并定义新的左右中格式 -----
\newcolumntype{L}{X}
\newcolumntype{C}{>{\centering \arraybackslash}X}
\newcolumntype{R}{>{\raggedleft \arraybackslash}X}
\newcolumntype{P}[1]{>{\centering \arraybackslash}p{#1}}

%----- 数学定理设置 -----
\theoremstyle{plain}
\newtheorem{definition}{定义}[section]
\newtheorem{proposition}{命题}[section]
\newtheorem{lemma}{引理}[section]
\newtheorem{theorem}{定理}[section]
\newtheorem{example}{例}
\newtheorem{corollary}{推论}[section]
\newtheorem{remark}{注}[section]
\renewcommand{\proofname}{证明}
\makeatletter
\renewenvironment{proof}[1][\proofname]{\par
	\pushQED{\qed}%
	\normalfont \topsep6\p@\@plus6\p@\relax
	\trivlist\item[\hskip\labelsep
	\bfseries #1\@addpunct{\,:\,}]\ignorespaces
}{%
	\popQED\endtrivlist\@endpefalse
}
\makeatother

%----- 参考文献格式 -----
%\bibliographystyle{plain} % abbrv, unsrt, siam
\bibliographystyle{thuthesis-numeric}
%\bibliographystyle{thuthesis-author-year}

%----- 参考文献引用格式 -----
\usepackage[numbers,sort&compress]{natbib}
%\usepackage[numbers,super,square,sort&compress]{natbib}
\def\bibfont{\small}  % 修改参考文献字体
\setlength{\bibsep}{7pt plus 3pt minus 3pt}  % 调整参考文献间距

%----- 微分符号 -----
\newcommand{\dif}{\mathop{}\!\mathrm{d}}

%----- 定义新命令 -----
\newcommand{\CC}{\ensuremath{\mathbb{C}}}
\newcommand{\RR}{\ensuremath{\mathbb{R}}}
\newcommand{\abs}[1]{\lvert#1\rvert}
\newcommand{\norm}[1]{\lVert#1\rVert}
\newcommand{\dx}[1][x]{\mathop{}\!\mathrm{d}#1}
\newcommand{\ii}{\mathrm{i}\mkern1mu} % imaginary
\newcommand{\refe}[2]{(\ref{#1})--(\ref{#2})}
\newcommand{\A}{\mathcal{A}}
\newcommand{\bA}{\boldsymbol{A}}
\newcommand{\red}[1]{\textcolor{red}{#1}}




\begin{document}
	
	
	\newpage
	\pagenumbering{roman} % 保持罗马数字编号
	\tableofcontents
	\section{引言}
	这是正文的开始,当前页码为0。
	\section{引言}
	这是正文的开始,当前页码为0。
	\section{引言}
	这是正文的开始,当前页码为0。
	\newpage
	\setcounter{page}{0} % 将页码设置为0
	\pagenumbering{arabic} % 从0开始使用阿拉伯数字编号
	
	\section{Euler-Maruyama格式的收敛速度(当 $H\not\equiv0$ 时)}
	\footnotetext{本文翻译自Strong approximation of time-changed stochastic
		differential equations involving drifts with random and
		non-random integrators的第四章.}
	本节讨论了Euler-Maruyama类型逼近方案在解SDE(3.1)时的强收敛速度,其中 $H(u,x)=H(u)$,并在额外假设 $\mathcal{H}_1$ 和 $\mathcal{H}_2$ 的基础上,给出了SDE系数的两组不同假设条件。这些不同的假设条件导致不同的收敛速度。本节所证明的结果回答了第1节中提出的问题 $(\mathbf{A})$,并将[10]中定理3.1的结果推广到当两个漂移项$H$和$F$同时出现的情况。然而,正如第1节中讨论的那样,本文采取的方法与[10]中的方法完全不同,因为对偶原则未被使用。
	
	首先,我们描述一个用于逆从属 $E$ 的逼近过程,该过程在[15,16]中给出。固定一个等距的步长 $\delta\in(0,1)$ 和一个时间范围 $T>0$。为了在区间 $[0,T]$ 上逼近 $E$,我们首先模拟从属过程 $D$ 的样本路径,该过程具有独立和平稳增量,通过设置 $D_0=0$,然后根据规则:$D_{i\delta} := D_{(i-1)\delta} + Z_i, i=1,2,3,\dots$,其中$\left\{Z_i\right\}_{i\in\mathbb{N}} $是独立同分布的序列,满足$Z_i \stackrel{d}{=} D_\delta.$
	我们在找到整数 $N$ 满足 $T \in [D_{N\delta}, D_{(N+1)\delta})$ 时停止此过程。注意,$\mathbb{N} \cup \{0\}$ 值的随机变量 $N$ 确实存在,因为 $D_t \to \infty$ , 当 $t \to \infty$ 时几乎必然成立。为了生成随机变量 $\{Z_i\}$,可以使用[2]中第6章所给的算法。接下来,定义
	
	\[
	E_t^\delta := \left(\min\{n \in \mathbb{N}; D_{n\delta} > t\} - 1\right) \delta.
	\]
	
	过程 $E^\delta = (E_t^\delta)_{t \geq 0}$ 的样本路径是单调递增的阶梯函数,具有常数跳跃大小 $\delta$,第 $i$ 个等待时间由 $Z_i = D_{i\delta} - D_{(i-1)\delta}$ 给出。实际上,很容易看出,对于 $n = 0, 1, 2, \dots, N$,
	
	\[
	E_t^\delta = n\delta \quad \text{当且仅当} \quad t \in [D_{n\delta}, D_{(n+1)\delta}).
	\]
	
	特别的, $E_T^\delta=N\delta.$ 过程 $E^\delta$ 有效地近似了[10,16]中的$E$ ; 也就是几乎处处有, 
	
	\begin{equation}\label{1}
		E_t-\delta\leq E_t^\delta\leq E_t\quad\text{ 对于所有的 }t\in[0,T].
	\end{equation}
	
	接下来, 对于$n=0,1,2,\ldots,N,$令
	$$\tau_n=D_{n\delta}.$$
	
	由于假设 $B$ 和 $D$ 之间的独立性,我们可以在时间步长 $\{0, \delta, 2\delta, \ldots, N\delta\}$ 上逼近布朗运动 $B$。有了这个假设,定义离散时间过程 $(X_{\tau_n}^\delta)_{n \in \{0, 1, 2, \ldots, N\}}$,其中 $X_0^\delta := x_0$,并且对于 $n = 0, 1, 2, \ldots, N-1$,有:
	\begin{align}\label{2}
		X_{\tau_{n+1}}^\delta &:= X_{\tau_n}^\delta + H(E_{\tau_n}^\delta)(\tau_{n+1} - \tau_n) + F(E_{\tau_n}^\delta, X_{\tau_n}^\delta)(E_{\tau_{n+1}}^\delta - E_{\tau_n}^\delta) \\
		& \quad + G(E_{\tau_n}^\delta, X_{\tau_n}^\delta)(B_{E_{\tau_{n+1}}^\delta} - B_{E_{\tau_n}^\delta}),\nonumber
	\end{align}
	
	由于 $E_{\tau_n}^\delta = n\delta$,上式等价于:
	\begin{align}\label{3}
		X_{\tau_{n+1}}^\delta := X_{\tau_n}^\delta + H(n\delta)(\tau_{n+1} - \tau_n) + F(n\delta, X_{\tau_n}^\delta)\delta + G(n\delta, X_{\tau_n}^\delta)(B_{(n+1)\delta} - B_{n\delta}).
	\end{align}
	
	特别地,当 $H \equiv F \equiv 0$ 且 $G \equiv 1$ 时,$(X_{\tau_n}^\delta)_{n \in \{0, 1, 2, \ldots, N\}}$ 变为一个离散化的时间变换布朗运动,其样本路径如图 2 所示。请注意,尽管表达式 (4.3) 可能看起来像是采取了非随机的时间步长,但我们实际上是通过随机时间步长 $\tau_0, \tau_1, \tau_2, \ldots, \tau_N$ 来离散化驱动过程 $E = (E_t)_{t \geq 0}$ 和 $B \circ E = (B_{E_t})_{t \geq 0}$,正如在(\ref{2})中所示;因此,时间变换过程的随机捕获事件(即产生恒定周期的事件)的关键特征,实际上是通过随机步长 $\tau_{n+1} - \tau_n = D_{(n+1)\delta} - D_{n\delta} \overset{\mathrm{d}}{=} D_\delta$ 来捕捉的。还要注意,我们并没有通过非随机时间步长来离散化SDE,因为这在实践中是困难的,因为驱动过程 $E$ 和 $B \circ E$ 既没有独立增量,也没有平稳增量。
	
	为了定义一个连续时间过程 $(X_t^\delta)_{t \in [0, T]}$,我们采用连续插值方法;即,对于 $s \in [\tau_n, \tau_{n+1})$,
	\begin{equation}\label{4}
		X_s^\delta := X_{\tau_n}^\delta + \int_{\tau_n}^s H(E_{\tau_n}) \, \mathrm{d}r + \int_{\tau_n}^s F(E_{\tau_n}, X_{\tau_n}^\delta) \, \mathrm{d}E_r + \int_{\tau_n}^s G(E_{\tau_n}, X_{\tau_n}^\delta) \, \mathrm{d}B_{E_r}.
	\end{equation}
	
	令
	\[
	n_t = \max\{n \in \mathbb{N} \cup \{0\}; \tau_n \leq t\} 
	\]
	
	
	那么显然对于任何 \( t > 0 \),都有 \( \tau_{n_{t}} \leq t < \tau_{n_{t}+1} \)。使用 (\ref{3}) 和恒等式
	\[
	X_{s}^{\delta} - x_{0} = \sum_{i=0}^{n_{s}-1} \left(X_{\tau_{i+1}}^{\delta} - X_{\tau_{i}}^{\delta}\right) + \left(X_{s}^{\delta} - X_{\tau_{n_{s}}}^{\delta}\right),
	\]
	我们可以将 \( X_{s}^{\delta} - x_{0} \) 表示为
	\[
	\sum_{i=0}^{n_s-1} \left[ H(E_{\tau_i})(\tau_{i+1} - \tau_i) + F(E_{\tau_i}, X_{\tau_i}^\delta) \delta + G(E_{\tau_i}, X_{\tau_i}^\delta) (B_{(i+1)\delta} - B_{i\delta}) \right] + \left( X_s^\delta - X_{\tau_{n_s}}^\delta \right).
	\]
	
	其中我们使用了 $i\delta = E_{D_{i\delta}} = E_{\tau_i}$。利用 (4.4) 以及 $\tau_i = \tau_{n_r}$ 对于任何 $r \in [\tau_i, \tau_{i+1})$,我们可以将后者重新写为方便的形式:
	\begin{equation}\label{5}
		X_s^\delta = x_0 + \int_0^s H(E_{\tau_{n_r}}) \, \mathrm{d}r + \int_0^s F(E_{\tau_{n_r}}, X_{\tau_{n_r}}^\delta) \, \mathrm{d}E_r + \int_0^s G(E_{\tau_{n_r}}, X_{\tau_{n_r}}^\delta) \, \mathrm{d}B_{E_r}.
	\end{equation}
	
	现在我们可以陈述本文的第一个主要定理,其中我们假设存在常数 $K > 0$ 和 $\theta \in (0, 1]$,使得对于所有 $u, v \geq 0$ 和 $x \in \mathbb{R}$,满足
	\[
	\mathcal{H}_3: |H(u) - H(v)| + |F(u, x) - F(v, x)| + |G(u, x) - G(v, x)| \leq K |u - v|^\theta (1 + |x|).
	\]
	
	回忆一下,步长为 $\delta > 0$ 的近似过程 $X^\delta$ 称为在 $[0, T]$ 上以 \textit{阶数} $\eta \in (0, \infty)$ 强收敛到解 $X$,如果存在有限的正常数 $C$ 和 $\delta_0$,使得对于所有 $\delta \in (0, \delta_0)$,有
	\[
	\mathbb{E} \left[ \sup_{0 \leq t \leq T} |X_t - X_t^\delta| \right] \leq C \delta^\eta.
	\]
	
	\begin{theorem}\label{th}
		设 $X$ 是满足随机微分方程 (3.1) 的解,且 $H(u, x) = H(u)$ 对所有 $(u, x) \in [0, T] \times \mathbb{R}$ 成立,并假设条件 $\mathcal{H}_1, \mathcal{H}_2$ 和 $\mathcal{H}_3$ 均成立。假设 $D$ 的拉普拉斯指数 $\psi$ 在 $\infty$ 处是 \textit{正则变化的},其指数为 $\beta \in (0, 1)$,且满足以下任一条件:
		
		\begin{itemize}
			\item[(a)] $h$ 为常数,且 $\beta \in (1/2, 1)$;
			\item[(b)] $h$ 是连续、单调递增的函数,并在 $\infty$ 处正则变化,指数为 $q \geq 0$,且 $\beta \in \left(\frac{2q + 1}{2q + 2}, 1\right)$。
		\end{itemize}
		设 $X^\delta$ 为定义在 (\ref{3})和 (\ref{4}) 中的 Euler-Maruyama 类型的近似过程。那么,存在一个常数 $C > 0$(与 $\delta$ 无关),使得对于所有 $\delta \in (0, 1)$,有
		\[
		\mathbb{E} \left[\sup_{0 \leq s \leq T} |X_s - X_s^\delta|\right] \leq C \delta^{\min\{\theta, 1/2\}}.
		\]
		因此,$X^\delta$ \textit{以阶数} $\min\{\theta, 1/2\}$ \textit{强收敛于} $X$,且在 $[0, T]$ 上 \textit{一致收敛}。
	\end{theorem}
	
	\begin{lemma}
		在定理\ref{th}的假设下,对于任意 $t \geq s \geq 0$,有
		\[
		\mathbb{E}_B[|X_t - X_s|] \leq \sqrt{2} h(E_t) \mathbb{E}_B[Y_t^{(2)}]^{1/2} \left\{(t - s) + (E_t - E_s)^{1/2} + (E_t - E_s)\right\},
		\]
		其中 $Y_t^{(2)}$ 定义见命题 3.1。
	\end{lemma}
	\begin{proof}
		通过Cauchy–Schwartz不等式, $\mathbb{E}_B[|X_t-X_s|]$ 可以被控制,
		\begin{gather*}
			\mathbb{E}_{B}\left[\int_{s}^{t}|H(E_{r})|\mathrm{d}r\right]+\mathbb{E}_{B}\left[\int_{s}^{t}|F(E_{r},X_{r})|\mathrm{d}E_{r}\right]+\mathbb{E}_{B}\left[\left|\int_{s}^{t}G(E_{r},X_{r})\mathrm{d}B_{E_{r}}\right|^{2}\right]^{1/2} \\
			\leq(t-s)h(E_t)+(E_t-E_s)h(E_t)\mathbb{E}_B[Y_t^{(1)}]+\sqrt{2}(E_t-E_s)^{1/2}h(E_t)\mathbb{E}_B[Y_t^{(2)}]^{1/2}. 
		\end{gather*}
		Jensen's 不等式给出了定理中的界限。
	\end{proof}
	
	现在我们准备证明定理\ref{th},其中局部化序列 $S_\ell$ 定义为:
	\[
	S_\ell := \inf\left\{ t \geq 0 : \sup_{0 \leq s \leq t} \left\{|X_s - X_s^\delta|\right\} > \ell \right\}
	\]
	
	下面我们准备证明定理\ref{th},该序列使得我们可以将过程 $(\sup_{0 \leq s \leq t} \{|X_s - X_s^\delta|\})_{t \in [0, T]}$ 视为有界的。然而,正如在 3.1 备注中所述,为了阐明证明的主要思路,我们省略了 $S_\ell$,并假设过程是有界的。相同的论证适用于第 4 和第 5 节中的所有论断。
	
	
	\textbf{定理1.1 的证明} \quad 令 $Z_t := \sup_{0 \leq s \leq t} |X_s - X_s^\delta|$,其中 $t \in [0, T]$。根据 (3.1) 和 (\ref{5}) 中分别给出的 $X$ 和 $X^\delta$ 的表示式,得到:
	\[
	Z_t \leq I_1 + I_2 + I_3,
	\]
	其中:
	\[
	\begin{aligned}
		I_1 & := \sup_{0 \leq s \leq t} \left| \int_0^s (H(E_r) - H(E_{\tau_{n_r}})) \mathrm{d}r \right|, \\
		I_2 & := \sup_{0 \leq s \leq t} \left| \int_0^s (F(E_r, X_r) - F(E_{\tau_{n_r}}, X_{\tau_{n_r}}^\delta)) \mathrm{d}E_r \right|, \\
		I_3 & := \sup_{0 \leq s \leq t} \left| \int_0^s (G(E_r, X_r) - G(E_{\tau_{n_r}}, X_{\tau_{n_r}}^\delta)) \mathrm{d}B_{E_r} \right|.
	\end{aligned}
	\]
	
	现在,在$I_1$的表达式中,回顾到$\tau_{n_r} \leq r < \tau_{n_r+1}$且$0 \leq E_r - E_{\tau_{n_r}} \leq (n_r+1)\delta - n_r\delta = \delta$。
	结合Cauchy-Schwartz不等式和假设$\mathcal{H}_3$,我们得到:
	\begin{equation}\label{6}
		I_1^2 \leq t \int_0^t (H(E_r) - H(E_{\tau_{n_r}}))^2 \, \mathrm{d}r \leq K^2 T^2 \delta^{2\theta}.
	\end{equation}
	
	至于$I_2$,通过Cauchy-Schwarz不等式注意到:
	
	$$
	\mathbb{E}_B[I_2^2] \leq E_t \int_0^t \mathbb{E}_B\left[ \left( F(E_r, X_r) - F(E_{\tau_{n_r}}, X_{\tau_{n_r}}^\delta) \right)^2 \right] \, \mathrm{d}E_r
	$$
	由于$\mathcal{H}_1, \mathcal{H}_3$ 和以下不等式成立:
	\begin{align*}
		|F(E_r, X_r) - F(E_{\tau_{n_r}}, X_{\tau_{n_r}}^\delta)| 
		&\leq |F(E_r, X_r) - F(E_{\tau_{n_r}}, X_r)| + |F(E_{\tau_{n_r}}, X_r) - F(E_{\tau_{n_r}}, X_{\tau_{n_r}})| \\
		&+ |F(E_{\tau_{n_r}}, X_{\tau_{n_r}}) - F(E_{\tau_{n_r}}, X_{\tau_{n_r}}^\delta)|
	\end{align*}
	则有:
	\begin{align}\label{7}
		\mathbb{E}_B[I_2^2] &\leq 3 E_T \int_0^t \mathbb{E}_B \left[ K^2 \delta^{2\theta} (1 + |X_r|)^2 + h^2(E_{\tau_{n_r}}) |X_r - X_{\tau_{n_r}}|^2 + h^2(E_{\tau_{n_r}}) Z_{\tau_{n_r}}^2 \right] \, \mathrm{d}E_r\\\nonumber
		&\leq 6 E_T^2 K^2 \delta^{2\theta} \mathbb{E}_B[Y_T^{(2)}] \\\nonumber
		&+ 3 E_T h^2(E_T) \left\{ \int_0^t \mathbb{E}_B[|X_r - X_{\tau_{n_r}}|^2] \, \mathrm{d}E_r + \int_0^t \mathbb{E}_B[Z_r^2] \, \mathrm{d}E_r \right\}.
	\end{align}
	现在,对于任意 $r \in [0, t]$,根据引理\ref{th}和 $0 \leq E_r - E_{\tau_{n_r}} \leq \delta$,我们有:
	\begin{equation}\label{8}
		\mathbb{E}_B[|X_r - X_{\tau_{n_r}}|^2] \leq 6 h^2(E_T) \mathbb{E}_B[Y_T^{(2)}] \left\{(r - \tau_{n_r})^2 + \delta + \delta^2 \right\}. 
	\end{equation}
	此外,
	\begin{align}\label{9}
		&\int_0^t (r - \tau_{n_r})^2 \, \mathrm{d}E_r = \sum_{i=0}^{n_t-1} \int_{t_i}^{\tau_{i+1}} (r - \tau_i)^2 \, \mathrm{d}E_r + \int_{\tau_{n_t}}^t (r - \tau_{n_t})^2 \, \mathrm{d}E_r\\\nonumber
		&\leq \delta \left( \sum_{i=0}^{n_t-1} (\tau_{i+1} - \tau_i)^2 + (t - \tau_{n_t})^2 \right) 
		\leq 2T \delta \left( \sum_{i=0}^{n_t-1} (\tau_{i+1} - \tau_i) + (t - \tau_{n_t}) \right)\\ \nonumber
		&\leq 2T^2 \delta.\nonumber
	\end{align}
	结合 (\ref{8}) 和 (\ref{9}) 与 (\ref{7}),并回忆起 $\delta < 1$,我们得到:
	\begin{equation}\label{10}
		\mathbb{E}_B[I_2^2] \leq \xi_1(E_T) \mathbb{E}_B[Y_T^{(2)}] \delta^{\min\{2\theta, 1\}} + 3 E_T h^2(E_T) \int_0^t \mathbb{E}_B[Z_r^2] \, \mathrm{d}E_r,
	\end{equation}
	其中 $\xi_1(u) := 36 u^2 h^4(u) + 36 u h^4(u) T^2 + 6 u^2 K^2$.
	
	关于 $I_3$,使用 BDG 不等式和与前述段落类似的计算,得出:
	\begin{equation}\label{11}
		\mathbb{E}_B[I_3^2] \leq \xi_2(E_T) \mathbb{E}_B[Y_T^{(2)}] \delta^{\min\{2\theta, 1\}} + 3 b_2 h^2(E_T) \int_0^t \mathbb{E}_B[Z_r^2] \, \mathrm{d}E_r,
	\end{equation}
	其中 $\xi_2(u) := \frac{b_2 \xi_1(u)}{u}$。
	
	将估计式 (\ref{6}), (\ref{10}) 和 (\ref{11}) 结合起来得到
	\[
	E_{B}\left[Z_{t}^{2}\right] \leq \xi_{3}\left(E_{T}\right) E_{B}\left[Y_{T}^{(2)}\right] \delta^{\min\{2\theta, 1\}} + 9\left(E_{T} + b_{2}\right) h^{2}\left(E_{T}\right) \int_{0}^{t} E_{B}\left[Z_{r}^{2}\right] d E_{r},
	\]
	其中 \(\xi_{3}(u) := 3\xi_{1}(u) + 3\xi_{2}(u) + 3 K^{2} T^{2}\)。利用第 IX 章第 6a 节中的 Gronwall 型不等式 [8] 的引理 6.3 并设置 \(t = T\),得到
	\[
	E_{B}\left[Z_{T}^{2}\right] \leq \xi_{3}\left(E_{T}\right) E_{B}\left[Y_{T}^{(2)}\right] e^{9 E_{T}\left(E_{T} + b_{2}\right) h^{2}\left(E_{T}\right)} \delta^{\min\{2\theta, 1\}}.
	\]
	对两边取 \(E_{D}\) 的期望并使用 Cauchy-Schwartz 不等式,得到
	\begin{equation}\label{12}
		E\left[Z_T^2\right] \leq E\left[\xi_3^4\left(E_T\right)\right]^{1/4} E\left[\left(Y_T^{(2)}\right)^4\right]^{1/4} E\left[e^{18 E_T\left(E_T + b_2\right) h^2\left(E_T\right)}\right]^{1/2} \delta^{\min\{2\theta, 1\}}.
	\end{equation}
	
	
	所需的结果通过证明 (\ref{12}) 右边的期望是有限的而得出。现在,假设 \(h\) 是常数。那么,根据命题 2.1 和 3.1,我们有 \(E\left[\xi_{3}^{4}\left(E_{T}\right)\right] < \infty\) 和 \(E\left[\left(Y_{T}^{(2)}\right)^{4}\right] \leq 8 E\left[Y_{T}^{(8)}\right] < \infty\)。至于 \(E\left[e^{18 E_{T}\left(E_{T} + b_{2}\right) h^{2}\left(E_{T}\right)}\right]\),由于引理 2.1,指数的形式是 \(f\left(E_{T}\right)\),其中 \(f \in RV_{2}(\infty)\),因此根据定理 2.1,该期望是有限的,当且仅当 \(2 < 1/(1-\beta)\)(即 \(\beta \in (1/2, 1)\))。
	
	另一方面,如果 \( h \in RV_{q}(\infty) \) 且 \( q \geq 0 \),根据命题 3.1,若 \( \beta \in \left( \frac{2q}{2q+1}, 1 \right) \),则 \( E\left[Y_{T}^{(8)}\right] < \infty \)。由于 \( E\left[ e^{18 E_T \left( E_T + b_2 \right) h^2 \left( E_T \right)} \right] \) 的指数具有形式 \( f\left( E_{T} \right) \),其中 \( f \in RV_{2q+2}(\infty) \),根据定理 2.1,该期望是有限的,当且仅当 \( 2q+2 < \frac{1}{1-\beta} \)(即 \( \beta \in \left( \frac{2q+1}{2q+2}, 1 \right) \))。因此,只要 \( \beta \in \left( \frac{2q+1}{2q+2}, 1 \right) \),结果即成立。
	
	\begin{remark}
		
		\begin{enumerate}
			\item 上述证明中使用的方法提供了一个一般性的思路,说明如何分析可能涉及随机时间变换的 SDE 近似方案的强收敛速度。
			\item 如果我们允许系数 \( H \) 也依赖于 \( x \),则上述证明将无法适用。事实上,这种情况将要求对 \( E_{B}\left[I_{1}^{2}\right] \) 进行类似于 (\ref{7}) 的估计,从而产生积分 \( \int_{0}^{t}E_{B}\left[\left|X_{r}-X_{\tau_{n_{r}}}\right|^{2}\right] dr \)。根据 (\ref{8}),这个积分的上界可以被包含在涉及积分 \( \int_{0}^{t}(r-\tau_{n_{r}})^{2}dr \) 的量级中。然而,与 (\ref{9}) 不同的是,后者的积分不会导致包含 \( \delta \) 的界限。
			\item 尽管我们通过 (\ref{4}) 中的连续插值对离散化过程 \( \left(X_{\tau_{n}}^{\delta}\right)_{n\in\{0,1,2,\ldots, N\}} \) 进行了插值,但在 \( H\equiv 0 \) 的情况下,采用分段常数插值 \( X_{t}^{\delta}:=X_{\tau_{n_{t}}}^{\delta} \) 也是可能的,正如 [9] 中所示。在这种情况下,\( Z_t \) 的界限将额外包含在区间 \([tn,s]\) 上的积分的上确界,包括
			\[
			I_{5}:=\sup_{0\leq s\leq t}\left|\int_{\tau_{n_{s}}}^{s}G(E_{r},X_{r})\, dB_{E_{r}}\right|.
			\]
			对 \( E_{B}[I_{5}^{2}] \) 的估计可以借助于 [4] 中关于随机积分的连续模结果来进行,然而这只会得到收敛阶数为 \( \frac{1}{2}-\varepsilon \) 的结果,其中 \( \varepsilon > 0 \)。详情请参见 [9] 中的备注 9(3)。
			\item 上述证明表明,参数 \( \beta \) 在确定 \( E[Z^2] \) 上界有限性方面起着重要作用。另一方面,\( \beta \) 对于 \( X^{\delta} \) 的收敛速度没有影响,这是因为我们构造的离散时间变化 \( E^{\delta} \) 使得无论 \( \beta \) 的值如何,\( E^{\delta} \) 都以 1 的阶数收敛到 \( E \),正如 (\ref{1}) 所示。上述论证表明,如果 \( E^{\delta} \) 的收敛速度依赖于 \( \beta \),则 \( X^{\delta} \) 的收敛速度也可能涉及 \( \beta \)。
		\end{enumerate}
	\end{remark}
	
	
	
	
	
\end{document}

